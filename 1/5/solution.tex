% SPDX-FileCopyrightText: Copyright (C) Nile Jocson <novoseiversia@gmail.com>
% SPDX-License-Identifier: MPL-2.0

\documentclass{article}

% SPDX-FileCopyrightText: Copyright (C) Nile Jocson <novoseiversia@gmail.com>
% SPDX-License-Identifier: MPL-2.0

\usepackage[a4paper, margin=1in]{geometry}

\usepackage{adjustbox}

\usepackage{amsmath}
\usepackage{amssymb}
\usepackage{physics}

\usepackage{tabularray}
\usepackage{tkz-tab}
\usepackage{xpatch}

\usepackage{pstricks-add}
\usepackage{pst-eucl}
\usepackage[crop=off]{auto-pst-pdf}



\renewcommand{\arraystretch}{1.75}
\renewcommand{\thesubsubsection}{\thesubsection.\alph{subsubsection}}



\DefTblrTemplate{caption}{default}{}
\DefTblrTemplate{capcont}{default}{}

\xpatchcmd{\tkzTabLine}{$0$}{$\bullet$}{}{}
\tikzset{t style/.style={style=solid}}



\newcommand*{\cjboilerplate}[2]{
	\psset{unit=5.5mm, ticks=none, xlabelsep=1pt, ylabelsep=1pt}

	\author{Nile Jocson \textless{}novoseiversia@gmail.com\textgreater{}}
	\title{Exercise Solutions for #1\\{\large #2}}
	\date{\today}

	\maketitle{}
	\null\vfill\noindent
	Copyright \copyright{} Nile Jocson \textless{}novoseiversia@gmail.com\textgreater{} \\
	Licensed under MPL-2.0. See LICENSE file.
		\pagebreak
}



\newenvironment{cjsection}[1]
{
	\section{#1}
}
{
	\pagebreak
}

\newcommand*{\cjitem}[1]{\subsection{#1}}
\newcommand*{\cjsubitem}[1]{\subsubsection{#1}}



\newcommand*{\cjsolsect}[1]{\hline --- #1: \\}
\newcommand*{\cjneeded}[2]{\hline --- Needed: \\ \(\square\) #1 \(= \mathord{?}\) & #2 \\}
\newcommand*{\cjgiven}[2]{\(\square\) #1 & #2 \\}

\newcommand*{\cjwhy}[2]{\hline \(\Rightarrow\) #1 & #2 \\}
\newcommand*{\cjsubwhy}[2]{\(\Rightarrow\) #1 & #2 \\}
\newcommand*{\cjcontinue}[1]{\(\Rightarrow\) #1 & \\}
\newcommand*{\cjfa}[1]{\hline \(\Rightarrow\) #1 & Final answer. \\}
\newcommand*{\cjfastep}[2]{\hline \(\Rightarrow\) #1 & Final answer. #2 \\}

\newcommand*{\cjsign}[1]{
	\hline & Create a table of signs. \\
	\begin{adjustbox}{width=0.49\textwidth}
		\begin{tikzpicture}
			#1
		\end{tikzpicture}
	\end{adjustbox} \\ \\
}

\newcommand*{\cjgraph}[3]{
	\begin{center}
		\begin{adjustbox}{width=\textwidth}
			\begin{pspicture*}(-#2,-#2)(#2,#2)
				\psaxes[labels=none]{<->}(0,0)(-#2,-#2)(#2,#2)
				#3
			\end{pspicture*}
		\end{adjustbox}
		#1
	\end{center}
	\pagebreak
}

\newcommand*{\cjsystem}[2]{
	\begin{equation*}
		#1
		\begin{cases}
			#2
		\end{cases}
	\end{equation*}
}

\newcommand*{\cjqed}{\(\blacksquare\)}



\NewDocumentEnvironment{cjsolution}{+b}
{
	\begin{longtblr}
	[
		expand = \cjwhy\cjsubwhy\cjcontinue\cjfa\cjfastep\cjsign\cjgiven\cjsolsect\cjneeded\cjsystem
	]
	{
		colspec = {|lX[r]|},
		width = \textwidth
	}
		#1
		& \cjqed{} \\
		\hline
	\end{longtblr}
}{}



\newcommand*{\cjdiv}{\divisionsymbol{}}
\newcommand*{\cjexp}[1]{\times 10^{#1}}
\newcommand*{\cjunit}[1]{\text{ #1}}
\newcommand*{\cjceil}[1]{\lceil#1\rceil}
\newcommand*{\cjlog}[2]{\text{log}_{#1} #2}




\begin{document}
	\cjboilerplate{Math 20}{Lines and Circles}

	\begin{cjsection}{}
		\cjitem{Find the value of \(k\) such that the lines with equations \(3x + 2y - 4 = 0\) and
		\(kx - 3y + 8\) are:}
			\cjsubitem{Parallel.}
				\begin{cjsolution}
					\cjwhy{\(2y = -3x + 4\)}{Rewrite the first equation in slope-intercept form.}
					\cjcontinue{\(y = -\frac{3}{2}x + 4\)}
					\cjwhy{\(-3y = -kx - 8\)}{Rewrite the second equation in slope-intercept form.}
					\cjcontinue{\(3y = kx + 8\)}
					\cjcontinue{\(y = \frac{k}{3}x + \frac{8}{3}\)}
					\cjwhy{\(\frac{k}{3} = -\frac{3}{2}\)}{Parallel slopes are equal.}
					\cjfa{\(k = -\frac{9}{2}\)}
				\end{cjsolution}

			\cjsubitem{Perpendicular.}
				\begin{cjsolution}
					\cjwhy{\(\frac{k}{3} = -\frac{1}{-\frac{3}{2}}\)}{Perpendicular slopes are the negative reciprocal of each other.}
					\cjcontinue{\(\frac{k}{3} = \frac{2}{3}\)}
					\cjfa{\(k = 2\)}
				\end{cjsolution}

		\cjitem{Line \(l\) is perpendicular to the line segment with endpoints \(P(-4, 7)\) and \(Q(2, -3)\).
		If \(l\) passes through the midpoint of the line segment \(\overline{PQ}\), find an equation for \(l\)
		in slope-intercept form.}
			\begin{cjsolution}
				\cjwhy{\(m = \frac{-3 - 7}{2 + 4}\)}{Find the slope of \(\overline{PQ}\).}
				\cjcontinue{\(m = \frac{-10}{6}\)}
				\cjcontinue{\(m = -\frac{5}{3}\)}
				\cjwhy{\(M = (\frac{-4 + 2}{2}, \frac{7 - 3}{2})\)}{Find the midpoint of \(\overline{PQ}\).}
				\cjcontinue{\(M = (\frac{-2}{2}, \frac{4}{2})\)}
				\cjcontinue{\(M = (-1, 2)\)}
				\cjwhy{\(y - 2 = -\frac{5}{3}(x + 1)\)}{Use the point-slope formula.}
				\cjcontinue{\(y - 2 = -\frac{5}{3}x - \frac{5}{3}\)}
				\cjcontinue{\(y = -\frac{5}{3}x - \frac{5}{3} + 2\)}
				\cjcontinue{\(y = -\frac{5}{3}x - \frac{5}{3} + \frac{6}{3}\)}
				\cjfa{\(y = -\frac{5}{3}x - \frac{1}{3}\)}
			\end{cjsolution}

		\cjitem{Find a general equation of the line that is parallel to the line with equation \(3x - y + 1 = 0\)
		and whose x-intercept is also the x-intercept of the line with equation \(2x - 3y + 6 = 0\)}
			\begin{cjsolution}
				\cjwhy{\(-y = -3x - 1\)}{Find the slope of the first equation.}
				\cjcontinue{\(y = 3x + 1\)}
				\cjcontinue{\(m = 3\)}
				\cjwhy{\(2x - 3(0) + 6 = 0\)}{Find the x-intercept of the second equation.}
				\cjcontinue{\(2x + 6 = 0\)}
				\cjcontinue{\(2x = -6\)}
				\cjcontinue{\(x = -3\)}
				\cjwhy{\(y - 0 = 3(x + 3)\)}{Use the point-slope formula.}
				\cjfa{\(y = 3x + 9\)}
			\end{cjsolution}

		\cjitem{From the following equations, determine the center and radius of the circle if it exists.}
			\cjsubitem{\({(x + 1)}^2 + {(y + 3)}^2 = 5\)}
				\begin{cjsolution}
					\cjfastep{\((h, k) = (-1, -3), r = \sqrt{5}\)}{Derive the center and radius.}
				\end{cjsolution}

			\cjsubitem{\(x^2 + y^2 + 8x + 7 = 0\)}
				\begin{cjsolution}
					\cjwhy{\(x^2 + y^2 + 8x = -7\)}{Isolate the constants.}
					\cjwhy{\(x^2 + 8x + 16 + y^2 = -7 + 16\)}{Complete the square.}
					\cjcontinue{\({(x + 4)}^2 + y^2 = 9\)}
					\cjcontinue{\({(x + 4)}^2 + y^2 = 3^2\)}
					\cjfa{\((h, k) = (-4, 0), r = 3\)}
				\end{cjsolution}
	\end{cjsection}
\end{document}
