% SPDX-FileCopyrightText: Copyright (C) Nile Jocson <novoseiversia@gmail.com>
% SPDX-License-Identifier: MPL-2.0

\documentclass{article}

% SPDX-FileCopyrightText: Copyright (C) Nile Jocson <novoseiversia@gmail.com>
% SPDX-License-Identifier: MPL-2.0

\usepackage{amsmath}
\usepackage{amssymb}
\usepackage[a4paper, margin=1in]{geometry}
\usepackage{hyperref}
\usepackage{physics}
\usepackage{tabularx}



\renewcommand{\arraystretch}{1.75}



\newcommand{\cjboilerplate}[2]{
	\renewcommand{\thesubsubsection}{\thesubsection.\alph{subsubsection}}

	\author{Nile Jocson \textless{}novoseiversia@gmail.com\textgreater{}}
	\title{Exercise Solutions for #1\\{\large #2}}
	\date{\today}

	\maketitle{}
		\pagebreak

	\tableofcontents{}
		\pagebreak
}



\newenvironment{cjsection}[1]
{
	\section{#1}
}
{
	\pagebreak
}

\newcommand{\cjitem}[1]{\subsection{#1}}
\newcommand{\cjsubitem}[1]{\subsubsection{#1}}



\newcommand{\cjnext}[1]{\(\Rightarrow\) #1}

\newcommand{\cjwhy}[2]{\hline \cjnext{#1} & #2 \\}
\newcommand{\cjcontinue}[1]{\cjnext{#1} & \\}

\newcommand{\cjqed}{\(\blacksquare\)}

\newenvironment{cjsolution}
{
	\tabularx{\textwidth}{| >{\raggedright\arraybackslash}X  >{\raggedleft\arraybackslash}X |}
}
{
		& \cjqed{} \\
		\hline
	\endtabularx
}



\newcommand*{\cjdiv}{\divisionsymbol{}}




\begin{document}
	\cjboilerplate{Math 20}{Conics (Parabola and Ellipse)}

	\begin{cjsection}{}
		\cjitem{Determine the vertex and orientation of the following parabolas.}
			\cjsubitem{\(4y^2 + 4y + x = 2\)}
				\begin{cjsolution}
					\cjwhy{\(4y^2 + 4y = -x + 2\)}{Isolate \(y\).}
					\cjcontinue{\(y^2 + y = -\frac{x}{4} + \frac{2}{4}\)}
					\cjcontinue{\(y^2 + y = -\frac{x}{4} + \frac{1}{2}\)}
					\cjwhy{\(y^2 + y + \frac{1}{4} = -\frac{x}{4} + \frac{1}{2} + \frac{1}{4}\)}{Complete the square.}
					\cjcontinue{\({(y + \frac{1}{2})}^2 = -\frac{x}{4} + \frac{3}{4}\)}
					\cjcontinue{\({(y + \frac{1}{2})}^2 = -\frac{1}{4}(x - 3)\)}
					\cjcontinue{\({(y + \frac{1}{2})}^2 = 4(-\frac{1}{16})(x - 3)\)}
					\cjfa{Opening leftwards, \((h, k) = (3, -\frac{1}{2})\)}
				\end{cjsolution}

			\cjsubitem{\(x^2 - 6x - 2y = 7\)}
				\begin{cjsolution}
					\cjwhy{\(x^2 - 6x = 2y + 7\)}{Isolate \(x\).}
					\cjwhy{\(x^2 - 6x + 9 = 2y + 7 + 9\)}{Complete the square.}
					\cjcontinue{\({(x - 3)}^2 = 2y + 16\)}
					\cjcontinue{\({(x - 3)}^2 = 2(y + 8)\)}
					\cjcontinue{\({(x - 3)}^2 = 4(\frac{1}{2})(y + 8)\)}
					\cjfa{Opening upwards, \((h, k) = (3, -8)\)}
				\end{cjsolution}

			\cjsubitem{\(2y^2 - 6y - 9x = 0\)}
				\begin{cjsolution}
					\cjwhy{\(2y^2 - 6y = 9x\)}{Isolate \(y\).}
					\cjcontinue{\(y^2 - 3y = \frac{9}{2}x\)}
					\cjwhy{\(y^2 - 3y + \frac{9}{4} = \frac{9}{2}x + \frac{9}{4}\)}{Complete the square.}
					\cjcontinue{\({(y - \frac{3}{2})}^2 = \frac{9}{2}x + \frac{9}{4}\)}
					\cjcontinue{\({(y - \frac{3}{2})}^2 = \frac{9}{2}(x + \frac{9}{4} \cdot \frac{2}{9})\)}
					\cjcontinue{\({(y - \frac{3}{2})}^2 = \frac{9}{2}(x + \frac{18}{36})\)}
					\cjcontinue{\({(y - \frac{3}{2})}^2 = \frac{9}{2}(x + \frac{1}{2})\)}
					\cjcontinue{\({(y - \frac{3}{2})}^2 = 4(\frac{9}{8})(x + \frac{1}{2})\)}
					\cjfa{Opening rightwards, \((h, k) = (-\frac{1}{2}, \frac{3}{2})\)}
				\end{cjsolution}

			\cjitem{Sketch the graph of the following parabolas.}
				\cjsubitem{\(3y^2 = 8x\)}
					\begin{cjsolution}
						\cjwhy{\(y^2 = 4(\frac{2}{3})x\)}{Rewrite in standard form.}
						\cjfastep{See Figure 1.}{Graph the parabola.}
					\end{cjsolution}
					\cjgraph{Figure 1. Graph of \(y^2 = 4(\frac{2}{3})x\).}{16}{
						\pstGeonode[PosAngle=40](0,0){V}
						\def\p{1.3333}
						\pstIParabola(V){\p}{-16}{16}
						\pstIParabolaFocusNode[linecolor=blue,PosAngle=-40](V){\p}{F}
						\pstIParabolaDirectrixLine[linecolor=red,nodesepA=-32,nodesepB=-32,PointName={D_1,D_2},PosAngle={225,180}](V){\p}{D1}{D2}
						\pstIParabolaAbsNode[linecolor=blue,PointName={L_1,L_2},PosAngle={-90, 90}](V){\p}{0.6667}{L1}{L2}
						\pstLine[linecolor=blue]{L1}{L2}
					}

				\cjsubitem{\(x^2 - 8x + 4y = -10\)}
					\begin{cjsolution}
						\cjwhy{\(x^2 - 8x = -4y - 10\)}{Rewrite in standard form.}
						\cjsubwhy{\(x^2 - 8x + 16 = -4y - 10 + 16\)}{Complete the square.}
						\cjcontinue{\({(x - 4)}^2 = -4y + 6\)}
						\cjcontinue{\({(x - 4)}^2 = 4(-1)(y - \frac{3}{2})\)}
						\cjfastep{See Figure 2.}{Graph the parabola.}
					\end{cjsolution}
					\cjgraph{Figure 2. Graph of \({(x - 4)}^2 = 4(-1)(y - \frac{3}{2})\).}{16}{
						\pstGeonode[PosAngle=140](4,1.5){V}
						\def\p{-2}
						\pstParabola(V){\p}{-16}{16}
						\pstParabolaFocusNode[linecolor=blue,PosAngle=40](V){\p}{F}
						\pstParabolaDirectrixLine[linecolor=red,nodesepA=-32,nodesepB=-32,PointName={D_1,D_2},PosAngle={90,90}](V){\p}{D1}{D2}
						\pstParabolaOrdNode[linecolor=blue,PointName={L_1,L_2},PosAngle={180,0}](V){\p}{0.5}{L1}{L2}
						\pstLine[linecolor=blue]{L1}{L2}
					}

			\cjitem{Sketch the graph of \(y = -x^2 + 6x - 8\). Label the vertex, x- and y-intercept(s).}
				\begin{cjsolution}
					\cjwhy{\(-x^2 + 6x = y + 8\)}{Rewrite in standard form.}
					\cjcontinue{\(x^2 - 6x = -y - 8\)}
					\cjsubwhy{\(x^2 - 6x + 9 = -y - 8 + 9\)}{Complete the square.}
					\cjcontinue{\({(x - 3)}^2 = -y + 1\)}
					\cjcontinue{\({(x - 3)}^2 = 4(-\frac{1}{4})(y - 1)\)}
					\cjwhy{\({(x - 3)}^2 = 4(-\frac{1}{4})(-1)\)}{Find the x-intercepts.}
					\cjcontinue{\({(x - 3)}^2 = 4(\frac{1}{4})\)}
					\cjcontinue{\({(x - 3)}^2 = 1\)}
					\cjcontinue{\(x = \pm 1 + 3\)}
					\cjcontinue{\(x = 1 + 3, x = -1 + 3\)}
					\cjcontinue{\(x_i \in \{2, 4\}\)}
					\cjwhy{\({(0 - 3)}^2 = 4(-\frac{1}{4})(y - 1)\)}{Find the y-intercepts.}
					\cjcontinue{\((-3)^2 = 4(-\frac{1}{4})(y - 1)\)}
					\cjcontinue{\(9 = -(y - 1)\)}
					\cjcontinue{\(9 = -y + 1\)}
					\cjcontinue{\(y = 1 - 9\)}
					\cjcontinue{\(y_i = -8\)}
					\cjfastep{See Figure 3.}{Graph the parabola.}
				\end{cjsolution}
				\cjgraph{Figure 3. Graph of \({(x - 3)}^2 = 4(-\frac{1}{4})(y - 1)\) with x- and y-intercepts.}{16}{
					\pstGeonode[PosAngle=135](3,1){V}
					\def\p{-0.5}
					\pstParabola(V){\p}{-16}{16}
					\pstParabolaFocusNode[linecolor=blue,PosAngle=-90](V){\p}{F}
					\pstParabolaDirectrixLine[linecolor=red,nodesepA=-32,nodesepB=-32,PointName={D_1,D_2},PosAngle={90,90}](V){\p}{D1}{D2}
					\pstParabolaOrdNode[linecolor=blue,PointName={L_1,L_2},PosAngle={200,-20}](V){\p}{0.75}{L1}{L2}
					\pstLine[linecolor=blue]{L1}{L2}
					\pstParabolaOrdNode[PointName={x_1,x_2},PosAngle={210,-30}](V){\p}{0}{x1}{x2}
					\pstParabolaAbsNode[PointName=y_1](V){\p}{0}{y1}
				}
	\end{cjsection}
\end{document}
