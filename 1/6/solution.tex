% SPDX-FileCopyrightText: Copyright (C) Nile Jocson <novoseiversia@gmail.com>
% SPDX-License-Identifier: MPL-2.0

\documentclass{article}

% SPDX-FileCopyrightText: Copyright (C) Nile Jocson <novoseiversia@gmail.com>
% SPDX-License-Identifier: MPL-2.0

\usepackage{amsmath}
\usepackage{amssymb}
\usepackage[a4paper, margin=1in]{geometry}
\usepackage{hyperref}
\usepackage{physics}
\usepackage{tabularx}



\renewcommand{\arraystretch}{1.75}



\newcommand{\cjboilerplate}[2]{
	\renewcommand{\thesubsubsection}{\thesubsection.\alph{subsubsection}}

	\author{Nile Jocson \textless{}novoseiversia@gmail.com\textgreater{}}
	\title{Exercise Solutions for #1\\{\large #2}}
	\date{\today}

	\maketitle{}
		\pagebreak

	\tableofcontents{}
		\pagebreak
}



\newenvironment{cjsection}[1]
{
	\section{#1}
}
{
	\pagebreak
}

\newcommand{\cjitem}[1]{\subsection{#1}}
\newcommand{\cjsubitem}[1]{\subsubsection{#1}}



\newcommand{\cjnext}[1]{\(\Rightarrow\) #1}

\newcommand{\cjwhy}[2]{\hline \cjnext{#1} & #2 \\}
\newcommand{\cjcontinue}[1]{\cjnext{#1} & \\}

\newcommand{\cjqed}{\(\blacksquare\)}

\newenvironment{cjsolution}
{
	\tabularx{\textwidth}{| >{\raggedright\arraybackslash}X  >{\raggedleft\arraybackslash}X |}
}
{
		& \cjqed{} \\
		\hline
	\endtabularx
}



\newcommand*{\cjdiv}{\divisionsymbol{}}




\begin{document}
	\cjboilerplate{Math 20}{Conics (Parabola and Ellipse)}

	\begin{cjsection}{}
		\cjitem{Determine the vertex and orientation of the following parabolas.}
			\cjsubitem{\(4y^2 + 4y + x = 2\)}
				\begin{cjsolution}
					\cjwhy{\(4y^2 + 4y = -x + 2\)}{Isolate \(y\).}
					\cjcontinue{\(y^2 + y = -\frac{x}{4} + \frac{2}{4}\)}
					\cjcontinue{\(y^2 + y = -\frac{x}{4} + \frac{1}{2}\)}
					\cjwhy{\(y^2 + y + \frac{1}{4} = -\frac{x}{4} + \frac{1}{2} + \frac{1}{4}\)}{Complete the square.}
					\cjcontinue{\({(y + \frac{1}{2})}^2 = -\frac{x}{4} + \frac{3}{4}\)}
					\cjcontinue{\({(y + \frac{1}{2})}^2 = -\frac{1}{4}(x - 3)\)}
					\cjcontinue{\({(y + \frac{1}{2})}^2 = 4(-\frac{1}{16})(x - 3)\)}
					\cjfa{Opening leftward, \((h, k) = (3, -\frac{1}{2})\)}
				\end{cjsolution}

			\cjsubitem{\(x^2 - 6x - 2y = 7\)}
				\begin{cjsolution}
					\cjwhy{\(x^2 - 6x = 2y + 7\)}{Isolate \(x\).}
					\cjwhy{\(x^2 - 6x + 9 = 2y + 7 + 9\)}{Complete the square.}
					\cjcontinue{\({(x - 3)}^2 = 2y + 16\)}
					\cjcontinue{\({(x - 3)}^2 = 2(y + 8)\)}
					\cjcontinue{\({(x - 3)}^2 = 4(\frac{1}{2})(y + 8)\)}
					\cjfa{Opening upward, \((h, k) = (3, -8)\)}
				\end{cjsolution}

			\cjsubitem{\(2y^2 - 6y - 9x = 0\)}
				\begin{cjsolution}
					\cjwhy{\(2y^2 - 6y = 9x\)}{Isolate \(y\).}
					\cjcontinue{\(y^2 - 3y = \frac{9}{2}x\)}
					\cjwhy{\(y^2 - 3y + \frac{9}{4} = \frac{9}{2}x + \frac{9}{4}\)}{Complete the square.}
					\cjcontinue{\({(y - \frac{3}{2})}^2 = \frac{9}{2}x + \frac{9}{4}\)}
					\cjcontinue{\({(y - \frac{3}{2})}^2 = \frac{9}{2}(x + \frac{9}{4} \cdot \frac{2}{9})\)}
					\cjcontinue{\({(y - \frac{3}{2})}^2 = \frac{9}{2}(x + \frac{18}{36})\)}
					\cjcontinue{\({(y - \frac{3}{2})}^2 = \frac{9}{2}(x + \frac{1}{2})\)}
					\cjcontinue{\({(y - \frac{3}{2})}^2 = 4(\frac{9}{8})(x + \frac{1}{2})\)}
					\cjfa{Opening rightward, \((h, k) = (-\frac{1}{2}, \frac{3}{2})\)}
				\end{cjsolution}

		\cjitem{Sketch the graph of the following parabolas.}
			\cjsubitem{\(3y^2 = 8x\)}
				\begin{cjsolution}
					\cjwhy{\(y^2 = 4(\frac{2}{3})x\)}{Rewrite in standard form.}
					\cjwhy{\(V = (0, 0)\)}{Identify important objects; this is a parabola opening rightward. \(V = (h, k)\)}
					\cjsubwhy{\(F = (\frac{2}{3}, 0)\)}{\(F = (h + p, k)\)}
					\cjsubwhy{\(B_1 = (\frac{2}{3}, -\frac{4}{3})\)}{\(B_1 = (h + p, k - 2p)\)}
					\cjsubwhy{\(B_2 = (\frac{2}{3}, \frac{4}{3})\)}{\(B_2 = (h + p, k + 2p)\)}
					\cjsubwhy{\(D \Rightarrow x = -\frac{2}{3}\)}{\(D \Rightarrow x = h - p\)}
					\cjfastep{See Figure 1.}{Graph the parabola.}
				\end{cjsolution}
				\cjgraph{Figure 1. Graph of \(y^2 = 4(\frac{2}{3})x\).}{16}{
					\pstGeonode[PosAngle=40](0,0){V}
					\def\p{1.3333}
					\pstIParabola(V){\p}{-16}{16}
					\pstIParabolaFocusNode[linecolor=blue,PosAngle=-40](V){\p}{F}
					\pstIParabolaDirectrixLine[linecolor=red,nodesepA=-32,nodesepB=-32,PointName={D_1,D_2},PosAngle={225,180}](V){\p}{D1}{D2}
					\pstIParabolaAbsNode[linecolor=blue,PointName={B_1,B_2},PosAngle={-90, 90}](V){\p}{0.6667}{B1}{B2}
					\pstLine[linecolor=blue]{B1}{B2}
				}

			\cjsubitem{\(x^2 - 8x + 4y = -10\)}
				\begin{cjsolution}
					\cjwhy{\(x^2 - 8x = -4y - 10\)}{Rewrite in standard form.}
					\cjsubwhy{\(x^2 - 8x + 16 = -4y - 10 + 16\)}{Complete the square.}
					\cjcontinue{\({(x - 4)}^2 = -4y + 6\)}
					\cjcontinue{\({(x - 4)}^2 = 4(-1)(y - \frac{3}{2})\)}
					\cjwhy{\(V = (4, \frac{3}{2})\)}{Identify important objects; this is a parabola opening downward. \(V = (h, k)\)}
					\cjsubwhy{\(F = (4, \frac{1}{2})\)}{\(F = (h, k + p)\)}
					\cjsubwhy{\(B_1 = (6, \frac{1}{2})\)}{\(B_1 = (h - 2p, k + p)\)}
					\cjsubwhy{\(B_2 = (2, \frac{1}{2})\)}{\(B_2 = (h + 2p, k + p)\)}
					\cjsubwhy{\(D \Rightarrow y = \frac{5}{2}\)}{\(D \Rightarrow y = k - p\)}
					\cjfastep{See Figure 2.}{Graph the parabola.}
				\end{cjsolution}
				\cjgraph{Figure 2. Graph of \({(x - 4)}^2 = 4(-1)(y - \frac{3}{2})\).}{16}{
					\pstGeonode[PosAngle=140](4,1.5){V}
					\def\p{-2}
					\pstParabola(V){\p}{-16}{16}
					\pstParabolaFocusNode[linecolor=blue,PosAngle=40](V){\p}{F}
					\pstParabolaDirectrixLine[linecolor=red,nodesepA=-32,nodesepB=-32,PointName={D_1,D_2},PosAngle={90,90}](V){\p}{D1}{D2}
					\pstParabolaOrdNode[linecolor=blue,PointName={B_1,B_2},PosAngle={180,0}](V){\p}{0.5}{B1}{B2}
					\pstLine[linecolor=blue]{B1}{B2}
				}

		\cjitem{Sketch the graph of \(y = -x^2 + 6x - 8\). Label the vertex, x- and y-intercept(s).}
			\begin{cjsolution}
				\cjwhy{\(-x^2 + 6x = y + 8\)}{Rewrite in standard form.}
				\cjcontinue{\(x^2 - 6x = -y - 8\)}
				\cjsubwhy{\(x^2 - 6x + 9 = -y - 8 + 9\)}{Complete the square.}
				\cjcontinue{\({(x - 3)}^2 = -y + 1\)}
				\cjcontinue{\({(x - 3)}^2 = 4(-\frac{1}{4})(y - 1)\)}
				\cjwhy{\({(x - 3)}^2 = 4(-\frac{1}{4})(-1)\)}{Find the x-intercepts.}
				\cjcontinue{\({(x - 3)}^2 = 4(\frac{1}{4})\)}
				\cjcontinue{\({(x - 3)}^2 = 1\)}
				\cjcontinue{\(x - 3 = \pm 1\)}
				\cjcontinue{\(x = \pm 1 + 3\)}
				\cjcontinue{\(x = 1 + 3, x = -1 + 3\)}
				\cjcontinue{\(x_i \in \{2, 4\}\)}
				\cjwhy{\({(0 - 3)}^2 = 4(-\frac{1}{4})(y - 1)\)}{Find the y-intercepts.}
				\cjcontinue{\((-3)^2 = 4(-\frac{1}{4})(y - 1)\)}
				\cjcontinue{\(9 = -(y - 1)\)}
				\cjcontinue{\(9 = -y + 1\)}
				\cjcontinue{\(y = 1 - 9\)}
				\cjcontinue{\(y_i = -8\)}
				\cjfastep{See Figure 3.}{Graph the parabola.}
			\end{cjsolution}
			\cjgraph{Figure 3. Graph of \({(x - 3)}^2 = 4(-\frac{1}{4})(y - 1)\) with x- and y-intercepts.}{16}{
				\pstGeonode[PosAngle=135](3,1){V}
				\def\p{-0.5}
				\pstParabola(V){\p}{-16}{16}
				\pstParabolaOrdNode[PointName={x_1,x_2},PosAngle={210,-30}](V){\p}{0}{x1}{x2}
				\pstParabolaAbsNode[PointName=y_1](V){\p}{0}{y1}
			}
	\end{cjsection}



	\begin{cjsection}{}
		\cjitem{Sketch the graph of the following ellipses.}
			\cjsubitem{\(\frac{x^2}{4} + \frac{{(y - 1)}^2}{9} = 1\)}
				\begin{cjsolution}
					\cjwhy{\(\frac{x^2}{2^2} + \frac{{(y - 1)}^2}{3^2} = 1\)}{Rewrite in standard form.}
					\cjwhy{\(C = (0, 1)\)}{Identify important objects; this is an ellipse with a vertical major axis. \(C = (h, k)\)}
					\cjsubwhy{\(c = \sqrt{5}\)}{\(c = \sqrt{b^2 - a^2}\)}
					\cjsubwhy{\(V_1 = (0, -2)\)}{\(V_1 = (h, k - b)\)}
					\cjsubwhy{\(V_2 = (0, 4)\)}{\(V_2 = (h, k + b)\)}
					\cjsubwhy{\(B_1 = (-2, 1)\)}{\(B_1 = (h - a, k)\)}
					\cjsubwhy{\(B_2 = (2, 1)\)}{\(B_2 = (h + a, k)\)}
					\cjsubwhy{\(F_1 = (0, 1 - \sqrt{5})\)}{\(F_1 = (h, k - c)\)}
					\cjsubwhy{\(F_2 = (0, 1 + \sqrt{5})\)}{\(F_2 = (h, k + c)\)}
					\cjfastep{See Figure 4.}{Graph the ellipse.}
				\end{cjsolution}
				\cjgraph{Figure 4. Graph of \(\frac{x^2}{2^2} + \frac{{(y - 1)}^2}{3^2} = 1\).}{16}{
					\pstGeonode[linecolor=red](0, 1){C}
					\def\a{2}
					\def\b{3}
					\pstEllipse(C)(\a,\b)
					\pstEllipseFocusNode[linecolor=red,PointName={F_1,F_2}](C)(\a,\b){F1}{F2}
					\pstEllipseNode[linecolor=blue,PointName=V_1,PosAngle=45](C)(\a,\b){90}{V1}
					\pstEllipseNode[linecolor=blue,PointName=V_2,PosAngle=-45](C)(\a,\b){-90}{V2}
					\pstEllipseNode[linecolor=blue,PointName=B_1](C)(\a,\b){180}{B1}
					\pstEllipseNode[linecolor=blue,PointName=B_2](C)(\a,\b){0}{B2}
				}

			\cjsubitem{\(\frac{{(x - 3)}^2}{25} + \frac{y^2 + 4y + 4}{9} = 1\)}
				\begin{cjsolution}
					\cjwhy{\(\frac{{(x - 3)}^2}{25} + \frac{{(y + 2)}^2}{9} = 1\)}{Factor by grouping.}
					\cjsubwhy{\(\frac{{(x - 3)}^2}{5^2} + \frac{{(y + 2)}^2}{3^2} = 1\)}{Rewrite in standard form.}
					\cjwhy{\(C = (3, -2)\)}{Identify important objects; this is an ellipse with a horizontal major axis. \(C = (h, k)\)}
					\cjsubwhy{\(c = 4\)}{\(c = \sqrt{a^2 - b^2}\)}
					\cjsubwhy{\(V_1 = (-2, -2)\)}{\(V_1 = (h - a, k)\)}
					\cjsubwhy{\(V_2 = (8, -2)\)}{\(V_2 = (h + a, k)\)}
					\cjsubwhy{\(B_1 = (3, -5)\)}{\(B_1 = (h, k - b)\)}
					\cjsubwhy{\(B_2 = (3, 1)\)}{\(B_2 = (h, k + b)\)}
					\cjsubwhy{\(F_1 = (-1, -2)\)}{\(F_1 = (h - c, k)\)}
					\cjsubwhy{\(F_2 = (7, -2)\)}{\(F_2 = (h + c, k)\)}
					\cjfastep{See Figure 5.}{Graph the ellipse.}
				\end{cjsolution}
				\cjgraph{Figure 5. Graph of \(\frac{{(x - 3)}^2}{5^2} + \frac{{(y + 2)}^2}{3^2} = 1\).}{16}{
					\pstGeonode[linecolor=red](3, -2){C}
					\def\a{5}
					\def\b{3}
					\pstEllipse(C)(\a,\b)
					\pstEllipseFocusNode[linecolor=red,PointName={F_1,F_2},PosAngle={0,180}](C)(\a,\b){F1}{F2}
					\pstEllipseNode[linecolor=blue,PointName=V_1,PosAngle=180](C)(\a,\b){180}{V1}
					\pstEllipseNode[linecolor=blue,PointName=V_2,PosAngle=0](C)(\a,\b){0}{V2}
					\pstEllipseNode[linecolor=blue,PointName=B_1,PosAngle=90](C)(\a,\b){90}{B1}
					\pstEllipseNode[linecolor=blue,PointName=B_2,PosAngle=-90](C)(\a,\b){-90}{B2}
				}

			\cjsubitem{\(2x^2 + 3y^2 + 16x - 18y = 13\)}
				\begin{cjsolution}
					\cjwhy{\(2x^2 + 16x + 3y^2 - 18y = 13\)}{Group terms.}
					\cjcontinue{\(2(x^2 + 8x) + 3(y^2 - 6y) = 13\)}
					\cjsubwhy{\(2(x^2 + 8x + 16) + 3(y^2 - 6y) = 13 + 2(16)\)}{Complete the square.}
					\cjcontinue{\(2(x^2 + 8x + 16) + 3(y^2 - 6y) = 13 + 32\)}
					\cjcontinue{\(2(x^2 + 8x + 16) + 3(y^2 - 6y) = 45\)}
					\cjcontinue{\(2{(x + 4)}^2 + 3(y^2 - 6y) = 45\)}
					\cjsubwhy{\(2{(x + 4)}^2 + 3(y^2 - 6y + 9) = 45 + 3(9)\)}{Complete the square.}
					\cjcontinue{\(2{(x + 4)}^2 + 3(y^2 - 6y + 9) = 45 + 27\)}
					\cjcontinue{\(2{(x + 4)}^2 + 3(y^2 - 6y + 9) = 72\)}
					\cjcontinue{\(2{(x + 4)}^2 + 3{(y - 3)}^2 = 72\)}
					\cjcontinue{\(\frac{2{(x + 4)}^2}{72} + \frac{3{(y - 3)}^2}{72} = 1\)}
					\cjcontinue{\(\frac{{(x + 4)}^2}{36} + \frac{{(y - 3)}^2}{24} = 1\)}
					\cjsubwhy{\(\frac{{(x + 4)}^2}{6^2} + \frac{{(y - 3)}^2}{{(\sqrt{24})}^2} = 1\)}{Rewrite in standard form.}
					\cjcontinue{\(\frac{{(x + 4)}^2}{6^2} + \frac{{(y - 3)}^2}{{(\sqrt{4}\sqrt{6})}^2} = 1\)}
					\cjcontinue{\(\frac{{(x + 4)}^2}{6^2} + \frac{{(y - 3)}^2}{{(2\sqrt{6})}^2} = 1\)}
					\cjwhy{\(C = (-4, 3)\)}{Identify important objects; this is an ellipse with a horizontal major axis. \(C = (h, k)\)}
					\cjsubwhy{\(c = 2\sqrt{3}\)}{\(c = \sqrt{a^2 - b^2}\)}
					\cjsubwhy{\(V_1 = (-10, 3)\)}{\(V_1 = (h - a, k)\)}
					\cjsubwhy{\(V_2 = (2, 3)\)}{\(V_2 = (h + a, k)\)}
					\cjsubwhy{\(B_1 = (-4, 3 - 2\sqrt{6})\)}{\(B_1 = (h, k - b)\)}
					\cjsubwhy{\(B_2 = (-4, 3 + 2\sqrt{6})\)}{\(B_2 = (h, k + b)\)}
					\cjsubwhy{\(F_1 = (-4 - 2\sqrt{3}, 3)\)}{\(F_1 = (h - c, k)\)}
					\cjsubwhy{\(F_2 = (-4 + 2\sqrt{3}, 3)\)}{\(F_2 = (h + c, k)\)}
					\cjfastep{See Figure 6.}{Graph the ellipse.}
				\end{cjsolution}
				\cjgraph{Figure 6. Graph of \(\frac{{(x + 4)}^2}{6^2} + \frac{{(y - 3)}^2}{{(2\sqrt{6})}^2} = 1\)}{16}{
					\pstGeonode[linecolor=red](-4, 3){C}
					\def\a{6}
					\def\b{4.8990}
					\pstEllipse(C)(\a,\b)
					\pstEllipseFocusNode[linecolor=red,PointName={F_1,F_2},PosAngle={0,180}](C)(\a,\b){F1}{F2}
					\pstEllipseNode[linecolor=blue,PointName=V_1,PosAngle=180](C)(\a,\b){180}{V1}
					\pstEllipseNode[linecolor=blue,PointName=V_2,PosAngle=0](C)(\a,\b){0}{V2}
					\pstEllipseNode[linecolor=blue,PointName=B_1,PosAngle=90](C)(\a,\b){90}{B1}
					\pstEllipseNode[linecolor=blue,PointName=B_2,PosAngle=-90](C)(\a,\b){-90}{B2}
				}

		\cjitem{Find an equation of the parabola that opens downward and whose vertex and focus are
		the vertices of the ellipse \(4{(x - 2)}^2 + {(y + 1)}^2 = 1\)}.
			\begin{cjsolution}
				\cjwhy{\(\frac{{(x - 2)}^2}{\frac{1}{4}} + \frac{{(y + 1)}^2}{1} = 1\)}{Rewrite in standard form.}
				\cjsubwhy{\(\frac{{(x - 2)}^2}{{\frac{1}{2}}^2} + \frac{{(y + 1)}^2}{1^2} = 1\)}{Since \(a < b\), this is an ellipse with a vertical major axis.}
				\cjsubwhy{\(V_1 = (2, -2)\)}{\(V_1 = (h, k - b)\)}
				\cjsubwhy{\(V_2 = (2, 0)\)}{\(V_2 = (h, k + b)\)}
				\cjwhy{\(V = (2, 0), F = (2, -2)\)}{Derive the vertex and focus. Since this is a parabola opening downward, the lower point is the focus.}
				\cjsubwhy{\(0 + p = -2\)}{Derive \(p\); \(F = (h, k + p)\)}
				\cjcontinue{\(p = -2\)}
				\cjfastep{\({(x - 2)}^2 = 4(-2)y\)}{Write the parabola equation using \((h, k)\) and \(p\).}
			\end{cjsolution}
	\end{cjsection}
\end{document}
