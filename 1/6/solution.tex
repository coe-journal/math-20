% SPDX-FileCopyrightText: Copyright (C) Nile Jocson <novoseiversia@gmail.com>
% SPDX-License-Identifier: MPL-2.0

\documentclass{article}

% SPDX-FileCopyrightText: Copyright (C) Nile Jocson <novoseiversia@gmail.com>
% SPDX-License-Identifier: MPL-2.0

\usepackage[a4paper, margin=1in]{geometry}

\usepackage{adjustbox}

\usepackage{amsmath}
\usepackage{amssymb}
\usepackage{physics}

\usepackage{tabularray}
\usepackage{tkz-tab}
\usepackage{xpatch}

\usepackage{pstricks-add}
\usepackage{pst-eucl}
\usepackage[crop=off]{auto-pst-pdf}



\renewcommand{\arraystretch}{1.75}
\renewcommand{\thesubsubsection}{\thesubsection.\alph{subsubsection}}



\DefTblrTemplate{caption}{default}{}
\DefTblrTemplate{capcont}{default}{}

\xpatchcmd{\tkzTabLine}{$0$}{$\bullet$}{}{}
\tikzset{t style/.style={style=solid}}



\newcommand*{\cjboilerplate}[2]{
	\psset{unit=5.5mm, ticks=none, xlabelsep=1pt, ylabelsep=1pt}

	\author{Nile Jocson \textless{}novoseiversia@gmail.com\textgreater{}}
	\title{Exercise Solutions for #1\\{\large #2}}
	\date{\today}

	\maketitle{}
	\null\vfill\noindent
	Copyright \copyright{} Nile Jocson \textless{}novoseiversia@gmail.com\textgreater{} \\
	Licensed under MPL-2.0. See LICENSE file.
		\pagebreak
}



\newenvironment{cjsection}[1]
{
	\section{#1}
}
{
	\pagebreak
}

\newcommand*{\cjitem}[1]{\subsection{#1}}
\newcommand*{\cjsubitem}[1]{\subsubsection{#1}}



\newcommand*{\cjsolsect}[1]{\hline --- #1: \\}
\newcommand*{\cjneeded}[2]{\hline --- Needed: \\ \(\square\) #1 \(= \mathord{?}\) & #2 \\}
\newcommand*{\cjgiven}[2]{\(\square\) #1 & #2 \\}

\newcommand*{\cjwhy}[2]{\hline \(\Rightarrow\) #1 & #2 \\}
\newcommand*{\cjsubwhy}[2]{\(\Rightarrow\) #1 & #2 \\}
\newcommand*{\cjcontinue}[1]{\(\Rightarrow\) #1 & \\}
\newcommand*{\cjfa}[1]{\hline \(\Rightarrow\) #1 & Final answer. \\}
\newcommand*{\cjfastep}[2]{\hline \(\Rightarrow\) #1 & Final answer. #2 \\}

\newcommand*{\cjsign}[1]{
	\hline & Create a table of signs. \\
	\begin{adjustbox}{width=0.49\textwidth}
		\begin{tikzpicture}
			#1
		\end{tikzpicture}
	\end{adjustbox} \\ \\
}

\newcommand*{\cjgraph}[3]{
	\begin{center}
		\begin{adjustbox}{width=\textwidth}
			\begin{pspicture*}(-#2,-#2)(#2,#2)
				\psaxes[labels=none]{<->}(0,0)(-#2,-#2)(#2,#2)
				#3
			\end{pspicture*}
		\end{adjustbox}
		#1
	\end{center}
	\pagebreak
}

\newcommand*{\cjsystem}[2]{
	\begin{equation*}
		#1
		\begin{cases}
			#2
		\end{cases}
	\end{equation*}
}

\newcommand*{\cjqed}{\(\blacksquare\)}



\NewDocumentEnvironment{cjsolution}{+b}
{
	\begin{longtblr}
	[
		expand = \cjwhy\cjsubwhy\cjcontinue\cjfa\cjfastep\cjsign\cjgiven\cjsolsect\cjneeded\cjsystem
	]
	{
		colspec = {|lX[r]|},
		width = \textwidth
	}
		#1
		& \cjqed{} \\
		\hline
	\end{longtblr}
}{}



\newcommand*{\cjdiv}{\divisionsymbol{}}
\newcommand*{\cjexp}[1]{\times 10^{#1}}
\newcommand*{\cjunit}[1]{\text{ #1}}
\newcommand*{\cjceil}[1]{\lceil#1\rceil}
\newcommand*{\cjlog}[2]{\text{log}_{#1} #2}




\begin{document}
	\cjboilerplate{Math 20}{Conics (Parabola and Ellipse)}

	\begin{cjsection}{}
		\cjitem{Determine the vertex and orientation of the following parabolas.}
			\cjsubitem{\(4y^2 + 4y + x = 2\)}
				\begin{cjsolution}
					\cjwhy{\(4y^2 + 4y = -x + 2\)}{Isolate \(y\).}
					\cjcontinue{\(y^2 + y = -\frac{x}{4} + \frac{2}{4}\)}
					\cjcontinue{\(y^2 + y = -\frac{x}{4} + \frac{1}{2}\)}
					\cjwhy{\(y^2 + y + \frac{1}{4} = -\frac{x}{4} + \frac{1}{2} + \frac{1}{4}\)}{Complete the square.}
					\cjcontinue{\({(y + \frac{1}{2})}^2 = -\frac{x}{4} + \frac{3}{4}\)}
					\cjcontinue{\({(y + \frac{1}{2})}^2 = -\frac{1}{4}(x - 3)\)}
					\cjcontinue{\({(y + \frac{1}{2})}^2 = 4(-\frac{1}{16})(x - 3)\)}
					\cjfa{Opening leftwards, \((h, k) = (3, -\frac{1}{2})\)}
				\end{cjsolution}

			\cjsubitem{\(x^2 - 6x - 2y = 7\)}
				\begin{cjsolution}
					\cjwhy{\(x^2 - 6x = 2y + 7\)}{Isolate \(x\).}
					\cjwhy{\(x^2 - 6x + 9 = 2y + 7 + 9\)}{Complete the square.}
					\cjcontinue{\({(x - 3)}^2 = 2y + 16\)}
					\cjcontinue{\({(x - 3)}^2 = 2(y + 8)\)}
					\cjcontinue{\({(x - 3)}^2 = 4(\frac{1}{2})(y + 8)\)}
					\cjfa{Opening upwards, \((h, k) = (3, -8)\)}
				\end{cjsolution}

			\cjsubitem{\(2y^2 - 6y - 9x = 0\)}
				\begin{cjsolution}
					\cjwhy{\(2y^2 - 6y = 9x\)}{Isolate \(y\).}
					\cjcontinue{\(y^2 - 3y = \frac{9}{2}x\)}
					\cjwhy{\(y^2 - 3y + \frac{9}{4} = \frac{9}{2}x + \frac{9}{4}\)}{Complete the square.}
					\cjcontinue{\({(y - \frac{3}{2})}^2 = \frac{9}{2}x + \frac{9}{4}\)}
					\cjcontinue{\({(y - \frac{3}{2})}^2 = \frac{9}{2}(x + \frac{9}{4} \cdot \frac{2}{9})\)}
					\cjcontinue{\({(y - \frac{3}{2})}^2 = \frac{9}{2}(x + \frac{18}{36})\)}
					\cjcontinue{\({(y - \frac{3}{2})}^2 = \frac{9}{2}(x + \frac{1}{2})\)}
					\cjcontinue{\({(y - \frac{3}{2})}^2 = 4(\frac{9}{8})(x + \frac{1}{2})\)}
					\cjfa{Opening rightwards, \((h, k) = (-\frac{1}{2}, \frac{3}{2})\)}
				\end{cjsolution}

			\cjitem{Sketch the graph of the following parabolas.}
				\cjsubitem{\(3y^2 = 8x\)}
					\begin{cjsolution}
						\cjwhy{\(y^2 = 4(\frac{2}{3})x\)}{Rewrite in standard form.}
						\cjgraph{
							\addplot+ [smooth] coordinates {
								(8, -4.61880)
								(4, -3.26598)
								(2, -2.30940)
								(1, -1.63299)
								(0, 0)
								(1, 1.63299)
								(2, 2.30940)
								(4, 3.26598)
								(8, 4.61880)
							};
							\addplot+ [smooth] coordinates {
								(-2/3, -6.53197)
								(-2/3, 6.53197)
							};
							\addplot+ [smooth] coordinates {
								(2/3, 0)
							};
							\addplot+ [smooth] coordinates {
								(2/3, -4/3)
								(2/3, 4/3)
							};
						}
					\end{cjsolution}

				\cjsubitem{\(x^2 - 8x + 4y = -10\)}
					\begin{cjsolution}
						\cjwhy{\(x^2 - 8x = -4y - 10\)}{Rewrite in standard form.}
						\cjsubwhy{\(x^2 - 8x + 16 = -4y - 10 + 16\)}{Complete the square.}
						\cjcontinue{\({(x - 4)}^2 = -4y + 6\)}
						\cjcontinue{\({(x - 4)}^2 = 4(-1)(y - \frac{3}{2})\)}
						\cjgraph{
							\addplot+ [smooth] coordinates {
								(-1.65685, 3/2-8)
								(0, 3/2-4)
								(1.17157, 3/2-2)
								(2, 3/2-1)
								(4, 3/2)
								(6, 3/2-1)
								(6.82842, 3/2-2)
								(8, 3/2-4)
								(9.65675, 3/2-8)
							};
							\addplot+ [smooth] coordinates {
								(-1.65685, 3/2+1)
								(9.65675, 3/2+1)
							};
							\addplot+ [smooth] coordinates {
								(4, 3/2-1)
							};
							\addplot+ [smooth] coordinates {
								(4 - 2, 3/2-1)
								(4 + 2, 3/2-1)
							};
						}
					\end{cjsolution}
	\end{cjsection}
\end{document}
