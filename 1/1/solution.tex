% SPDX-FileCopyrightText: Copyright (C) Nile Jocson <novoseiversia@gmail.com>
% SPDX-License-Identifier: MPL-2.0

\documentclass{article}

% SPDX-FileCopyrightText: Copyright (C) Nile Jocson <novoseiversia@gmail.com>
% SPDX-License-Identifier: MPL-2.0

\usepackage[a4paper, margin=1in]{geometry}

\usepackage{adjustbox}

\usepackage{amsmath}
\usepackage{amssymb}
\usepackage{physics}

\usepackage{tabularray}
\usepackage{tkz-tab}
\usepackage{xpatch}

\usepackage{pstricks-add}
\usepackage{pst-eucl}
\usepackage[crop=off]{auto-pst-pdf}



\renewcommand{\arraystretch}{1.75}
\renewcommand{\thesubsubsection}{\thesubsection.\alph{subsubsection}}



\DefTblrTemplate{caption}{default}{}
\DefTblrTemplate{capcont}{default}{}

\xpatchcmd{\tkzTabLine}{$0$}{$\bullet$}{}{}
\tikzset{t style/.style={style=solid}}



\newcommand*{\cjboilerplate}[2]{
	\psset{unit=5.5mm, ticks=none, xlabelsep=1pt, ylabelsep=1pt}

	\author{Nile Jocson \textless{}novoseiversia@gmail.com\textgreater{}}
	\title{Exercise Solutions for #1\\{\large #2}}
	\date{\today}

	\maketitle{}
	\null\vfill\noindent
	Copyright \copyright{} Nile Jocson \textless{}novoseiversia@gmail.com\textgreater{} \\
	Licensed under MPL-2.0. See LICENSE file.
		\pagebreak
}



\newenvironment{cjsection}[1]
{
	\section{#1}
}
{
	\pagebreak
}

\newcommand*{\cjitem}[1]{\subsection{#1}}
\newcommand*{\cjsubitem}[1]{\subsubsection{#1}}



\newcommand*{\cjsolsect}[1]{\hline --- #1: \\}
\newcommand*{\cjneeded}[2]{\hline --- Needed: \\ \(\square\) #1 \(= \mathord{?}\) & #2 \\}
\newcommand*{\cjgiven}[2]{\(\square\) #1 & #2 \\}

\newcommand*{\cjwhy}[2]{\hline \(\Rightarrow\) #1 & #2 \\}
\newcommand*{\cjsubwhy}[2]{\(\Rightarrow\) #1 & #2 \\}
\newcommand*{\cjcontinue}[1]{\(\Rightarrow\) #1 & \\}
\newcommand*{\cjfa}[1]{\hline \(\Rightarrow\) #1 & Final answer. \\}
\newcommand*{\cjfastep}[2]{\hline \(\Rightarrow\) #1 & Final answer. #2 \\}

\newcommand*{\cjsign}[1]{
	\hline & Create a table of signs. \\
	\begin{adjustbox}{width=0.49\textwidth}
		\begin{tikzpicture}
			#1
		\end{tikzpicture}
	\end{adjustbox} \\ \\
}

\newcommand*{\cjgraph}[3]{
	\begin{center}
		\begin{adjustbox}{width=\textwidth}
			\begin{pspicture*}(-#2,-#2)(#2,#2)
				\psaxes[labels=none]{<->}(0,0)(-#2,-#2)(#2,#2)
				#3
			\end{pspicture*}
		\end{adjustbox}
		#1
	\end{center}
	\pagebreak
}

\newcommand*{\cjsystem}[2]{
	\begin{equation*}
		#1
		\begin{cases}
			#2
		\end{cases}
	\end{equation*}
}

\newcommand*{\cjqed}{\(\blacksquare\)}



\NewDocumentEnvironment{cjsolution}{+b}
{
	\begin{longtblr}
	[
		expand = \cjwhy\cjsubwhy\cjcontinue\cjfa\cjfastep\cjsign\cjgiven\cjsolsect\cjneeded\cjsystem
	]
	{
		colspec = {|lX[r]|},
		width = \textwidth
	}
		#1
		& \cjqed{} \\
		\hline
	\end{longtblr}
}{}



\newcommand*{\cjdiv}{\divisionsymbol{}}
\newcommand*{\cjexp}[1]{\times 10^{#1}}
\newcommand*{\cjunit}[1]{\text{ #1}}
\newcommand*{\cjceil}[1]{\lceil#1\rceil}
\newcommand*{\cjlog}[2]{\text{log}_{#1} #2}




\begin{document}
	\cjboilerplate{Math 20}{Linear, Quadratic, and Rational Equations}

	\begin{cjsection}{}
		\cjitem{Find the solution set of the following equations.}
			\cjsubitem{\(x + 9 = 5 - 3x\)}
				\begin{cjsolution}
					\cjwhy{\(x + 3x = 5 - 9\)}{Solve for \(x\).}
					\cjcontinue{\(4x = -4\)}
					\cjcontinue{\(x = -1\)}
				\end{cjsolution}

			\cjsubitem{\(\frac{2x + 3}{4} - \frac{x - 1}{2} = -\frac{1}{3}\)}
				\begin{cjsolution}
					\cjwhy{\(\frac{3(2x + 3)}{12} - \frac{6(x - 1)}{12} = -\frac{4}{12}\)}{\(\text{LCM} = 12\)}
					\cjcontinue{\(\frac{6x + 9}{12} - \frac{6x - 6}{12} = -\frac{4}{12}\)}
					\cjcontinue{\((6x + 9) - (6x - 6) = -4\)}
					\cjcontinue{\(6x + 9 - 6x + 6 = -4\)}
					\cjcontinue{\(15 = -4\)}
					\cjcontinue{\(x = \emptyset\)}
				\end{cjsolution}

			\cjsubitem{\(3x^2 - 2x + 1 = 0\)}
				\begin{cjsolution}
					\cjwhy{\(\frac{-(-2) \pm \sqrt{{(-2)}^2 - 4(3)(1)}}{2(3)}\)}{Use the quadratic formula.}
					\cjcontinue{\(\frac{2 \pm \sqrt{-8}}{6}\)}
					\cjcontinue{\(\frac{2 \pm \sqrt{4}\sqrt{-2}}{6}\)}
					\cjcontinue{\(\frac{2 \pm 2i\sqrt{2}}{6}\)}
					\cjcontinue{\(\frac{1 \pm i\sqrt{2}}{3}\)}
					\cjcontinue{\(\frac{1}{3} \pm \frac{\sqrt{2}}{3}i\)}
					\cjcontinue{\(x = \{\frac{1}{3} + \frac{\sqrt{2}}{3}i, \frac{1}{3} - \frac{\sqrt{2}}{3}i\}\)}
				\end{cjsolution}

			\cjsubitem{\(4x^2 + 2x = 2\)}
				\begin{cjsolution}
					\cjwhy{\(4x^2 + 2x - 2 = 0\)}{Rewrite in standard form.}
					\cjcontinue{\(2x^2 + x - 1 = 0\)}
					\cjwhy{\(2x^2 + 2x - x - 1 = 0\)}{Factor by grouping.}
					\cjcontinue{\(2x(x + 1) - 1(x + 1) = 0\)}
					\cjcontinue{\((2x - 1)(x + 1) = 0\)}
					\cjcontinue{\(x = \{-1, \frac{1}{2}\}\)}
				\end{cjsolution}

			\cjsubitem{\(16x^2 + 9 = 24x\)}
				\begin{cjsolution}
					\cjwhy{\(16x^2 - 24x + 9 = 0\)}{Rewrite in standard form.}
					\cjwhy{\(16x^2 - 12x - 12x + 9 = 0\)}{Factor by grouping.}
					\cjcontinue{\(4x(4x - 3) - 3(4x - 3) = 0\)}
					\cjcontinue{\({(4x - 3)}^2 = 0\)}
					\cjcontinue{\(x = \frac{3}{4}\)}
				\end{cjsolution}

			\cjsubitem{\(\frac{x}{x - 1} + \frac{x - 5}{x^2 + 2x - 3} = \frac{1}{x + 3}\)}
				\begin{cjsolution}
					\cjwhy{\(\frac{x}{x - 1} + \frac{x - 5}{(x - 1)(x + 3)} = \frac{1}{x + 3}\)}{Factor by grouping.}
					\cjwhy{\(\frac{x(x + 3)}{(x - 1)(x + 3)} + \frac{x - 5}{(x - 1)(x + 3)} = \frac{x - 1}{(x - 1)(x + 3)}\)}{\(\text{LCM} = (x - 1)(x + 3)\)}
					\cjcontinue{\(\frac{x^2 + 3x}{(x - 1)(x + 3)} + \frac{x - 5}{(x - 1)(x + 3)} = \frac{x - 1}{(x - 1)(x + 3)}\)}
					\cjwhy{\(x^2 + 3x + x - 5 = x - 1\)}{Eliminate denominator. \(x \in \{-3, 1\}\) are undefined points.}
					\cjcontinue{\(x^2 + 3x + x - 5 - x + 1 = 0\)}
					\cjcontinue{\(x^2 + 3x - 4 = 0\)}
					\cjwhy{\((x + 4)(x - 1) = 0\)}{Factor by grouping.}
					\cjcontinue{\(x = -4\)}
				\end{cjsolution}
	\end{cjsection}
\end{document}
