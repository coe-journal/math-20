% SPDX-FileCopyrightText: Copyright (C) Nile Jocson <novoseiversia@gmail.com>
% SPDX-License-Identifier: MPL-2.0

\documentclass{article}

% SPDX-FileCopyrightText: Copyright (C) Nile Jocson <novoseiversia@gmail.com>
% SPDX-License-Identifier: MPL-2.0

\usepackage{amsmath}
\usepackage{amssymb}
\usepackage[a4paper, margin=1in]{geometry}
\usepackage{hyperref}
\usepackage{physics}
\usepackage{tabularx}



\renewcommand{\arraystretch}{1.75}



\newcommand{\cjboilerplate}[2]{
	\renewcommand{\thesubsubsection}{\thesubsection.\alph{subsubsection}}

	\author{Nile Jocson \textless{}novoseiversia@gmail.com\textgreater{}}
	\title{Exercise Solutions for #1\\{\large #2}}
	\date{\today}

	\maketitle{}
		\pagebreak

	\tableofcontents{}
		\pagebreak
}



\newenvironment{cjsection}[1]
{
	\section{#1}
}
{
	\pagebreak
}

\newcommand{\cjitem}[1]{\subsection{#1}}
\newcommand{\cjsubitem}[1]{\subsubsection{#1}}



\newcommand{\cjnext}[1]{\(\Rightarrow\) #1}

\newcommand{\cjwhy}[2]{\hline \cjnext{#1} & #2 \\}
\newcommand{\cjcontinue}[1]{\cjnext{#1} & \\}

\newcommand{\cjqed}{\(\blacksquare\)}

\newenvironment{cjsolution}
{
	\tabularx{\textwidth}{| >{\raggedright\arraybackslash}X  >{\raggedleft\arraybackslash}X |}
}
{
		& \cjqed{} \\
		\hline
	\endtabularx
}



\newcommand*{\cjdiv}{\divisionsymbol{}}




\begin{document}
	\cjboilerplate{Math 20}{Linear, Quadratic, and Rational Equations}

	\begin{cjsection}{}
		\cjitem{Find the solution set of the following equations.}
			\cjsubitem{\(x + 9 = 5 - 3x\)}
				\begin{cjsolution}
					\cjwhy{\(x + 3x = 5 - 9\)}{Solve for \(x\).}
					\cjcontinue{\(4x = -4\)}
					\cjcontinue{\(x = -1\)}
				\end{cjsolution}

			\cjsubitem{\(\frac{2x + 3}{4} - \frac{x - 1}{2} = -\frac{1}{3}\)}
				\begin{cjsolution}
					\cjwhy{\(\frac{3(2x + 3)}{12} - \frac{6(x - 1)}{12} = -\frac{4}{12}\)}{\(\text{LCM} = 12\)}
					\cjcontinue{\(\frac{6x + 9}{12} - \frac{6x - 6}{12} = -\frac{4}{12}\)}
					\cjcontinue{\((6x + 9) - (6x - 6) = -4\)}
					\cjcontinue{\(6x + 9 - 6x + 6 = -4\)}
					\cjcontinue{\(15 = -4\)}
					\cjcontinue{\(x \in \emptyset\)}
				\end{cjsolution}

			\cjsubitem{\(3x^2 - 2x + 1 = 0\)}
				\begin{cjsolution}
					\cjwhy{\(\frac{-(-2) \pm \sqrt{{(-2)}^2 - 4(3)(1)}}{2(3)}\)}{Use the quadratic formula.}
					\cjcontinue{\(\frac{2 \pm \sqrt{-8}}{6}\)}
					\cjcontinue{\(\frac{2 \pm \sqrt{4}\sqrt{-2}}{6}\)}
					\cjcontinue{\(\frac{2 \pm 2i\sqrt{2}}{6}\)}
					\cjcontinue{\(\frac{1 \pm i\sqrt{2}}{3}\)}
					\cjcontinue{\(\frac{1}{3} \pm \frac{\sqrt{2}}{3}i\)}
					\cjcontinue{\(x \in \{\frac{1}{3} + \frac{\sqrt{2}}{3}i, \frac{1}{3} - \frac{\sqrt{2}}{3}i\}\)}
				\end{cjsolution}

			\cjsubitem{\(4x^2 + 2x = 2\)}
				\begin{cjsolution}
					\cjwhy{\(4x^2 + 2x - 2 = 0\)}{Rewrite in standard form.}
					\cjcontinue{\(2x^2 + x - 1 = 0\)}
					\cjwhy{\(2x^2 + 2x - x - 1 = 0\)}{Factor by grouping.}
					\cjcontinue{\(2x(x + 1) - 1(x + 1) = 0\)}
					\cjcontinue{\((2x - 1)(x + 1) = 0\)}
					\cjcontinue{\(x \in \{-1, \frac{1}{2}\}\)}
				\end{cjsolution}

			\cjsubitem{\(16x^2 + 9 = 24x\)}
				\begin{cjsolution}
					\cjwhy{\(16x^2 - 24x + 9 = 0\)}{Rewrite in standard form.}
					\cjwhy{\(16x^2 - 12x - 12x + 9 = 0\)}{Factor by grouping.}
					\cjcontinue{\(4x(4x - 3) - 3(4x - 3) = 0\)}
					\cjcontinue{\({(4x - 3)}^2 = 0\)}
					\cjcontinue{\(x = \frac{3}{4}\)}
				\end{cjsolution}

			\cjsubitem{\(\frac{x}{x - 1} + \frac{x - 5}{x^2 + 2x - 3} = \frac{1}{x + 3}\)}
				\begin{cjsolution}
					\cjwhy{\(\frac{x}{x - 1} + \frac{x - 5}{(x - 1)(x + 3)} = \frac{1}{x + 3}\)}{Factor by grouping.}
					\cjwhy{\(\frac{x(x + 3)}{(x - 1)(x + 3)} + \frac{x - 5}{(x - 1)(x + 3)} = \frac{x - 1}{(x - 1)(x + 3)}\)}{\(\text{LCM} = (x - 1)(x + 3)\)}
					\cjcontinue{\(\frac{x^2 + 3x}{(x - 1)(x + 3)} + \frac{x - 5}{(x - 1)(x + 3)} = \frac{x - 1}{(x - 1)(x + 3)}\)}
					\cjwhy{\(x^2 + 3x + x - 5 = x - 1\)}{Eliminate denominator. \(x \in \{-3, 1\}\) are undefined points.}
					\cjcontinue{\(x^2 + 3x + x - 5 - x + 1 = 0\)}
					\cjcontinue{\(x^2 + 3x - 4 = 0\)}
					\cjwhy{\((x + 4)(x - 1) = 0\)}{Factor by grouping.}
					\cjcontinue{\(x = -4\)}
				\end{cjsolution}

		\cjitem{Find all real values of \(k\) such that the equation \(x^2 + kx + k = x - 2\) has exactly one solution.}
			\begin{cjsolution}
				\cjwhy{\(x^2 + kx + k - x + 2 = 0\)}{Rewrite in standard form.}
				\cjcontinue{\(x^2 + (k - 1)x + (k + 2) = 0\)}
				\cjwhy{\({(k - 1)}^2 - 4(1)(k + 2) = 0\)}{A quadratic equation has exactly one solution if the value of its discriminant is 0.}
				\cjcontinue{\(k^2 - 2k + 1 - 4k - 8 = 0\)}
				\cjcontinue{\(k^2 - 6k - 7 = 0\)}
				\cjwhy{\((k - 7)(k + 1) = 0\)}{Factor by grouping.}
				\cjcontinue{\(k \in \{-1, 7\}\)}
			\end{cjsolution}
	\end{cjsection}
\end{document}
