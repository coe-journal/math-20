% SPDX-FileCopyrightText: Copyright (C) Nile Jocson <novoseiversia@gmail.com>
% SPDX-License-Identifier: MPL-2.0

\documentclass{article}

% SPDX-FileCopyrightText: Copyright (C) Nile Jocson <novoseiversia@gmail.com>
% SPDX-License-Identifier: MPL-2.0

\usepackage{amsmath}
\usepackage{amssymb}
\usepackage[a4paper, margin=1in]{geometry}
\usepackage{hyperref}
\usepackage{physics}
\usepackage{tabularx}



\renewcommand{\arraystretch}{1.75}



\newcommand{\cjboilerplate}[2]{
	\renewcommand{\thesubsubsection}{\thesubsection.\alph{subsubsection}}

	\author{Nile Jocson \textless{}novoseiversia@gmail.com\textgreater{}}
	\title{Exercise Solutions for #1\\{\large #2}}
	\date{\today}

	\maketitle{}
		\pagebreak

	\tableofcontents{}
		\pagebreak
}



\newenvironment{cjsection}[1]
{
	\section{#1}
}
{
	\pagebreak
}

\newcommand{\cjitem}[1]{\subsection{#1}}
\newcommand{\cjsubitem}[1]{\subsubsection{#1}}



\newcommand{\cjnext}[1]{\(\Rightarrow\) #1}

\newcommand{\cjwhy}[2]{\hline \cjnext{#1} & #2 \\}
\newcommand{\cjcontinue}[1]{\cjnext{#1} & \\}

\newcommand{\cjqed}{\(\blacksquare\)}

\newenvironment{cjsolution}
{
	\tabularx{\textwidth}{| >{\raggedright\arraybackslash}X  >{\raggedleft\arraybackslash}X |}
}
{
		& \cjqed{} \\
		\hline
	\endtabularx
}



\newcommand*{\cjdiv}{\divisionsymbol{}}




\begin{document}
	\cjboilerplate{Math 20}{Conics (Hyperbola), Systems of Linear Equations}

	\begin{cjsection}{}
		\cjitem{Identify the following conic sections.}
			\cjsubitem{\(2x^2 - 3y^2 + 4x + 6y - 1 = 0\)}
				\begin{cjsolution}
					\cjwhy{\(2x^2 + 4x - 3y^2 + 6y = 1\)}{Group terms.}
					\cjcontinue{\(2(x^2 + 2x) - 3(y^2 - 2y) = 1\)}
					\cjsubwhy{\(2(x^2 + 2x + 1) - 3(y^2 - 2y) = 1 + 2(1)\)}{Complete the square.}
					\cjcontinue{\(2(x^2 + 2x + 1) - 3(y^2 - 2y) = 3\)}
					\cjcontinue{\(2{(x + 1)}^2 - 3(y^2 - 2y) = 3\)}
					\cjsubwhy{\(2{(x + 1)}^2 - 3(y^2 - 2y + 1) = 3 - 3(1)\)}{Complete the square.}
					\cjcontinue{\(2{(x + 1)}^2 - 3(y^2 - 2y + 1) = 0\)}
					\cjfastep{Not a conic.}{Cannot divide both sides.}
				\end{cjsolution}

			\cjsubitem{\(2x^2 + 3y^2 + 16x - 18y - 53 = 0\)}
				\begin{cjsolution}
					\cjwhy{\(2x^2 + 16x + 3y^2 - 18y = 53\)}{Group terms.}
					\cjcontinue{\(2(x^2 + 8x) + 3(y^2 - 6y) = 53\)}
					\cjsubwhy{\(2(x^2 + 8x + 16) + 3(y^2 - 6y) = 53 + 2(16)\)}{Complete the square.}
					\cjcontinue{\(2(x^2 + 8x + 16) + 3(y^2 - 6y) = 85\)}
					\cjcontinue{\(2{(x + 4)}^2 + 3(y^2 - 6y) = 85\)}
					\cjsubwhy{\(2{(x + 4)}^2 + 3(y^2 - 6y + 9) = 85 + 3(9)\)}{Complete the square.}
					\cjcontinue{\(2{(x + 4)}^2 + 3(y^2 - 6y + 9) = 112\)}
					\cjcontinue{\(2{(x + 4)}^2 + 3{(y - 3)}^2 = 112\)}
					\cjfa{Ellipse.}
				\end{cjsolution}

			\cjsubitem{\(9x + y^2 + 4y - 5 = 0\)}
				\begin{cjsolution}
					\cjwhy{\(y^2 + 4y = -9x + 5\)}{Group terms.}
					\cjsubwhy{\(y^2 + 4y + 4 = -9x + 9\)}{Complete the square.}
					\cjcontinue{\({(y + 2)}^2 = -9(x - 1)\)}
					\cjfa{Parabola.}
				\end{cjsolution}

			\cjsubitem{\(4x^2 - x = y^2 + 1\)}
				\begin{cjsolution}
					\cjwhy{\(4x^2 - x - y^2 = 1\)}{Group terms.}
					\cjcontinue{\(4(x^2 - \frac{1}{4}x) - y^2 = 1\)}
					\cjsubwhy{\(4(x^2 - \frac{1}{4}x + \frac{1}{64}) - y^2 = 1 + \frac{1}{16}\)}{Complete the square.}
					\cjcontinue{\(4{(x - \frac{1}{8})}^2 - y^2 = \frac{17}{16}\)}
					\cjfa{Hyperbola.}
				\end{cjsolution}

			\cjsubitem{\(7y - y^2 - x = 0\)}
				\begin{cjsolution}
					\cjwhy{\(-y^2 + 7y = x\)}{Group terms.}
					\cjcontinue{\(y^2 - 7y = -x\)}
					\cjsubwhy{\(y^2 - 7y + \frac{49}{4} = -x + \frac{49}{4}\)}{Complete the square.}
					\cjcontinue{\({(y^2 - \frac{7}{2})}^2 = -1(x - \frac{49}{4})\)}
					\cjfa{Parabola.}
				\end{cjsolution}

		\cjitem{Sketch the graph of the following hyperbolas.}
			\cjsubitem{\(x^2 - 16y^2 = 40\)}
				\begin{cjsolution}
					\cjwhy{\(\frac{x^2}{40} - \frac{16y^2}{40} = 1\)}{Rewrite in standard form.}
					\cjcontinue{\(\frac{x^2}{40} - \frac{y^2}{\frac{1}{16}(40)} = 1\)}
					\cjcontinue{\(\frac{x^2}{40} - \frac{y^2}{\frac{5}{2}} = 1\)}
					\cjcontinue{\(\frac{x^2}{\sqrt{40}^2} - \frac{y^2}{{(\sqrt{\frac{5}{2}})}^2} = 1\)}
					\cjcontinue{\(\frac{x^2}{{(\sqrt{4}\sqrt{10})}^2} - \frac{y^2}{{(\sqrt{\frac{5}{2}})}^2} = 1\)}
					\cjcontinue{\(\frac{x^2}{{(2\sqrt{10})}^2} - \frac{y^2}{{(\sqrt{\frac{5}{2}})}^2} = 1\)}
					\cjfastep{See Figure 1.}{Graph the hyperbola.}
				\end{cjsolution}
				\cjgraph{Figure 1. Graph of \(\frac{x^2}{{(2\sqrt{10})}^2} - \frac{y^2}{{(\sqrt{\frac{5}{2}})}^2} = 1\).}{16}{
					\pstGeonode[linecolor=red,PosAngle=45](0,0){C}
					\def\a{6.32455}
					\def\b{1.58113}
					\pstHyperbola(C)(\a,\b)
					\pstHyperbolaOrdNode[linecolor=blue,PointName={V_1,V_2},PosAngle={95,85}](C)(\a,\b){0}{V1}{V2}
					\pstHyperbolaFocusNode[linecolor=red,PointName={F_1,F_2},PosAngle={148,32}](C)(\a,\b){F1}{F2}
					\pstHyperbolaDirectrixLine[linecolor=red,nodesepA=-32,nodesepB=-32,PointName={D_{A_1},D_{A_2},D_{B_1},D_{B_2}},PosAngle={-30,0,-150,180}](C)(\a,\b){DA1}{DA2}{DB1}{DB2}
				}

			\cjsubitem{\(4y^2 - (x + 3)^2 = 16\)}
				\begin{cjsolution}
					\cjwhy{\(\frac{4y^2}{16} - \frac{(x + 3)^2}{16} = 1\)}{Rewrite in standard form.}
					\cjcontinue{\(\frac{y^2}{4} - \frac{(x + 3)^2}{16} = 1\)}
					\cjcontinue{\(\frac{y^2}{2^2} - \frac{(x + 3)^2}{4^2} = 1\)}
					\cjfastep{See Figure 2.}{Graph the hyperbola.}
				\end{cjsolution}
				\cjgraph{Figure 2. Graph of \(\frac{y^2}{2^2} - \frac{(x + 3)^2}{4^2} = 1\).}{16}{
					\pstGeonode[linecolor=red,PosAngle=45](-3, 0){C}
					\def\a{2}
					\def\b{4}
					\pstIHyperbola(C)(\a,\b)
					\pstIHyperbolaAbsNode[linecolor=blue,PointName={V_1,V_2},PosAngle={-90, 90}](C)(\a,\b){-3}{V1}{V2}
					\pstIHyperbolaFocusNode[linecolor=red,PointName={F_1, F_2}, PosAngle={-90, 90}](C)(\a,\b){F1}{F2}
					\pstIHyperbolaDirectrixLine[linecolor=red,nodesepA=-32,nodesepB=-32,PointName={D_{A_1},D_{A_2},D_{B_1},D_{B_2}},PosAngle={-135,-45,135,45}](C)(\a,\b){DA1}{DA2}{DB1}{DB2}
				}

			\cjsubitem{\(x^2 - y^2 + 6x - 4y = 4\)}
				\begin{cjsolution}
					\cjwhy{\(x^2 + 6x - y^2 - 4y = 4\)}{Group terms.}
					\cjsubwhy{\(x^2 + 6x + 9 - y^2 - 4y = 4 + 9\)}{Complete the square.}
					\cjcontinue{\(x^2 + 6x + 9 - y^2 - 4y = 13\)}
					\cjcontinue{\({(x + 3)}^2 - y^2 - 4y = 13\)}
					\cjcontinue{\({(x + 3)}^2 - (y^2 + 4y) = 13\)}
					\cjsubwhy{\({(x + 3)}^2 - (y^2 + 4y + 4) = 13 - 4\)}{Complete the square.}
					\cjcontinue{\({(x + 3)}^2 - (y^2 + 4y + 4) = 9\)}
					\cjcontinue{\({(x + 3)}^2 - {(y + 2)}^2 = 9\)}
					\cjsubwhy{\(\frac{{(x + 3)}^2}{9} - \frac{{(y + 2)}^2}{9} = 1\)}{Rewrite in standard form.}
					\cjcontinue{\(\frac{{(x + 3)}^2}{3^2} - \frac{{(y + 2)}^2}{3^2} = 1\)}
					\cjfastep{See Figure 3.}{Graph the hyperbola.}
				\end{cjsolution}
				\cjgraph{Figure 3. Graph of \(\frac{{(x + 3)}^2}{3^2} - \frac{{(y + 2)}^2}{3^2} = 1\).}{16}{
					\pstGeonode[linecolor=red,PosAngle=90](-3, -2){C}
					\def\a{3}
					\def\b{3}
					\pstHyperbola(C)(\a,\b)
					\pstHyperbolaOrdNode[linecolor=blue,PointName={V_1,V_2},PosAngle={180,0}](C)(\a,\b){-2}{V1}{V2}
					\pstHyperbolaFocusNode[linecolor=red,PointName={F_1,F_2},PosAngle={180,0}](C)(\a,\b){F1}{F2}
					\pstHyperbolaDirectrixLine[linecolor=red,nodesepA=-32,nodesepB=-32,PointName={D_{A_1},D_{A_2},D_{B_1},D_{B_2}},PosAngle={0,0,180,180}](C)(\a,\b){DA1}{DA2}{DB1}{DB2}
				}

		\cjitem{Find an equation of the hyperbola having \((2, 1)\) and \((-2, 1)\) as its foci and
		has a conjugate axis of length 3.}
			\begin{cjsolution}
				\cjwhy{\(h = \frac{2 + -2}{2}, k = 1\)}{This is a hyperbola with a horizontal transverse axis (from the changing x-coordinate of its foci).
				\(h\) can be found by averaging the x-coordinates of the foci, and \(k\) is simply equivalent to the y-coordinates of the foci.}
				\cjcontinue{\(h = 0, k = 1\)}
				\cjwhy{\(0 - c = 2\)}{Find \(c\) using \(F_1(h - c, k)\).}
				\cjcontinue{\(c = -2\)}
				\cjwhy{\(b = \frac{3}{2}\)}{By definition, \(b\) is half the length of the conjugate axis.}
				\cjwhy{\({(-2)}^2 = a^2 + {(\frac{3}{2})}^2\)}{Find \(a\) using \(c^2 = a^2 + b^2\).}
				\cjcontinue{\(4 = a^2 + \frac{9}{4}\)}
				\cjcontinue{\(a^2 = 4 - \frac{9}{4}\)}
				\cjcontinue{\(a^2 = \frac{16}{4} - \frac{9}{4}\)}
				\cjcontinue{\(a^2 = \frac{7}{4}\)}
				\cjcontinue{\(a = \sqrt{\frac{7}{4}}\)}
				\cjfastep{\(\frac{x^2}{{(\sqrt{\frac{7}{4}})}^2} - \frac{{(y - 1)}^2}{{(\frac{3}{2})}^2} = 1\)}{Write the hyperbola equation using \((h, k)\), \(a\), and \(b\).}
			\end{cjsolution}
	\end{cjsection}



	\begin{cjsection}{Solve the following systems of equations.}
		\cjitem{}
			\cjsystem{
				2x + 3y = 4 \\
				x - 2y = -5
			}
			\begin{cjsolution}
				\cjwhy{\(2x + 3y = 4, -2x + 4y = 10\)}{Solve for \(y\)}
				\cjcontinue{\(7y = 14\)}
				\cjcontinue{\(y = 2\)}
				\cjwhy{\(x - 2(2) = -5\)}{Solve for \(x\).}
				\cjcontinue{\(x - 4 = -5\)}
				\cjcontinue{\(x = -1\)}
				\cjfa{\(x = -1, y = 2\)}
			\end{cjsolution}

		\cjitem{}
			\cjsystem{
				-\frac{1}{x} - \frac{3}{y} = 4 \\
				\frac{2}{x} - \frac{1}{y} = 6
			}
			\begin{cjsolution}
				\cjwhy{\(-\frac{2}{x} - \frac{6}{y} = 8, \frac{2}{x} - \frac{1}{y} = 6\)}{Solve for \(y\).}
				\cjcontinue{\(-\frac{7}{y} = 14\)}
				\cjcontinue{\(14y = -7\)}
				\cjcontinue{\(y = -\frac{1}{2}\)}
				\cjwhy{\(\frac{2}{x} - \frac{1}{-\frac{1}{2}} = 6\)}{Solve for \(x\).}
				\cjcontinue{\(\frac{2}{x} + 2 = 6\)}
				\cjcontinue{\(\frac{2}{x} = 4\)}
				\cjcontinue{\(4x = 2\)}
				\cjcontinue{\(x = \frac{1}{2}\)}
				\cjfa{\(x = \frac{1}{2}, y = -\frac{1}{2}\)}
			\end{cjsolution}

		\cjitem{}
			\cjsystem{
				3x - y + 2z = 5 \\
				x + 3y - z = -5 \\
				x + y + z = 1
			}
			\begin{cjsolution}
				\cjwhy{\(3x - y + 2z = 5, -3x - 3y - 3z = -3\)}{Eliminate \(x\).}
				\cjcontinue{\(-4y - z = 2\)}
				\cjwhy{\(x + 3y - z = -5, -x - y - z = -1\)}{Eliminate \(x\).}
				\cjcontinue{\(2y - 2z = -6\)}
				\cjwhy{\(-4y - z = 2, 4y - 4z = -12\)}{Solve for \(z\).}
				\cjcontinue{\(-5z = -10\)}
				\cjcontinue{\(z = 2\)}
				\cjwhy{\(-4y - 2 = 2\)}{Solve for \(y\).}
				\cjcontinue{\(-4y = 4\)}
				\cjcontinue{\(y = -1\)}
				\cjwhy{\(x - 1 + 2 = 1\)}{Solve for \(x\).}
				\cjcontinue{\(x = 1 + 1 - 2\)}
				\cjcontinue{\(x = 0\)}
				\cjfa{\(x = 0, y = -1, z = 2\)}
			\end{cjsolution}

		\cjitem{}
			\cjsystem{
				x + 2y - z = 4 \\
				3x - y + z = 5 \\
				2x + 3y + 2z = 7 \\
			}
			\begin{cjsolution}
				\cjwhy{\(x + 2y - z = 4, 3x - y + z = 5\)}{Eliminate \(z\).}
				\cjcontinue{\(4x + y = 9\)}
				\cjwhy{\(2x + 4y - 2z = 8, 2x + 3y + 2z = 7\)}{Eliminate \(z\).}
				\cjcontinue{\(4x + 7y = 15\)}
				\cjwhy{\(-4x - y = -9, 4x + 7y = 15\)}{Solve for \(y\).}
				\cjcontinue{\(6y = 6\)}
				\cjcontinue{\(y = 1\)}
				\cjwhy{\(4x + 1 = 9\)}{Solve for \(x\).}
				\cjcontinue{\(4x = 8\)}
				\cjcontinue{\(x = 2\)}
				\cjwhy{\(2 + 2(1) - z = 4\)}{Solve for \(z\).}
				\cjcontinue{\(4 - z = 4\)}
				\cjcontinue{\(z = 0\)}
				\cjfa{\(x = 2, y = 1, z = 0\)}
			\end{cjsolution}

		\cjitem{}
			\cjsystem{
				3x + y = 9 \\
				-2x + z = -7 \\
				2y + 5z = -5 \\
			}
			\begin{cjsolution}
				\cjwhy{\(6x + 2y = 18, -6x + 3z = -21\)}{Eliminate \(x\).}
				\cjcontinue{\(2y + 3z = -3\)}
				\cjwhy{\(2y + 5z = -5, -2y - 3z = 3\)}{Solve for \(z\).}
				\cjcontinue{\(2z = -2\)}
				\cjcontinue{\(z = -1\)}
				\cjwhy{\(2y + 5(-1) = -5\)}{Solve for \(y\).}
				\cjcontinue{\(2y - 5 = -5\)}
				\cjcontinue{\(2y = 0\)}
				\cjcontinue{\(y = 0\)}
				\cjwhy{\(3x + 0 = 9\)}{Solve for \(x\).}
				\cjcontinue{\(x = 3\)}
				\cjfa{\(x = 3, y = 0, z = -1\)}
			\end{cjsolution}
	\end{cjsection}
\end{document}
