% SPDX-FileCopyrightText: Copyright (C) Nile Jocson <novoseiversia@gmail.com>
% SPDX-License-Identifier: MPL-2.0

\documentclass{article}

% SPDX-FileCopyrightText: Copyright (C) Nile Jocson <novoseiversia@gmail.com>
% SPDX-License-Identifier: MPL-2.0

\usepackage[a4paper, margin=1in]{geometry}

\usepackage{adjustbox}

\usepackage{amsmath}
\usepackage{amssymb}
\usepackage{physics}

\usepackage{tabularray}
\usepackage{tkz-tab}
\usepackage{xpatch}

\usepackage{pstricks-add}
\usepackage{pst-eucl}
\usepackage[crop=off]{auto-pst-pdf}



\renewcommand{\arraystretch}{1.75}
\renewcommand{\thesubsubsection}{\thesubsection.\alph{subsubsection}}



\DefTblrTemplate{caption}{default}{}
\DefTblrTemplate{capcont}{default}{}

\xpatchcmd{\tkzTabLine}{$0$}{$\bullet$}{}{}
\tikzset{t style/.style={style=solid}}



\newcommand*{\cjboilerplate}[2]{
	\psset{unit=5.5mm, ticks=none, xlabelsep=1pt, ylabelsep=1pt}

	\author{Nile Jocson \textless{}novoseiversia@gmail.com\textgreater{}}
	\title{Exercise Solutions for #1\\{\large #2}}
	\date{\today}

	\maketitle{}
	\null\vfill\noindent
	Copyright \copyright{} Nile Jocson \textless{}novoseiversia@gmail.com\textgreater{} \\
	Licensed under MPL-2.0. See LICENSE file.
		\pagebreak
}



\newenvironment{cjsection}[1]
{
	\section{#1}
}
{
	\pagebreak
}

\newcommand*{\cjitem}[1]{\subsection{#1}}
\newcommand*{\cjsubitem}[1]{\subsubsection{#1}}



\newcommand*{\cjsolsect}[1]{\hline --- #1: \\}
\newcommand*{\cjneeded}[2]{\hline --- Needed: \\ \(\square\) #1 \(= \mathord{?}\) & #2 \\}
\newcommand*{\cjgiven}[2]{\(\square\) #1 & #2 \\}

\newcommand*{\cjwhy}[2]{\hline \(\Rightarrow\) #1 & #2 \\}
\newcommand*{\cjsubwhy}[2]{\(\Rightarrow\) #1 & #2 \\}
\newcommand*{\cjcontinue}[1]{\(\Rightarrow\) #1 & \\}
\newcommand*{\cjfa}[1]{\hline \(\Rightarrow\) #1 & Final answer. \\}
\newcommand*{\cjfastep}[2]{\hline \(\Rightarrow\) #1 & Final answer. #2 \\}

\newcommand*{\cjsign}[1]{
	\hline & Create a table of signs. \\
	\begin{adjustbox}{width=0.49\textwidth}
		\begin{tikzpicture}
			#1
		\end{tikzpicture}
	\end{adjustbox} \\ \\
}

\newcommand*{\cjgraph}[3]{
	\begin{center}
		\begin{adjustbox}{width=\textwidth}
			\begin{pspicture*}(-#2,-#2)(#2,#2)
				\psaxes[labels=none]{<->}(0,0)(-#2,-#2)(#2,#2)
				#3
			\end{pspicture*}
		\end{adjustbox}
		#1
	\end{center}
	\pagebreak
}

\newcommand*{\cjsystem}[2]{
	\begin{equation*}
		#1
		\begin{cases}
			#2
		\end{cases}
	\end{equation*}
}

\newcommand*{\cjqed}{\(\blacksquare\)}



\NewDocumentEnvironment{cjsolution}{+b}
{
	\begin{longtblr}
	[
		expand = \cjwhy\cjsubwhy\cjcontinue\cjfa\cjfastep\cjsign\cjgiven\cjsolsect\cjneeded\cjsystem
	]
	{
		colspec = {|lX[r]|},
		width = \textwidth
	}
		#1
		& \cjqed{} \\
		\hline
	\end{longtblr}
}{}



\newcommand*{\cjdiv}{\divisionsymbol{}}
\newcommand*{\cjexp}[1]{\times 10^{#1}}
\newcommand*{\cjunit}[1]{\text{ #1}}
\newcommand*{\cjceil}[1]{\lceil#1\rceil}
\newcommand*{\cjlog}[2]{\text{log}_{#1} #2}




\begin{document}
	\cjboilerplate{Math 20}{Conics (Hyperbola), Systems of Linear Equations}

	\begin{cjsection}{}
		\cjitem{Identify the following conic sections.}
			\cjsubitem{\(2x^2 - 3y^2 + 4x + 6y - 1 = 0\)}
				\begin{cjsolution}
					\cjwhy{\(2x^2 + 4x - 3y^2 + 6y = 1\)}{Group terms.}
					\cjcontinue{\(2(x^2 + 2x) - 3(y^2 - 2y) = 1\)}
					\cjsubwhy{\(2(x^2 + 2x + 1) - 3(y^2 - 2y) = 1 + 2(1)\)}{Complete the square.}
					\cjcontinue{\(2(x^2 + 2x + 1) - 3(y^2 - 2y) = 3\)}
					\cjcontinue{\(2{(x + 1)}^2 - 3(y^2 - 2y) = 3\)}
					\cjsubwhy{\(2{(x + 1)}^2 - 3(y^2 - 2y + 1) = 3 - 3(1)\)}{Complete the square.}
					\cjcontinue{\(2{(x + 1)}^2 - 3(y^2 - 2y + 1) = 0\)}
					\cjfastep{Not a conic.}{Cannot divide both sides.}
				\end{cjsolution}

			\cjsubitem{\(2x^2 + 3y^2 + 16x - 18y - 53 = 0\)}
				\begin{cjsolution}
					\cjwhy{\(2x^2 + 16x + 3y^2 - 18y = 53\)}{Group terms.}
					\cjcontinue{\(2(x^2 + 8x) + 3(y^2 - 6y) = 53\)}
					\cjsubwhy{\(2(x^2 + 8x + 16) + 3(y^2 - 6y) = 53 + 2(16)\)}{Complete the square.}
					\cjcontinue{\(2(x^2 + 8x + 16) + 3(y^2 - 6y) = 85\)}
					\cjcontinue{\(2{(x + 4)}^2 + 3(y^2 - 6y) = 85\)}
					\cjsubwhy{\(2{(x + 4)}^2 + 3(y^2 - 6y + 9) = 85 + 3(9)\)}{Complete the square.}
					\cjcontinue{\(2{(x + 4)}^2 + 3(y^2 - 6y + 9) = 112\)}
					\cjcontinue{\(2{(x + 4)}^2 + 3{(y - 3)}^2 = 112\)}
					\cjfa{Ellipse.}
				\end{cjsolution}

			\cjsubitem{\(9x + y^2 + 4y - 5 = 0\)}
				\begin{cjsolution}
					\cjwhy{\(y^2 + 4y = -9x + 5\)}{Group terms.}
					\cjsubwhy{\(y^2 + 4y + 4 = -9x + 9\)}{Complete the square.}
					\cjcontinue{\({(y + 2)}^2 = -9(x - 1)\)}
					\cjfa{Parabola.}
				\end{cjsolution}

			\cjsubitem{\(4x^2 - x = y^2 + 1\)}
				\begin{cjsolution}
					\cjwhy{\(4x^2 - x - y^2 = 1\)}{Group terms.}
					\cjcontinue{\(4(x^2 - \frac{1}{4}x) - y^2 = 1\)}
					\cjsubwhy{\(4(x^2 - \frac{1}{4}x + \frac{1}{64}) - y^2 = 1 + \frac{1}{16}\)}{Complete the square.}
					\cjcontinue{\(4{(x - \frac{1}{8})}^2 - y^2 = \frac{17}{16}\)}
					\cjfa{Hyperbola.}
				\end{cjsolution}

			\cjsubitem{\(7y - y^2 - x = 0\)}
				\begin{cjsolution}
					\cjwhy{\(-y^2 + 7y = x\)}{Group terms.}
					\cjcontinue{\(y^2 - 7y = -x\)}
					\cjsubwhy{\(y^2 - 7y + \frac{49}{4} = -x + \frac{49}{4}\)}{Complete the square.}
					\cjcontinue{\({(y^2 - \frac{7}{2})}^2 = -1(x - \frac{49}{4})\)}
					\cjfa{Parabola.}
				\end{cjsolution}

		\cjitem{Sketch the graph of the following hyperbolas.}
			\cjsubitem{\(x^2 - 16y^2 = 40\)}
				\begin{cjsolution}
					\cjwhy{\(\frac{x^2}{40} - \frac{16y^2}{40} = 1\)}{Rewrite in standard form.}
					\cjcontinue{\(\frac{x^2}{40} - \frac{y^2}{\frac{1}{16}(40)} = 1\)}
					\cjcontinue{\(\frac{x^2}{40} - \frac{y^2}{\frac{5}{2}} = 1\)}
					\cjcontinue{\(\frac{x^2}{\sqrt{40}^2} - \frac{y^2}{\sqrt{\frac{5}{2}}^2} = 1\)}
					\cjcontinue{\(\frac{x^2}{{(\sqrt{4}\sqrt{10})}^2} - \frac{y^2}{\sqrt{\frac{5}{2}}^2} = 1\)}
					\cjcontinue{\(\frac{x^2}{{(2\sqrt{10})}^2} - \frac{y^2}{\sqrt{\frac{5}{2}}^2} = 1\)}
					\cjfastep{See Figure 1.}{Graph the hyperbola.}
				\end{cjsolution}
				\cjgraph{Figure 1. Graph of \(\frac{x^2}{{(2\sqrt{10})}^2} - \frac{y^2}{\sqrt{\frac{5}{2}}^2} = 1\).}{16}{
					\pstGeonode[linecolor=red,PosAngle=45](0,0){C}
					\def\a{6.32455}
					\def\b{1.58113}
					\pstHyperbola(C)(\a,\b)
					\pstHyperbolaOrdNode[linecolor=blue,PointName={V_1,V_2},PosAngle={95,85}](C)(\a,\b){0}{V1}{V2}
					\pstHyperbolaFocusNode[linecolor=red,PointName={F_1,F_2},PosAngle={148,32}](C)(\a,\b){F1}{F2}
					\pstHyperbolaDirectrixLine[linecolor=red,nodesepA=-32,nodesepB=-32,PointName={D_{A_1},D_{A_2},D_{B_1},D_{B_2}},PosAngle={-30,0,-150,180}](C)(\a,\b){DA1}{DA2}{DB1}{DB2}
				}

			\cjsubitem{\(4y^2 - (x + 3)^2 = 16\)}
				\begin{cjsolution}
					\cjwhy{\(\frac{4y^2}{16} - \frac{(x + 3)^2}{16} = 1\)}{Rewrite in standard form.}
					\cjcontinue{\(\frac{y^2}{4} - \frac{(x + 3)^2}{16} = 1\)}
					\cjcontinue{\(\frac{y^2}{2^2} - \frac{(x + 3)^2}{4^2} = 1\)}
					\cjfastep{See Figure 2.}{Graph the hyperbola.}
				\end{cjsolution}
				\cjgraph{Figure 2. Graph of \(\frac{y^2}{2^2} - \frac{(x + 3)^2}{4^2} = 1\).}{16}{
					\pstGeonode[linecolor=red,PosAngle=45](-3, 0){C}
					\def\a{2}
					\def\b{4}
					\pstIHyperbola(C)(\a,\b)
					\pstIHyperbolaAbsNode[linecolor=blue,PointName={V_1,V_2},PosAngle={-90, 90}](C)(\a,\b){-3}{V1}{V2}
					\pstIHyperbolaFocusNode[linecolor=red,PointName={F_1, F_2}, PosAngle={-90, 90}](C)(\a,\b){F1}{F2}
					\pstIHyperbolaDirectrixLine[linecolor=red,nodesepA=-32,nodesepB=-32,PointName={D_{A_1},D_{A_2},D_{B_1},D_{B_2}},PosAngle={-135,-45,135,45}](C)(\a,\b){DA1}{DA2}{DB1}{DB2}
				}

			\cjsubitem{\(x^2 - y^2 + 6x - 4y = 4\)}
				\begin{cjsolution}
					\cjwhy{\(x^2 + 6x - y^2 - 4y = 4\)}{Group terms.}
					\cjsubwhy{\(x^2 + 6x + 9 - y^2 - 4y = 4 + 9\)}{Complete the square.}
					\cjcontinue{\(x^2 + 6x + 9 - y^2 - 4y = 13\)}
					\cjcontinue{\({(x + 3)}^2 - y^2 - 4y = 13\)}
					\cjcontinue{\({(x + 3)}^2 - (y^2 + 4y) = 13\)}
					\cjsubwhy{\({(x + 3)}^2 - (y^2 + 4y + 4) = 13 - 4\)}{Complete the square.}
					\cjcontinue{\({(x + 3)}^2 - (y^2 + 4y + 4) = 9\)}
					\cjcontinue{\({(x + 3)}^2 - {(y + 2)}^2 = 9\)}
					\cjsubwhy{\(\frac{{(x + 3)}^2}{9} - \frac{{(y + 2)}^2}{9} = 1\)}{Rewrite in standard form.}
					\cjcontinue{\(\frac{{(x + 3)}^2}{3^2} - \frac{{(y + 2)}^2}{3^2} = 1\)}
					\cjfastep{See Figure 3.}{Graph the hyperbola.}
				\end{cjsolution}
				\cjgraph{Figure 3. Graph of \(\frac{{(x + 3)}^2}{3^2} - \frac{{(y + 2)}^2}{3^2} = 1\).}{16}{
					\pstGeonode[linecolor=red,PosAngle=90](-3, -2){C}
					\def\a{3}
					\def\b{3}
					\pstHyperbola(C)(\a,\b)
					\pstHyperbolaOrdNode[linecolor=blue,PointName={V_1,V_2},PosAngle={180,0}](C)(\a,\b){-2}{V1}{V2}
					\pstHyperbolaFocusNode[linecolor=red,PointName={F_1,F_2},PosAngle={180,0}](C)(\a,\b){F1}{F2}
					\pstHyperbolaDirectrixLine[linecolor=red,nodesepA=-32,nodesepB=-32,PointName={D_{A_1},D_{A_2},D_{B_1},D_{B_2}},PosAngle={0,0,180,180}](C)(\a,\b){DA1}{DA2}{DB1}{DB2}
				}
	\end{cjsection}
\end{document}
