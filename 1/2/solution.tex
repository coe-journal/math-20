% SPDX-FileCopyrightText: Copyright (C) Nile Jocson <novoseiversia@gmail.com>
% SPDX-License-Identifier: MPL-2.0

\documentclass{article}

% SPDX-FileCopyrightText: Copyright (C) Nile Jocson <novoseiversia@gmail.com>
% SPDX-License-Identifier: MPL-2.0

\usepackage{amsmath}
\usepackage{amssymb}
\usepackage[a4paper, margin=1in]{geometry}
\usepackage{hyperref}
\usepackage{physics}
\usepackage{tabularx}



\renewcommand{\arraystretch}{1.75}



\newcommand{\cjboilerplate}[2]{
	\renewcommand{\thesubsubsection}{\thesubsection.\alph{subsubsection}}

	\author{Nile Jocson \textless{}novoseiversia@gmail.com\textgreater{}}
	\title{Exercise Solutions for #1\\{\large #2}}
	\date{\today}

	\maketitle{}
		\pagebreak

	\tableofcontents{}
		\pagebreak
}



\newenvironment{cjsection}[1]
{
	\section{#1}
}
{
	\pagebreak
}

\newcommand{\cjitem}[1]{\subsection{#1}}
\newcommand{\cjsubitem}[1]{\subsubsection{#1}}



\newcommand{\cjnext}[1]{\(\Rightarrow\) #1}

\newcommand{\cjwhy}[2]{\hline \cjnext{#1} & #2 \\}
\newcommand{\cjcontinue}[1]{\cjnext{#1} & \\}

\newcommand{\cjqed}{\(\blacksquare\)}

\newenvironment{cjsolution}
{
	\tabularx{\textwidth}{| >{\raggedright\arraybackslash}X  >{\raggedleft\arraybackslash}X |}
}
{
		& \cjqed{} \\
		\hline
	\endtabularx
}



\newcommand*{\cjdiv}{\divisionsymbol{}}




\begin{document}
	\cjboilerplate{Math 20}{Equations in Quadratic Form and with Radicals and Absolute Values}

	\begin{cjsection}{Solve for \(x\)}
		\cjitem{\(\sqrt{2x + 3} - \sqrt{x - 2} = \sqrt{x + 1}\)}
			\begin{cjsolution}
				\cjwhy{\({(\sqrt{2x + 3} - \sqrt{x - 2})}^2 = x + 1\)}{Square both sides.}
				\cjcontinue{\(2x + 3 - 2\sqrt{2x + 3}\sqrt{x - 2} + x - 2 = x + 1\)}
				\cjcontinue{\(2x + 3 + x - 2 - x - 1 = 2\sqrt{2x + 3}\sqrt{x - 2}\)}
				\cjcontinue{\(2x = 2\sqrt{2x + 3}\sqrt{x - 2}\)}
				\cjcontinue{\(x = \sqrt{2x + 3}\sqrt{x - 2}\)}
				\cjsubwhy{\(x^2 = (2x + 3)(x - 2)\)}{Square both sides.}
				\cjcontinue{\(x^2 = 2x^2 - 4x + 3x - 6\)}
				\cjcontinue{\(x^2 = 2x^2 - x - 6\)}
				\cjcontinue{\(2x^2 - x^2 - x - 6 = 0\)}
				\cjcontinue{\(x^2 - x - 6 = 0\)}
				\cjsubwhy{\((x - 3)(x + 2) = 0\)}{Factor by grouping.}
				\cjcontinue{\(x \subseteq \{-2, 3\}\)}
				\cjwhy{\(\sqrt{2(-2) + 3} - \sqrt{-2 - 2} = \sqrt{-2 + 1}\)}{Verify \(x = -2\)}
				\cjcontinue{\(\sqrt{-4 + 3} - \sqrt{-2 - 2} = \sqrt{-2 + 1}\)}
				\cjcontinue{\(\sqrt{-1} - \sqrt{-4} = \sqrt{-1}\)}
				\cjcontinue{\(i - 2i = i\)}
				\cjcontinue{\(-i = i\)}
				\cjcontinue{\(x \neq -2\)}
				\cjwhy{\(\sqrt{2(3) + 3} - \sqrt{3 - 2} = \sqrt{3 + 1}\)}{Verify \(x = 3\)}
				\cjcontinue{\(\sqrt{6 + 3} - \sqrt{3 - 2} = \sqrt{3 + 1}\)}
				\cjcontinue{\(\sqrt{9} - \sqrt{1} = \sqrt{4}\)}
				\cjcontinue{\(3 - 1 = 2\)}
				\cjcontinue{\(2 = 2\)}
				\cjcontinue{\(x = 3\)}
			\end{cjsolution}

		\cjitem{\(1 = x + \sqrt{2x - 3}\)}
			\begin{cjsolution}
				\cjwhy{\(1 - x = \sqrt{2x - 3}\)}{Isolate the root.}
				\cjsubwhy{\({(1 - x)}^2 = 2x - 3\)}{Square both sides.}
				\cjcontinue{\(1 - 2x + x^2 = 2x - 3\)}
				\cjcontinue{\(1 - 2x + x^2 - 2x + 3 = 0\)}
				\cjcontinue{\(x^2 - 4x + 4 = 0\)}
				\cjsubwhy{\({(x - 2)}^2\)}{Factor by grouping.}
				\cjcontinue{\(x = 2\)}
				\cjwhy{\(1 = 2 + \sqrt{2(2) - 3}\)}{Verify \(x = 2\)}
				\cjcontinue{\(1 = 2 + \sqrt{4 - 3}\)}
				\cjcontinue{\(1 = 2 + \sqrt{1}\)}
				\cjcontinue{\(1 = 2 + 1\)}
				\cjcontinue{\(1 = 3\)}
				\cjcontinue{\(x \neq 2\)}
				\cjfa{\(x \in \emptyset\)}
			\end{cjsolution}

		\cjitem{\(\abs{\frac{3x - 4}{2x + 3}} = 1\)}
			\begin{cjsolution}
				\cjwhy{\(\frac{3x - 4}{2x + 3} = -1\)}{\(\abs{a} = b \Rightarrow a = \pm b\). Solve for \(a = -b\)}
				\cjcontinue{\(\frac{3x - 4}{2x + 3} = -\frac{2x + 3}{2x + 3}\)}
				\cjsubwhy{\(3x - 4 = -(2x + 3)\)}{Eliminate denominator. \(x = -\frac{3}{2}\) is an undefined point.}
				\cjcontinue{\(3x - 4 = -2x - 3\)}
				\cjcontinue{\(3x + 2x = -3 + 4\)}
				\cjcontinue{\(5x = 1\)}
				\cjcontinue{\(x = \frac{1}{5}\)}
				\cjwhy{\(\frac{3x - 4}{2x + 3} = 1\)}{\(\abs{a} = b \Rightarrow a = \pm b\). Solve for \(a = +b\)}
				\cjcontinue{\(\frac{3x - 4}{2x + 3} = \frac{2x + 3}{2x + 3}\)}
				\cjsubwhy{\(3x - 4 = 2x + 3\)}{Eliminate denominator. \(x = -\frac{3}{2}\) is an undefined point.}
				\cjcontinue{\(3x - 2x = 3 + 4\)}
				\cjcontinue{\(x = 7\)}
				\cjfa{\(x \in \{\frac{1}{5}, 7\}\)}
			\end{cjsolution}

		\cjitem{\(-7(\frac{1}{x} - 1) = 4 - 2{(\frac{1}{x} - 1)}^2\)}
			\begin{cjsolution}
				\cjwhy{\(-7t = 4 - 2t^2\)}{\(t = (\frac{1}{x} - 1)\). \(x = 0\) is an undefined point.}
				\cjcontinue{\(2t^2 - 7t - 4 = 0\)}
				\cjsubwhy{\(2t^2 - 8t + t - 4 = 0\)}{Factor by grouping.}
				\cjcontinue{\(2t(t - 4) + 1(t - 4) = 0\)}
				\cjcontinue{\((2t + 1)(t - 4) = 0\)}
				\cjcontinue{\((2t + 1)(t - 4) = 0\)}
				\cjcontinue{\(t \in \{-\frac{1}{2}, 4\}\)}
				\cjwhy{\(\frac{1}{x} - 1 = -\frac{1}{2}\)}{Solve for \(x\) using \(t = -\frac{1}{2}\).}
				\cjcontinue{\(\frac{1}{x} = -\frac{1}{2} + 1\)}
				\cjcontinue{\(\frac{1}{x} = \frac{1}{2}\)}
				\cjcontinue{\(x = 2\)}
				\cjwhy{\(\frac{1}{x} - 1 = 4\)}{Solve for \(x\) using \(t = 4\).}
				\cjcontinue{\(\frac{1}{x} = 4 + 1\)}
				\cjcontinue{\(\frac{1}{x} = 5\)}
				\cjcontinue{\(x = \frac{1}{5}\)}
				\cjfa{\(x \in \{\frac{1}{5}, 2\}\)}
			\end{cjsolution}
	\end{cjsection}
\end{document}
