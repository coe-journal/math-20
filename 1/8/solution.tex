% SPDX-FileCopyrightText: Copyright (C) Nile Jocson <novoseiversia@gmail.com>
% SPDX-License-Identifier: MPL-2.0

\documentclass{article}

% SPDX-FileCopyrightText: Copyright (C) Nile Jocson <novoseiversia@gmail.com>
% SPDX-License-Identifier: MPL-2.0

\usepackage[a4paper, margin=1in]{geometry}

\usepackage{adjustbox}

\usepackage{amsmath}
\usepackage{amssymb}
\usepackage{physics}

\usepackage{tabularray}
\usepackage{tkz-tab}
\usepackage{xpatch}

\usepackage{pstricks-add}
\usepackage{pst-eucl}
\usepackage[crop=off]{auto-pst-pdf}



\renewcommand{\arraystretch}{1.75}
\renewcommand{\thesubsubsection}{\thesubsection.\alph{subsubsection}}



\DefTblrTemplate{caption}{default}{}
\DefTblrTemplate{capcont}{default}{}

\xpatchcmd{\tkzTabLine}{$0$}{$\bullet$}{}{}
\tikzset{t style/.style={style=solid}}



\newcommand*{\cjboilerplate}[2]{
	\psset{unit=5.5mm, ticks=none, xlabelsep=1pt, ylabelsep=1pt}

	\author{Nile Jocson \textless{}novoseiversia@gmail.com\textgreater{}}
	\title{Exercise Solutions for #1\\{\large #2}}
	\date{\today}

	\maketitle{}
	\null\vfill\noindent
	Copyright \copyright{} Nile Jocson \textless{}novoseiversia@gmail.com\textgreater{} \\
	Licensed under MPL-2.0. See LICENSE file.
		\pagebreak
}



\newenvironment{cjsection}[1]
{
	\section{#1}
}
{
	\pagebreak
}

\newcommand*{\cjitem}[1]{\subsection{#1}}
\newcommand*{\cjsubitem}[1]{\subsubsection{#1}}



\newcommand*{\cjsolsect}[1]{\hline --- #1: \\}
\newcommand*{\cjneeded}[2]{\hline --- Needed: \\ \(\square\) #1 \(= \mathord{?}\) & #2 \\}
\newcommand*{\cjgiven}[2]{\(\square\) #1 & #2 \\}

\newcommand*{\cjwhy}[2]{\hline \(\Rightarrow\) #1 & #2 \\}
\newcommand*{\cjsubwhy}[2]{\(\Rightarrow\) #1 & #2 \\}
\newcommand*{\cjcontinue}[1]{\(\Rightarrow\) #1 & \\}
\newcommand*{\cjfa}[1]{\hline \(\Rightarrow\) #1 & Final answer. \\}
\newcommand*{\cjfastep}[2]{\hline \(\Rightarrow\) #1 & Final answer. #2 \\}

\newcommand*{\cjsign}[1]{
	\hline & Create a table of signs. \\
	\begin{adjustbox}{width=0.49\textwidth}
		\begin{tikzpicture}
			#1
		\end{tikzpicture}
	\end{adjustbox} \\ \\
}

\newcommand*{\cjgraph}[3]{
	\begin{center}
		\begin{adjustbox}{width=\textwidth}
			\begin{pspicture*}(-#2,-#2)(#2,#2)
				\psaxes[labels=none]{<->}(0,0)(-#2,-#2)(#2,#2)
				#3
			\end{pspicture*}
		\end{adjustbox}
		#1
	\end{center}
	\pagebreak
}

\newcommand*{\cjsystem}[2]{
	\begin{equation*}
		#1
		\begin{cases}
			#2
		\end{cases}
	\end{equation*}
}

\newcommand*{\cjqed}{\(\blacksquare\)}



\NewDocumentEnvironment{cjsolution}{+b}
{
	\begin{longtblr}
	[
		expand = \cjwhy\cjsubwhy\cjcontinue\cjfa\cjfastep\cjsign\cjgiven\cjsolsect\cjneeded\cjsystem
	]
	{
		colspec = {|lX[r]|},
		width = \textwidth
	}
		#1
		& \cjqed{} \\
		\hline
	\end{longtblr}
}{}



\newcommand*{\cjdiv}{\divisionsymbol{}}
\newcommand*{\cjexp}[1]{\times 10^{#1}}
\newcommand*{\cjunit}[1]{\text{ #1}}
\newcommand*{\cjceil}[1]{\lceil#1\rceil}
\newcommand*{\cjlog}[2]{\text{log}_{#1} #2}




\begin{document}
	\cjboilerplate{Math 20}{Linear Inequalities System, Nonlinear Systems}

	\begin{cjsection}{}
		\cjitem{Solve algebraically for the solution sets of the following systems of equations.}
			\cjsubitem{}
				\cjsystem{}{
					y = x^2 - 2x - 8 \\
					4x + 3y + 3 = 0
				}
				\begin{cjsolution}
					\cjwhy{\(3y = -4x - 3\)}{Rewrite in terms of \(y\).}
					\cjcontinue{\(y = -\frac{4}{3}x - 1\)}
					\cjwhy{\(x^2 - 2x - 8 = -\frac{4}{3}x - 1\)}{Solve for \(x\).}
					\cjcontinue{\(x^2 - 2x + \frac{4}{3}x - 8 + 1 = 0\)}
					\cjcontinue{\(x^2 - \frac{6}{3}x + \frac{4}{3}x - 7 = 0\)}
					\cjcontinue{\(x^2 - \frac{2}{3}x - 7 = 0\)}
					\cjcontinue{\(3x^2 - 2x - 21 = 0\)}
					\cjsubwhy{\(3x^2 - 9x + 7x - 21 = 0\)}{Factor by grouping.}
					\cjcontinue{\(3x(x - 3) + 7(x - 3) = 0\)}
					\cjcontinue{\((3x + 7)(x - 3) = 0\)}
					\cjcontinue{\(x \in \{-\frac{7}{3}, 3\}\)}
					\cjwhy{\(4(-\frac{7}{3}) + 3y + 3 = 0\)}{Solve for \(y, x = -\frac{7}{3}\)}
					\cjcontinue{\(-\frac{28}{3} + 3y + 3 = 0\)}
					\cjcontinue{\(-\frac{28}{3} + 3y + \frac{9}{3} = 0\)}
					\cjcontinue{\(3y - \frac{19}{3} = 0\)}
					\cjcontinue{\(3y = \frac{19}{3}\)}
					\cjcontinue{\(y = \frac{19}{9}\)}
					\cjwhy{\(4(3) + 3y + 3 = 0\)}{Solve for \(y, x = 3\)}
					\cjcontinue{\(12 + 3y + 3 = 0\)}
					\cjcontinue{\(3y + 15 = 0\)}
					\cjcontinue{\(3y = -15\)}
					\cjcontinue{\(y = -5\)}
					\cjfa{\((x, y) \in \{(-\frac{7}{3}, \frac{19}{9}), (3, -5)\}\)}
				\end{cjsolution}

			\cjsubitem{}
				\cjsystem{}{
					10x^2 - xy + 4y^2 = 28 \\
					2x^2 - 3xy - 2y^2 = 0
				}
				\begin{cjsolution}
					\cjwhy{\(2x^2 - 3yx - 2y^2 = 0\)}{Rewrite in terms of \(x\).}
					\cjsubwhy{\(\frac{3y \pm \sqrt{{(3y)}^2 - 4(2)(-2y^2)}}{4}\)}{Use the quadratic formula.}
					\cjcontinue{\(\frac{3y \pm \sqrt{9y^2 + 16y^2}}{4}\)}
					\cjcontinue{\(\frac{3y \pm \sqrt{25y^2}}{4}\)}
					\cjcontinue{\(\frac{3y \pm 5y}{4}\)}
					\cjcontinue{\(\frac{3y + 5y}{4}, \frac{3y - 5y}{4}\)}
					\cjcontinue{\(\frac{8y}{4}, \frac{-2y}{4}\)}
					\cjcontinue{\(x \in \{-\frac{1}{2}y, 2y\}\)}
					\cjwhy{\(10{(-\frac{1}{2}y)}^2 - (-\frac{1}{2}y)y + 4y^2 = 28\)}{Solve for \(y, x = -\frac{1}{2}y\).}
					\cjcontinue{\(10(\frac{1}{4}y^2) + \frac{1}{2}y^2 + 4y^2 = 28\)}
					\cjcontinue{\(\frac{5}{2}y^2 + \frac{1}{2}y^2 + 4y^2 = 28\)}
					\cjcontinue{\(\frac{5}{2}y^2 + \frac{1}{2}y^2 + \frac{8}{2}y^2 = 28\)}
					\cjcontinue{\(\frac{14}{2}y^2 = 28\)}
					\cjcontinue{\(7y^2 = 28\)}
					\cjcontinue{\(y^2 = 4\)}
					\cjcontinue{\(y = \pm 2\)}
					\cjwhy{\(x = -\frac{1}{2}(-2)\)}{Solve for \(x, y = -2\)}
					\cjcontinue{\(x = 1\)}
					\cjsubwhy{\(x = -\frac{1}{2}(2)\)}{Solve for \(x, y = 2\)}
					\cjcontinue{\(x = -1\)}
					\cjwhy{\(10{(2y)}^2 - (2y)y + 4y^2 = 28\)}{Solve for \(y, x = 2y\).}
					\cjcontinue{\(10{(4y^2)} - 2y^2 + 4y^2 = 28\)}
					\cjcontinue{\(40y^2 - 2y^2 + 4y^2 = 28\)}
					\cjcontinue{\(42y^2 = 28\)}
					\cjcontinue{\(y^2 = \frac{28}{42}\)}
					\cjcontinue{\(y^2 = \frac{2}{3}\)}
					\cjcontinue{\(y = \pm \sqrt{\frac{2}{3}}\)}
					\cjwhy{\(x = -2\sqrt{\frac{2}{3}}\)}{Solve for \(x, y = -\sqrt{\frac{2}{3}}\)}
					\cjsubwhy{\(x = 2\sqrt{\frac{2}{3}}\)}{Solve for \(x, y = \sqrt{\frac{2}{3}}\)}
					\cjfa{\((x, y) = \{(1, -2), (-1, 2), (-2\sqrt{\frac{2}{3}}, -\sqrt{\frac{2}{3}}), (2\sqrt{\frac{2}{3}}, \sqrt{\frac{2}{3}})\}\)}
				\end{cjsolution}

		\cjitem{Sketch the solution region for each of the given system of inequalities.}
			\cjsubitem{}
				\cjsystem{}{
					y \le 2x + 1 \\
					x < 5 \\
					y < x + 2
				}
				\cjgraph{Figure 1.}{16}{
					\pspolygon[fillstyle=vlines,hatchsep=8pt](-8.5,-17)(-8.5,-16)(1,3)(5,7)(5,-17)
					\pstLineCoef[nodesep=-32,PointName={A_1,A_2}]{-2,1,-1}{A1}{A2}
					\pstLineCoef[linestyle=dashed,nodesep=-32,PointName={B_1,B_2},PosAngle={-45,0}]{1,0,-5}{B1}{B2}
					\pstLineCoef[linestyle=dashed,nodesep=-32,PointName={C_1,C_2},PosAngle=135]{-1,1,-2}{C1}{C2}
				}

			\cjsubitem{}
				\cjsystem{}{
					x^2 + y^2 \le 16 \\
					y \le x^2 - 3 \\
					y < 0
				}
				\cjgraph{Figure 2.}{16}{
					\pspolygon[linestyle=none,fillstyle=vlines,hatchsep=8pt](-4,0)(-3.87298,-1)(-3.4641,-2)(-3,-2.64575)(-2,-3.4641)(-1,-3.87298)(0,-4)(1,-3.87298)(2,-3.4641)(3,-2.64575)(3.4641,-2)(3.87298,-1)(4,0)(1.73205,0)(1.3,-1.31)(1,-2)(0.5,-2.75)(0,-3)(-0.5,-2.75)(-1,-2)(-1.3,-1.31)(-1.73205,0)
					\pstGeonode[PointName={A_1,A_2},PosAngle=45](0,0){A1}(4,0){A2}
					\pstCircleOA{A1}{A2}
					\pstGeonode[PosAngle=-45](0,-3){B}
					\def\p{0.5}
					\pstParabola[plotpoints=500](B){\p}{-16}{16}
					\pstLineCoef[linestyle=dashed,linecolor=white,nodesep=-32,PointName={C_1,C_2},PosAngle={-45,45}]{0,1,0}{C1}{C2}
				}
	\end{cjsection}
\end{document}
