% SPDX-FileCopyrightText: Copyright (C) Nile Jocson <novoseiversia@gmail.com>
% SPDX-License-Identifier: MPL-2.0

\documentclass{article}

% SPDX-FileCopyrightText: Copyright (C) Nile Jocson <novoseiversia@gmail.com>
% SPDX-License-Identifier: MPL-2.0

\usepackage{amsmath}
\usepackage{amssymb}
\usepackage[a4paper, margin=1in]{geometry}
\usepackage{hyperref}
\usepackage{physics}
\usepackage{tabularx}



\renewcommand{\arraystretch}{1.75}



\newcommand{\cjboilerplate}[2]{
	\renewcommand{\thesubsubsection}{\thesubsection.\alph{subsubsection}}

	\author{Nile Jocson \textless{}novoseiversia@gmail.com\textgreater{}}
	\title{Exercise Solutions for #1\\{\large #2}}
	\date{\today}

	\maketitle{}
		\pagebreak

	\tableofcontents{}
		\pagebreak
}



\newenvironment{cjsection}[1]
{
	\section{#1}
}
{
	\pagebreak
}

\newcommand{\cjitem}[1]{\subsection{#1}}
\newcommand{\cjsubitem}[1]{\subsubsection{#1}}



\newcommand{\cjnext}[1]{\(\Rightarrow\) #1}

\newcommand{\cjwhy}[2]{\hline \cjnext{#1} & #2 \\}
\newcommand{\cjcontinue}[1]{\cjnext{#1} & \\}

\newcommand{\cjqed}{\(\blacksquare\)}

\newenvironment{cjsolution}
{
	\tabularx{\textwidth}{| >{\raggedright\arraybackslash}X  >{\raggedleft\arraybackslash}X |}
}
{
		& \cjqed{} \\
		\hline
	\endtabularx
}



\newcommand*{\cjdiv}{\divisionsymbol{}}




\begin{document}
	\cjboilerplate{Math 20}{The 2-Dimensional Coordinate System}

	\begin{cjsection}{}
		\cjitem{Given \(A(4, 2)\), \(B(6, -4)\) and \(C(2, -7)\), find the distance between C and the midpoint \(M\) of \(\overline{AB}\).}
			\begin{cjsolution}
				\cjwhy{\(M = (\frac{4 + 6}{2}, \frac{2 - 4}{2})\)}{Use the midpoint formula.}
				\cjcontinue{\(M = (\frac{10}{2}, \frac{-2}{2})\)}
				\cjcontinue{\(M = (5, -1)\)}
				\cjwhy{\(d_{CM} = \sqrt{{(5 - 2)}^2 + {(-7 + 1)}^2}\)}{Use the distance formula.}
				\cjcontinue{\(d_{CM} = \sqrt{{(3)}^2 + {(-6)}^2}\)}
				\cjcontinue{\(d_{CM} = \sqrt{9 + 36}\)}
				\cjcontinue{\(d_{CM} = \sqrt{45}\)}
				\cjcontinue{\(d_{CM} = \sqrt{9}\sqrt{5}\)}
				\cjfa{\(d_{CM} = 3\sqrt{5}\)}
			\end{cjsolution}

		\cjitem{Solve algebraically for the x- and y-intercepts of the graphs of the following equations.}
			\cjsubitem{\(y^2 = x - 2\)}
				\begin{cjsolution}
					\cjwhy{\(0 = x - 2\)}{Find the x-intercepts.}
					\cjcontinue{\(x_i = 2\)}
					\cjwhy{\(y^2 = 0 - 2\)}{Find the y-intercepts.}
					\cjcontinue{\(y^2 = -2\)}
					\cjsubwhy{\(y_i \in \emptyset\)}{No y-intercepts. The square of a real number cannot be negative.}
					\cjfa{\(x_i = 2, y_i \in \emptyset\)}
				\end{cjsolution}

			\cjsubitem{\(y = \frac{4x^2 + 9}{x^2 - 9}\)}
				\begin{cjsolution}
					\cjwhy{\(\frac{4x^2 + 9}{x^2 - 9} = 0\)}{Find the x-intercepts.}
					\cjsubwhy{\(\frac{4x^2 + 9}{(x - 3)(x + 3)} = 0\)}{Factor using difference of two squares.}
					\cjsubwhy{\(4x^2 + 9 = 0\)}{\(x \in \{-3, 3\}\) are undefined points.}
					\cjsubwhy{\(\frac{\pm \sqrt{-4(4)(9)}}{2(4)}\)}{Use the quadratic formula.}
					\cjcontinue{\(\frac{\pm \sqrt{-144}}{8}\)}
					\cjsubwhy{\(x_i \in \emptyset\)}{No x-intercepts. The square root of a negative number is imaginary.}
					\cjsubwhy{\(y = \frac{4(0)^2 + 9}{0^2 - 9}\)}{Find the y-intercepts.}
					\cjcontinue{\(y = \frac{9}{-9}\)}
					\cjcontinue{\(y_i = -1\)}
					\cjfa{\(x_i \in \emptyset, y_i = -1\)}
				\end{cjsolution}

			\cjsubitem{\(y = \abs{x - 2} - 2\)}
				\begin{cjsolution}
					\cjwhy{\(\abs{x - 2} - 2 = 0\)}{Find the x-intercepts.}
					\cjcontinue{\(\abs{x - 2} = 2\)}
					\cjsubwhy{\(x - 2 = 2\)}{\(\abs{a} = b \Rightarrow a = \pm b\). Solve for \(a = b\).}
					\cjcontinue{\(x_i = 4\)}
					\cjsubwhy{\(x - 2 = -2\)}{\(\abs{a} = b \Rightarrow a = \pm b\). Solve for \(a = -b\).}
					\cjcontinue{\(x_i = 0\)}
					\cjwhy{\(y = \abs{0 - 2} - 2\)}{Find the y-intercepts.}
					\cjcontinue{\(y = \abs{-2} - 2\)}
					\cjcontinue{\(y = 2 - 2\)}
					\cjcontinue{\(y_i = 0\)}
					\cjfa{\(x_i \in \{0, 4\}, y_i = 0\)}
				\end{cjsolution}

			\cjsubitem{\(y = x^2 + 1\)}
				\begin{cjsolution}
					\cjwhy{\(x^2 + 1 = 0\)}{Find the x-intercepts.}
					\cjsubwhy{\(\frac{\pm \sqrt{-(1)(1)}}{2(1)}\)}{Use the quadratic equation.}
					\cjcontinue{\(\frac{\pm \sqrt{-1}}{2}\)}
					\cjsubwhy{\(x_i \in \emptyset\)}{No x-intercepts. The square root of a negative number is imaginary.}
					\cjwhy{\(y = 0^2 + 1\)}{Find the y-intercepts.}
					\cjcontinue{\(y_i = 1\)}
					\cjfa{\(x_i \in \emptyset, y_i = 1\)}
				\end{cjsolution}

			\cjsubitem{\(x^2 + y^2 = 25\)}
				\begin{cjsolution}
					\cjwhy{\(x^2 + 0^2 = 25\)}{Find the x-intercepts.}
					\cjcontinue{\(x^2 = 25\)}
					\cjcontinue{\(x = \pm 5\)}
					\cjcontinue{\(x_i \in \{-5, 5\}\)}
					\cjwhy{\(0^2 + y^2 = 25\)}{Find the y-intercepts.}
					\cjcontinue{\(y^2 = 25\)}
					\cjcontinue{\(y = \pm 5\)}
					\cjcontinue{\(y_i \in \{-5, 5\}\)}
					\cjfa{\(x_i \in \{-5, 5\}, y_i \in \{-5, 5\}\)}
				\end{cjsolution}

			\cjsubitem{\(x = \frac{y^2}{y^4 - 4}\)}
				\begin{cjsolution}
					\cjwhy{\(x = \frac{0^2}{0^2 - 4}\)}{Find the x-intercepts.}
					\cjcontinue{\(x_i = 0\)}
					\cjwhy{\(\frac{y^2}{y^4 - 4} = 0\)}{Find the y-intercepts.}
					\cjcontinue{\(\frac{y^2}{(y^2 - 2)(y^2 + 2)} = 0\)}
					\cjsubwhy{\(y^2 = 0\)}{\(y \in \{-\sqrt{2}, \sqrt{2}\}\) are undefined points.}
					\cjsubwhy{\(y_i = 0\)}{\(0^2 = 0\)}
					\cjfa{\(x_i = 0, y_i = 0\)}
				\end{cjsolution}

			\cjsubitem{\(xy = 1\)}
				\begin{cjsolution}
					\cjwhy{\(x(0) = 1\)}{Find the x-intercepts.}
					\cjcontinue{\(0 = 1\)}
					\cjcontinue{\(x_i \in \emptyset\)}
					\cjwhy{\(0y = 1\)}{Find the y-intercepts.}
					\cjcontinue{\(0 = 1\)}
					\cjcontinue{\(y_i \in \emptyset\)}
					\cjfa{\(x_i \in \emptyset, y_i \in \emptyset\)}
				\end{cjsolution}

		\cjitem{Determine the value(s) of \(k\) for which the graph of \(y = 2x^2 + kx + 8\) does
		not intersect the x-axis. How about if the graph is to intersect the x-axis at exactly one point?}
			\begin{cjsolution}
				\cjwhy{\(2x^2 + kx + 8 = 0\)}{Set \(y = 0\), and rewrite in standard form.}
				\cjwhy{\(k^2 - 4(2)(8) < 0\)}{If the discriminant is negative, the quadratic equation has no real solutions, and as such, no x-intercepts.}
				\cjcontinue{\(k^2 - 64 < 0\)}
				\cjsubwhy{\((k - 8)(k + 8) < 0\)}{Factor using difference of two squares.}
				\cjsign{
					\tkzTabInit[lgt=3, espcl=2, deltacl=0]
						{/.8, \(k - 8\) /.8, \(k + 8\) /.8, \((k - 8)(k + 8)\) /.8}
						{,\(-4\),\(4\),}
					\tkzTabLine{,-,t,-,t,+,}
					\tkzTabLine{,-,t,+,t,+,}
					\tkzTabLine{,+,t,-,t,+,}
				}
				\cjcontinue{\(k \in (-8, 8)\)}
				\cjwhy{\(k^2 - 4(2)(8) = 0\)}{If the disctiminant is zero, the quadratic equation has exactly one real solution, and as such, exactly one x-intercept.}
				\cjcontinue{\(k^2 - 16 = 0\)}
				\cjsubwhy{\((k - 8)(k + 8) = 0\)}{Factor using difference of two squares.}
				\cjcontinue{\(k \in \{-8, 8\}\)}
				\cjfastep{\(k \in (-8, 8)\)}{If the graph should have no x-intercept.}
				\cjsubwhy{\(k \in \{-8, 8\}\)}{If the graph should have exactly one x-intercept.}
			\end{cjsolution}
	\end{cjsection}
\end{document}
