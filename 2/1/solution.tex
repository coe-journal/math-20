% SPDX-FileCopyrightText: Copyright (C) Nile Jocson <novoseiversia@gmail.com>
% SPDX-License-Identifier: MPL-2.0

\documentclass{article}

% SPDX-FileCopyrightText: Copyright (C) Nile Jocson <novoseiversia@gmail.com>
% SPDX-License-Identifier: MPL-2.0

\usepackage[a4paper, margin=1in]{geometry}

\usepackage{adjustbox}

\usepackage{amsmath}
\usepackage{amssymb}
\usepackage{physics}

\usepackage{tabularray}
\usepackage{tkz-tab}
\usepackage{xpatch}

\usepackage{pstricks-add}
\usepackage{pst-eucl}
\usepackage[crop=off]{auto-pst-pdf}



\renewcommand{\arraystretch}{1.75}
\renewcommand{\thesubsubsection}{\thesubsection.\alph{subsubsection}}



\DefTblrTemplate{caption}{default}{}
\DefTblrTemplate{capcont}{default}{}

\xpatchcmd{\tkzTabLine}{$0$}{$\bullet$}{}{}
\tikzset{t style/.style={style=solid}}



\newcommand*{\cjboilerplate}[2]{
	\psset{unit=5.5mm, ticks=none, xlabelsep=1pt, ylabelsep=1pt}

	\author{Nile Jocson \textless{}novoseiversia@gmail.com\textgreater{}}
	\title{Exercise Solutions for #1\\{\large #2}}
	\date{\today}

	\maketitle{}
	\null\vfill\noindent
	Copyright \copyright{} Nile Jocson \textless{}novoseiversia@gmail.com\textgreater{} \\
	Licensed under MPL-2.0. See LICENSE file.
		\pagebreak
}



\newenvironment{cjsection}[1]
{
	\section{#1}
}
{
	\pagebreak
}

\newcommand*{\cjitem}[1]{\subsection{#1}}
\newcommand*{\cjsubitem}[1]{\subsubsection{#1}}



\newcommand*{\cjsolsect}[1]{\hline --- #1: \\}
\newcommand*{\cjneeded}[2]{\hline --- Needed: \\ \(\square\) #1 \(= \mathord{?}\) & #2 \\}
\newcommand*{\cjgiven}[2]{\(\square\) #1 & #2 \\}

\newcommand*{\cjwhy}[2]{\hline \(\Rightarrow\) #1 & #2 \\}
\newcommand*{\cjsubwhy}[2]{\(\Rightarrow\) #1 & #2 \\}
\newcommand*{\cjcontinue}[1]{\(\Rightarrow\) #1 & \\}
\newcommand*{\cjfa}[1]{\hline \(\Rightarrow\) #1 & Final answer. \\}
\newcommand*{\cjfastep}[2]{\hline \(\Rightarrow\) #1 & Final answer. #2 \\}

\newcommand*{\cjsign}[1]{
	\hline & Create a table of signs. \\
	\begin{adjustbox}{width=0.49\textwidth}
		\begin{tikzpicture}
			#1
		\end{tikzpicture}
	\end{adjustbox} \\ \\
}

\newcommand*{\cjgraph}[3]{
	\begin{center}
		\begin{adjustbox}{width=\textwidth}
			\begin{pspicture*}(-#2,-#2)(#2,#2)
				\psaxes[labels=none]{<->}(0,0)(-#2,-#2)(#2,#2)
				#3
			\end{pspicture*}
		\end{adjustbox}
		#1
	\end{center}
	\pagebreak
}

\newcommand*{\cjsystem}[2]{
	\begin{equation*}
		#1
		\begin{cases}
			#2
		\end{cases}
	\end{equation*}
}

\newcommand*{\cjqed}{\(\blacksquare\)}



\NewDocumentEnvironment{cjsolution}{+b}
{
	\begin{longtblr}
	[
		expand = \cjwhy\cjsubwhy\cjcontinue\cjfa\cjfastep\cjsign\cjgiven\cjsolsect\cjneeded\cjsystem
	]
	{
		colspec = {|lX[r]|},
		width = \textwidth
	}
		#1
		& \cjqed{} \\
		\hline
	\end{longtblr}
}{}



\newcommand*{\cjdiv}{\divisionsymbol{}}
\newcommand*{\cjexp}[1]{\times 10^{#1}}
\newcommand*{\cjunit}[1]{\text{ #1}}
\newcommand*{\cjceil}[1]{\lceil#1\rceil}
\newcommand*{\cjlog}[2]{\text{log}_{#1} #2}




\begin{document}
	\cjboilerplate{Math 20}{Functions, Graphs, Symmetry}

	\begin{cjsection}{}
		\cjitem{Transform each of the following real-valued functions to a functional notation \(f(x)\);
		find its domain and range; and compute \(f(1)\).}
			\cjsubitem{\(f\) assigns the number \(10\) to any one-digit integer.}
				\begin{cjsolution}
					\cjfa{\(f(x) = 10\)}
					\cjsubwhy{\(\text{dom}(f) = [-9, 9]\)}{The domain is the set of one digit integers.}
					\cjsubwhy{\(\text{ran}(f) = 10\)}{The function can only equate to \(10\).}
					\cjcontinue{\(f(1) = 10\)}
				\end{cjsolution}

			\cjsubitem{\(f\) maps a given real number to the real number that is \(4\) more than its square root.}
				\begin{cjsolution}
					\cjfastep{\(f(x) = \sqrt{x} + 4\)}{Assuming that only the principal root is needed.}
					\cjsubwhy{\(\text{dom}(f) = [0, +\infty)\)}{The square root of a negative number cannot be real.}
					\cjcontinue{\(\text{ran}(f) = [4, +\infty]\)}
					\cjcontinue{\(f(1) = 5\)}
				\end{cjsolution}

			\cjsubitem{\(\{(x, y) \mid y = \frac{1}{6 - x}\}\)}
				\begin{cjsolution}
					\cjfa{\(f(x) = \frac{1}{6 - x}\)}
					\cjsubwhy{\(\text{dom}(f) = \mathbb{R} \setminus \{6\}\)}{\(x = 6\) is an undefined point.}
					\cjcontinue{\(\text{ran}(f) = \mathbb{R}\)}
					\cjcontinue{\(f(1) = \frac{1}{5}\)}
				\end{cjsolution}

			\cjsubitem{\(\{(x, y) \mid y = \sqrt{x^2 - 5x + 6}\}\)}
				\begin{cjsolution}
					\cjwhy{\((x - 2)(x - 3) \ge 0\)}{Find the domain by solving \(x^2 - 5x + 6 \ge 0\); factor by grouping.}
					\cjsign{
						\tkzTabInit[lgt=3, espcl=2, deltacl=0]
							{/.8, \(x - 2\) /.8, \(x - 3\) /.8, \((x - 2)(x - 3)\) /.8}
							{,\(2\),\(3\),}
						\tkzTabLine{,-,t,+,t,+,}
						\tkzTabLine{,-,t,-,t,+,}
						\tkzTabLine{,+,t,-,t,+,}
					}
					\cjcontinue{\(x \in (-\infty, 2] \cup [3, +\infty)\)}
					\cjfa{\(f(x) = \sqrt{x^2 - 5x + 6}\)}
					\cjcontinue{\(\text{dom}(f) = (-\infty, 2] \cup [3, +\infty)\)}
					\cjcontinue{\(\text{ran}(f) = [0, +\infty)\)}
					\cjcontinue{\(f(1) = \sqrt{2}\)}
				\end{cjsolution}

			\cjsubitem{\(\{(x, y) \mid y = \frac{1}{\sqrt[3]{x^2 - 1}}\}\)}
				\begin{cjsolution}
					\cjwhy{\((x - 1)(x + 1) = 0\)}{Find undefined points by solving \(x^2 - 1 = 0\); factor using difference of two squares.}
					\cjcontinue{\(x \in \{-1, 1\}\)}
					\cjwhy{\(x = \frac{1}{\sqrt[3]{y^2 - 1}}\)}{Find the inverse function.}
					\cjcontinue{\(\frac{1}{x} = \sqrt[3]{y^2 - 1}\)}
					\cjcontinue{\(\frac{1}{x^3} = y^2 - 1\)}
					\cjcontinue{\(y^2 = \frac{1}{x^3} + 1\)}
					\cjcontinue{\(y^2 = \frac{1}{x^3} + \frac{x^3}{x^3}\)}
					\cjcontinue{\(y^2 = \frac{1 + x^3}{x^3}\)}
					\cjcontinue{\(f'(x) = \pm \sqrt{\frac{1 + x^3}{x^3}}\)}
					\cjsubwhy{\(\sqrt{\frac{1 + x^3}{x^3}} > 0\)}{Solve for the domain. Note that \(x = 0\) is an undefined point.}
					\cjsign{
						\tkzTabInit[lgt=2, espcl=2, deltacl=0]
							{/.8, \(1 + x^3\) /.8, \(x^3\) /.8, \(\sqrt{\frac{1 + x^3}{x^3}}\) /.8}
							{,\(-1\),\(0\),}
						\tkzTabLine{,-,t,+,t,+,}
						\tkzTabLine{,-,t,-,t,+,}
						\tkzTabLine{,+,t,-,t,+,}
					}
					\cjcontinue{\(\text{dom}(f') = (-\infty, -1] \cup (0, +\infty)\)}
					\cjfa{\(f(x) = \frac{1}{\sqrt[3]{x^2 - 1}}\)}
					\cjcontinue{\(\text{dom}(f) = \mathbb{R} \setminus \{-1, 1\}\)}
					\cjcontinue{\(\text{ran}(f) = (-\infty, -1] \cup (0, +\infty)\)}
					\cjcontinue{\(f(1)\) is undefined.}
				\end{cjsolution}

		\cjitem{Given \(f(x) = \frac{4 - x^2}{x - 2}\)}
			\cjsubitem{Find \(\text{dom}(f)\) and solve for its zeroes.}
				\begin{cjsolution}
					\cjwhy{\(\frac{(2 - x)(2 + x)}{x - 2} = 0\)}{Factor using difference of two squares.}
					\cjcontinue{\(\frac{-(x - 2)(2 + x)}{x - 2} = 0\)}
					\cjsubwhy{\(-(2 + x) = 0\)}{\(x = 2\) is an undefined point.}
					\cjcontinue{\(-2 - x = 0\)}
					\cjcontinue{\(x = -2\)}
					\cjfa{\(\text{dom}(f) = \mathbb{R} \setminus \{2\}\)}
					\cjcontinue{\(x = -2\)}
				\end{cjsolution}

			\cjsubitem{Is the function odd or even? Justify your answer algebraically.}
				\begin{cjsolution}
					\cjwhy{\(f(x) = -2 - x, f(-x) = -2 + x\)}{Check if the function is even. Note the simplified function, \(f(x) = -2 - x\).}
					\cjcontinue{\(f(-x) \neq f(x)\)}
					\cjcontinue{Not even.}
					\cjwhy{\(-f(x) = -(-2 - x)\)}{Find \(-f(x)\).}
					\cjcontinue{\(-f(x) = 2 + x\)}
					\cjcontinue{\(f(-x) \neq -f(x)\)}
					\cjcontinue{Not odd.}
					\cjfa{Neither even nor odd.}
				\end{cjsolution}

			\cjsubitem{Determine its x- and y- intercepts.}
				\begin{cjsolution}
					\cjwhy{\(x_i = -2\)}{The zeroes of a function are its x-intercepts.}
					\cjwhy{\(y = -2 - 0\)}{Find the y-intercepts. Use the simplified function.}
					\cjcontinue{\(y_i = -2\)}
					\cjfa{\(x_i = -2, y_i = -2\)}
				\end{cjsolution}
	\end{cjsection}
\end{document}
