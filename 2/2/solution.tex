% SPDX-FileCopyrightText: Copyright (C) Nile Jocson <novoseiversia@gmail.com>
% SPDX-License-Identifier: MPL-2.0

\documentclass{article}

% SPDX-FileCopyrightText: Copyright (C) Nile Jocson <novoseiversia@gmail.com>
% SPDX-License-Identifier: MPL-2.0

\usepackage[a4paper, margin=1in]{geometry}

\usepackage{adjustbox}

\usepackage{amsmath}
\usepackage{amssymb}
\usepackage{physics}

\usepackage{tabularray}
\usepackage{tkz-tab}
\usepackage{xpatch}

\usepackage{pstricks-add}
\usepackage{pst-eucl}
\usepackage[crop=off]{auto-pst-pdf}



\renewcommand{\arraystretch}{1.75}
\renewcommand{\thesubsubsection}{\thesubsection.\alph{subsubsection}}



\DefTblrTemplate{caption}{default}{}
\DefTblrTemplate{capcont}{default}{}

\xpatchcmd{\tkzTabLine}{$0$}{$\bullet$}{}{}
\tikzset{t style/.style={style=solid}}



\newcommand*{\cjboilerplate}[2]{
	\psset{unit=5.5mm, ticks=none, xlabelsep=1pt, ylabelsep=1pt}

	\author{Nile Jocson \textless{}novoseiversia@gmail.com\textgreater{}}
	\title{Exercise Solutions for #1\\{\large #2}}
	\date{\today}

	\maketitle{}
	\null\vfill\noindent
	Copyright \copyright{} Nile Jocson \textless{}novoseiversia@gmail.com\textgreater{} \\
	Licensed under MPL-2.0. See LICENSE file.
		\pagebreak
}



\newenvironment{cjsection}[1]
{
	\section{#1}
}
{
	\pagebreak
}

\newcommand*{\cjitem}[1]{\subsection{#1}}
\newcommand*{\cjsubitem}[1]{\subsubsection{#1}}



\newcommand*{\cjsolsect}[1]{\hline --- #1: \\}
\newcommand*{\cjneeded}[2]{\hline --- Needed: \\ \(\square\) #1 \(= \mathord{?}\) & #2 \\}
\newcommand*{\cjgiven}[2]{\(\square\) #1 & #2 \\}

\newcommand*{\cjwhy}[2]{\hline \(\Rightarrow\) #1 & #2 \\}
\newcommand*{\cjsubwhy}[2]{\(\Rightarrow\) #1 & #2 \\}
\newcommand*{\cjcontinue}[1]{\(\Rightarrow\) #1 & \\}
\newcommand*{\cjfa}[1]{\hline \(\Rightarrow\) #1 & Final answer. \\}
\newcommand*{\cjfastep}[2]{\hline \(\Rightarrow\) #1 & Final answer. #2 \\}

\newcommand*{\cjsign}[1]{
	\hline & Create a table of signs. \\
	\begin{adjustbox}{width=0.49\textwidth}
		\begin{tikzpicture}
			#1
		\end{tikzpicture}
	\end{adjustbox} \\ \\
}

\newcommand*{\cjgraph}[3]{
	\begin{center}
		\begin{adjustbox}{width=\textwidth}
			\begin{pspicture*}(-#2,-#2)(#2,#2)
				\psaxes[labels=none]{<->}(0,0)(-#2,-#2)(#2,#2)
				#3
			\end{pspicture*}
		\end{adjustbox}
		#1
	\end{center}
	\pagebreak
}

\newcommand*{\cjsystem}[2]{
	\begin{equation*}
		#1
		\begin{cases}
			#2
		\end{cases}
	\end{equation*}
}

\newcommand*{\cjqed}{\(\blacksquare\)}



\NewDocumentEnvironment{cjsolution}{+b}
{
	\begin{longtblr}
	[
		expand = \cjwhy\cjsubwhy\cjcontinue\cjfa\cjfastep\cjsign\cjgiven\cjsolsect\cjneeded\cjsystem
	]
	{
		colspec = {|lX[r]|},
		width = \textwidth
	}
		#1
		& \cjqed{} \\
		\hline
	\end{longtblr}
}{}



\newcommand*{\cjdiv}{\divisionsymbol{}}
\newcommand*{\cjexp}[1]{\times 10^{#1}}
\newcommand*{\cjunit}[1]{\text{ #1}}
\newcommand*{\cjceil}[1]{\lceil#1\rceil}
\newcommand*{\cjlog}[2]{\text{log}_{#1} #2}




\begin{document}
	\cjboilerplate{Math 20}{Some Types of Functions, Operations}

	\begin{cjsection}{}
		\cjitem{}
			Given
			\cjsystem{g(x) = }{
				x + 4,   & \text{if } x < -2 \\
				\abs{x}, & \text{if } -1 < x < 1 \\
				2,       & \text{if } x > 3
			}
			Sketch its graph and label its x- and y- intercepts.
			\begin{cjsolution}
				\cjwhy{\(0 = x + 4\)}{Find the x-intercepts of \(y = x + 4\)}
				\cjcontinue{\(x_i = -4\)}
				\cjsubwhy{\(0 = \abs{x}\)}{Find the x-intercepts of \(y = \abs{x}\)}
				\cjcontinue{\(x_i = 0\)}
				\cjwhy{\(y = \abs{0}\)}{Find the y-intercepts of \(y = \abs{x}\)}
				\cjcontinue{\(y_i = 0\)}
				\cjfastep{See Figure 1.}{Graph the system.}
			\end{cjsolution}
			\cjgraph{Figure 1.}{16}{
				\pstGeonode[PointSymbol=o,PointName=none](-2, 2){A1}
				\pstGeonode[PointSymbol=*,PointName=x_i,PosAngle=-45](-4, 0){A2}
				\pstLineAB[nodesepA=0.1,nodesepB=-32]{A1}{A2}
				\pstGeonode[PointSymbol=o,PointName=none](-1, 1){B1}
				\pstGeonode[PointSymbol=*,PointName={x_i,y_i},PosAngle={-135, -45}](0, 0){B2}(0, 0){B3}
				\pstGeonode[PointSymbol=o,PointName=none](1, 1){B4}
				\pstLineAB[nodesepA=0.1]{B1}{B2}
				\pstLineAB[nodesepA=0.1]{B4}{B3}
				\pstGeonode[PointSymbol=o,PointName=none](3, 2){C1}
				\pstGeonode[PointSymbol=none,PointName=none](4, 2){C2}
				\pstLineAB[nodesepA=0.1,nodesepB=-32]{C1}{C2}
			}

		\cjitem{}
			Let \(f\), \(g\), and \(h\) be the functions defined from the expressions below.
			\[f(x) = \frac{x + 2}{x - 3}\]
			\[g(x) = \frac{x + 3}{x + 2}\]
			\[h(x) = \sqrt{3x - 1}\]
			Determine \((f - g)(x)\), \((\frac{f}{g})(x)\), \((fg)(x)\), \((f \circ g)(x)\),
			\((h \circ f)(x)\), \((g \circ h)(x)\) and obtain their respective domains.

			\begin{cjsolution}
				\cjwhy{\((f - g)(x) = \frac{x + 2}{x - 3} - \frac{x + 3}{x + 2}\)}{Find \((f - g)(x)\).}
				\cjcontinue{\((f - g)(x) = \frac{(x + 2)(x + 2)}{(x - 3)(x + 2)} - \frac{(x - 3)(x + 3)}{(x - 3)(x + 2)}\)}
				\cjcontinue{\((f - g)(x) = \frac{(x + 2)(x + 2) - (x - 3)(x + 3)}{(x - 3)(x + 2)}\)}
				\cjcontinue{\((f - g)(x) = \frac{x^2 + 4x + 4 - (x^2 - 9)}{(x - 3)(x + 2)}\)}
				\cjcontinue{\((f - g)(x) = \frac{x^2 + 4x + 4 - x^2 + 9}{(x - 3)(x + 2)}\)}
				\cjcontinue{\((f - g)(x) = \frac{4x + 13}{(x - 3)(x + 2)}\)}
				\cjcontinue{\(\text{dom}(f - g) = \mathbb{R} \setminus \{-2, 3\}\)}
				\cjwhy{\((\frac{f}{g})(x) = \frac{\frac{x + 2}{x - 3}}{\frac{x + 3}{x + 2}}\)}{Find \((\frac{f}{g})(x)\).}
				\cjcontinue{\((\frac{f}{g})(x) = \frac{x + 2}{x - 3} \cdot \frac{x + 2}{x + 3}\)}
				\cjcontinue{\((\frac{f}{g})(x) = \frac{{(x + 2)}^2}{(x - 3)(x + 3)}\)}
				\cjcontinue{\(\text{dom}(\frac{f}{g}) = \mathbb{R} \setminus \{-3, -2, 3\}\)}
				\cjwhy{\((fg)(x) = \frac{x + 2}{x - 3} \cdot \frac{x + 3}{x + 2}\)}{Find \((fg)(x)\).}
				\cjcontinue{\((fg)(x) = \frac{(x + 2)(x + 3)}{(x - 3)(x + 2)}\)}
				\cjcontinue{\((fg)(x) = \frac{x + 3}{x - 3}\)}
				\cjcontinue{\(\text{dom}(fg) = \mathbb{R} \setminus \{-2, 3\}\)}
				\cjwhy{\((f \circ g)(x) = \frac{\frac{x + 3}{x + 2} + 2}{\frac{x + 3}{x + 2} - 3}\)}{Find \((f \circ g)(x)\).}
				\cjcontinue{\((f \circ g)(x) = \frac{\frac{x + 3}{x + 2} + \frac{2x + 4}{x + 2}}{\frac{x + 3}{x + 2} - \frac{3x + 6}{x + 2}}\)}
				\cjcontinue{\((f \circ g)(x) = \frac{\frac{x + 3 + (2x + 4)}{x + 2}}{\frac{x + 3 - (3x + 6)}{x + 2}}\)}
				\cjcontinue{\((f \circ g)(x) = \frac{\frac{x + 3 + 2x + 4}{x + 2}}{\frac{x + 3 - 3x - 6}{x + 2}}\)}
				\cjcontinue{\((f \circ g)(x) = \frac{\frac{3x + 7}{x + 2}}{\frac{-2x - 3}{x + 2}}\)}
				\cjcontinue{\((f \circ g)(x) = \frac{3x + 7}{x + 2} \cdot \frac{x + 2}{-2x - 3}\)}
				\cjcontinue{\((f \circ g)(x) = \frac{3x + 7}{-2x - 3}\)}
				\cjcontinue{\(\text{dom}(f \circ g) = \mathbb{R} \setminus \{-2, -\frac{3}{2}\}\)}
				\cjwhy{\((h \circ f)(x) = \sqrt{3(\frac{x + 2}{x - 3}) - 1}\)}{Find \((h \circ f)(x)\).}
				\cjcontinue{\((h \circ f)(x) = \sqrt{\frac{3x + 6}{x - 3} - 1}\)}
				\cjcontinue{\((h \circ f)(x) = \sqrt{\frac{3x + 6}{x - 3} - \frac{x - 3}{x - 3}}\)}
				\cjcontinue{\((h \circ f)(x) = \sqrt{\frac{3x + 6 - (x - 3)}{x - 3}}\)}
				\cjcontinue{\((h \circ f)(x) = \sqrt{\frac{3x + 6 - x + 3}{x - 3}}\)}
				\cjcontinue{\((h \circ f)(x) = \sqrt{\frac{2x + 9}{x - 3}}\)}
				\cjsubwhy{\(\frac{2x + 9}{x - 3} \ge 0\)}{Solve for the domain. Note that \(x = 3\) is undefined.}
				\cjsign{
					\tkzTabInit[lgt=2, espcl=2, deltacl=0]
						{/.8, \(2x + 9\) /.8, \(x - 3\) /.8, \(\frac{2x + 9}{x - 3}\) /.8}
						{,\(-\frac{9}{2}\),\(3\),}
					\tkzTabLine{,-,t,+,t,+,}
					\tkzTabLine{,-,t,-,t,+,}
					\tkzTabLine{,+,t,-,t,+,}
				}
				\cjcontinue{\(\text{dom}(h \circ f) = (-\infty, -\frac{9}{2}] \cup (3, +\infty)\)}
				\cjwhy{\((g \circ h)(x) = \frac{\sqrt{3x - 1} + 3}{\sqrt{3x - 1} + 2}\)}{Find \((g \circ h)(x)\).}
				\cjsubwhy{\((g \circ h)(x) = \frac{\sqrt{3x - 1} + 3}{\sqrt{3x - 1} + 2} \cdot \frac{\sqrt{3x - 1} - 2}{\sqrt{3x - 1} - 2}\)}{Rationalize.}
				\cjcontinue{\((g \circ h)(x) = \frac{(\sqrt{3x - 1} + 3)(\sqrt{3x - 1} - 2)}{3x - 1 - 4}\)}
				\cjcontinue{\((g \circ h)(x) = \frac{(\sqrt{3x - 1} + 3)(\sqrt{3x - 1} - 2)}{3x - 5}\)}
				\cjsubwhy{\(3x - 1 \ge 0\)}{Solve for the domain. Note that \(x = \frac{5}{3}\) is undefined.}
				\cjcontinue{\(3x \ge 1\)}
				\cjcontinue{\(x \ge \frac{1}{3}\)}
				\cjcontinue{\(\text{dom}(g \circ h) = [\frac{1}{3}, +\infty) \setminus \{\frac{5}{3}\}\)}
				\cjfa{\((f - g)(x) = \frac{4x + 13}{(x - 3)(x + 2)}\)}
				\cjcontinue{\(\text{dom}(f - g) = \mathbb{R} \setminus \{-2, 3\}\)}
				\cjcontinue{\((\frac{f}{g})(x) = \frac{{(x + 2)}^2}{(x - 3)(x + 3)}\)}
				\cjcontinue{\(\text{dom}(\frac{f}{g}) = \mathbb{R} \setminus \{-3, -2, 3\}\)}
				\cjcontinue{\((fg)(x) = \frac{x + 3}{x - 3}\)}
				\cjcontinue{\(\text{dom}(fg) = \mathbb{R} \setminus \{-2, 3\}\)}
				\cjcontinue{\((f \circ g)(x) = \frac{3x + 7}{-2x - 3}\)}
				\cjcontinue{\(\text{dom}(f \circ g) = \mathbb{R} \setminus \{-2, -\frac{3}{2}\}\)}
				\cjcontinue{\((h \circ f)(x) = \sqrt{\frac{2x + 9}{x - 3}}\)}
				\cjcontinue{\(\text{dom}(h \circ f) = (-\infty, -\frac{9}{2}] \cup (3, +\infty)\)}
				\cjcontinue{\((g \circ h)(x) = \frac{(\sqrt{3x - 1} + 3)(\sqrt{3x - 1} - 2)}{3x - 5}\)}
				\cjcontinue{\(\text{dom}(g \circ h) = [\frac{1}{3}, +\infty) \setminus \{\frac{5}{3}\}\)}
			\end{cjsolution}

		\cjitem{}
			A ball is thrown upward from the roof of a building that is 30 meters high. If it
			is known that the position of the ball with respect to the ground after \(x\) seconds is
			\(h(x) = -5x^2 + 20x + 30\) meters, determine the maximum height reached by the ball, and
			how long it takes before the ball hits the ground.
			\begin{cjsolution}
				\cjsolsect{Let}
				\cjgiven{\(M\)}{Maximum height.}
				\cjgiven{\(t\)}{Time to hit the ground.}
				\cjwhy{\(-5x^2 + 20x + 30 = y\)}{The trajectory of a ball is most likely a parabola. Rewrite \(h(x)\) in standard form and in terms of \(y\).}
				\cjcontinue{\(-5x^2 + 20x = y - 30\)}
				\cjcontinue{\(-5(x^2 - 4x) = y - 30\)}
				\cjcontinue{\(-5(x^2 - 4x + 4) = y - 30 - 5(4)\)}
				\cjcontinue{\(-5(x^2 - 4x + 4) = y - 30 - 20\)}
				\cjcontinue{\(-5(x^2 - 4x + 4) = y - 50\)}
				\cjcontinue{\(-5{(x - 2)}^2 = y - 50\)}
				\cjcontinue{\({(x - 2)}^2 = (-\frac{1}{5}){y - 50}\)}
				\cjcontinue{\({(x - 2)}^2 = 4(-\frac{1}{20}){y - 50}\)}
				\cjwhy{\(-5x^2 + 20x + 30 = 0\)}{The x-intercept of the function is the time is takes for the ball to hit the ground.}
				\cjsubwhy{\(x = \frac{-20 \pm \sqrt{{(-20)}^2 - 4(-5)(30)}}{2(-5)}\)}{Use the quadratic formula.}
				\cjcontinue{\(x = \frac{-20 \pm \sqrt{400 + 600}}{-10}\)}
				\cjcontinue{\(x = \frac{-20 \pm \sqrt{1000}}{-10}\)}
				\cjcontinue{\(x = \frac{-20 \pm \sqrt{100}\sqrt{10}}{-10}\)}
				\cjcontinue{\(x = \frac{-20 \pm 10\sqrt{10}}{-10}\)}
				\cjcontinue{\(x = 2 \pm \sqrt{10}\)}
				\cjsubwhy{\(x = 2 + \sqrt{10}\)}{We only care about the cases where \(x \ge 0\) since we can't reverse time.}
				\cjfastep{\(M = 10\cjunit{meters}\)}{The maximum height is the y-coordinate of the vertex of the parabola.}
				\cjcontinue{\(t = 2 + \sqrt{10}\cjunit{seconds}\)}
				\cjcontinue{See Figure 2 for a visualization.}
			\end{cjsolution}
			\cjgraph{Figure 2. Zoom in to see labels.}{75}{
				\pstGeonode(2, 50){V}
				\def\p{-0.1}
				\pstParabola(V){\p}{-2}{16}
				\pstGeonode[PointName=x_i,PosAngle=45](5.16227, 0){xi}
			}
	\end{cjsection}
\end{document}
