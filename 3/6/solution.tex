% SPDX-FileCopyrightText: Copyright (C) Nile Jocson <novoseiversia@gmail.com>
% SPDX-License-Identifier: MPL-2.0

\documentclass{article}

% SPDX-FileCopyrightText: Copyright (C) Nile Jocson <novoseiversia@gmail.com>
% SPDX-License-Identifier: MPL-2.0

\usepackage{amsmath}
\usepackage{amssymb}
\usepackage[a4paper, margin=1in]{geometry}
\usepackage{hyperref}
\usepackage{physics}
\usepackage{tabularx}



\renewcommand{\arraystretch}{1.75}



\newcommand{\cjboilerplate}[2]{
	\renewcommand{\thesubsubsection}{\thesubsection.\alph{subsubsection}}

	\author{Nile Jocson \textless{}novoseiversia@gmail.com\textgreater{}}
	\title{Exercise Solutions for #1\\{\large #2}}
	\date{\today}

	\maketitle{}
		\pagebreak

	\tableofcontents{}
		\pagebreak
}



\newenvironment{cjsection}[1]
{
	\section{#1}
}
{
	\pagebreak
}

\newcommand{\cjitem}[1]{\subsection{#1}}
\newcommand{\cjsubitem}[1]{\subsubsection{#1}}



\newcommand{\cjnext}[1]{\(\Rightarrow\) #1}

\newcommand{\cjwhy}[2]{\hline \cjnext{#1} & #2 \\}
\newcommand{\cjcontinue}[1]{\cjnext{#1} & \\}

\newcommand{\cjqed}{\(\blacksquare\)}

\newenvironment{cjsolution}
{
	\tabularx{\textwidth}{| >{\raggedright\arraybackslash}X  >{\raggedleft\arraybackslash}X |}
}
{
		& \cjqed{} \\
		\hline
	\endtabularx
}



\newcommand*{\cjdiv}{\divisionsymbol{}}




\begin{document}
	\cjboilerplate{Math 20}{Sum, Difference, Cofunction, Double Measure Identities}

	\begin{cjsection}{Given \(\sin(\beta) = \frac{3}{5}\) where \(\beta\) lie in the interval \((\frac{\pi}{2}, 2\pi)\), find \(\cos(\frac{\beta}{2})\)}
		\begin{cjsolution}
			\cjwhy{\(\beta \in (\frac{\pi}{2}, \pi)\)}{Since \(\sin(\beta)\) is positive, \(\beta\) must be in \((0, \pi)\). Combining the two intervals, we can see that we are in QII.}
			\cjwhy{\(\cos(\beta) = \pm \sqrt{1 - \sin^2(\beta)}\)}{Find \(\cos(\theta)\) using Pythagoras.}
			\cjcontinue{\(\cos(\beta) = \pm \sqrt{1 - {(\frac{3}{5})}^2}\)}
			\cjcontinue{\(\cos(\beta) = \pm \sqrt{1 - \frac{9}{25}}\)}
			\cjcontinue{\(\cos(\beta) = \pm \sqrt{\frac{25}{25} - \frac{9}{25}}\)}
			\cjcontinue{\(\cos(\beta) = \pm \sqrt{\frac{16}{25}}\)}
			\cjcontinue{\(\cos(\beta) = \pm \frac{4}{5}\)}
			\cjsubwhy{\(\cos(\beta) = -\frac{4}{5}\)}{Since we are in QII, \(\cos(\beta)\) is negative.}
			\cjwhy{\(\cos(\frac{\beta}{2}) = \pm \sqrt{\frac{1 + \cos(\beta)}{2}}\)}{\(\cos(\frac{\theta}{2}) = \pm \sqrt{\frac{1 + \cos(\theta)}{2}}\)}
			\cjcontinue{\(\cos(\frac{\beta}{2}) = \pm \sqrt{\frac{1 - \frac{4}{5}}{2}}\)}
			\cjcontinue{\(\cos(\frac{\beta}{2}) = \pm \sqrt{\frac{\frac{5}{5} - \frac{4}{5}}{2}}\)}
			\cjcontinue{\(\cos(\frac{\beta}{2}) = \pm \sqrt{\frac{\frac{1}{5}}{2}}\)}
			\cjcontinue{\(\cos(\frac{\beta}{2}) = \pm \sqrt{\frac{1}{10}}\)}
			\cjcontinue{\(\cos(\frac{\beta}{2}) = \pm \frac{1}{\sqrt{10}}\)}
			\cjsubwhy{\(\cos(\frac{\beta}{2}) = \pm \frac{1}{\sqrt{10}} \cdot \frac{\sqrt{10}}{\sqrt{10}}\)}{Rationalize.}
			\cjcontinue{\(\cos(\frac{\beta}{2}) = \pm \frac{\sqrt{10}}{10}\)}
			\cjfastep{\(\cos(\frac{\beta}{2}) = \frac{\sqrt{10}}{10}\)}{Given \(\beta \in (\frac{\pi}{2}, \pi)\), \(\frac{\beta}{2} \in (\frac{\pi}{4}, \frac{\pi}{2})\), which is in QI. Therefore \(\cos(\frac{\beta}{2})\) is positive.}
		\end{cjsolution}
	\end{cjsection}



	\begin{cjsection}{Express the following as indicated then evaluate if possible.}
		\cjitem{\(\sin(105\degree)\cos(15\degree)\) as a sum.}
			\begin{cjsolution}
				\cjwhy{\(\sin(105\degree)\cos(15\degree) = \frac{1}{2}(\sin(105\degree + 15\degree) + \sin(105\degree - 15\degree))\)}{\(\sin(a)\cos(b) = \frac{1}{2}(\sin(a + b) + \sin(a - b))\)}
				\cjcontinue{\(\sin(105\degree)\cos(15\degree) = \frac{1}{2}(\sin(120\degree) + \sin(90\degree))\)}
				\cjcontinue{\(\sin(105\degree)\cos(15\degree) = \frac{1}{2}(\frac{\sqrt{3}}{2} + 1)\)}
				\cjcontinue{\(\sin(105\degree)\cos(15\degree) = \frac{\sqrt{3}}{4} + \frac{1}{2}\)}
				\cjcontinue{\(\sin(105\degree)\cos(15\degree) = \frac{\sqrt{3}}{4} + \frac{2}{4}\)}
				\cjfa{\(\sin(105\degree)\cos(15\degree) = \frac{2 + \sqrt{3}}{4}\)}
			\end{cjsolution}

		\cjitem{\(\cos(\frac{5\pi}{24}) + \cos(\frac{\pi}{24})\) as a product.}
			\begin{cjsolution}
				\cjwhy{\(\cos(\frac{5\pi}{24}) + \cos(\frac{\pi}{24}) = 2\cos(\frac{\frac{5\pi}{24} + \frac{\pi}{24}}{2})\cos(\frac{\frac{5\pi}{24} - \frac{\pi}{24}}{2})\)}{\(\cos(a) + \cos(b) = 2\cos(\frac{a + b}{2})\cos(\frac{a - b}{2})\)}
				\cjcontinue{\(\cos(\frac{5\pi}{24}) + \cos(\frac{\pi}{24}) = 2\cos(\frac{\frac{6\pi}{24}}{2})\cos(\frac{\frac{4\pi}{24}}{2})\)}
				\cjcontinue{\(\cos(\frac{5\pi}{24}) + \cos(\frac{\pi}{24}) = 2\cos(\frac{\frac{\pi}{4}}{2})\cos(\frac{\frac{\pi}{6}}{2})\)}
				\cjsubwhy{\(\cos(\frac{5\pi}{24}) + \cos(\frac{\pi}{24}) = 2\sqrt{\frac{1 + \cos(\frac{\pi}{4})}{2}}\sqrt{\frac{1 + \cos(\frac{\pi}{6})}{2}}\)}{\(\cos(\frac{\theta}{2}) = \pm \sqrt{\frac{1 + \cos(\theta)}{2}}\)}
				\cjcontinue{\(\cos(\frac{5\pi}{24}) + \cos(\frac{\pi}{24}) = 2\sqrt{\frac{1 + \frac{\sqrt{2}}{2}}{2}}\sqrt{\frac{1 + \frac{\sqrt{3}}{2}}{2}}\)}
				\cjcontinue{\(\cos(\frac{5\pi}{24}) + \cos(\frac{\pi}{24}) = 2\sqrt{\frac{1}{2} + \frac{\sqrt{2}}{4}}\sqrt{\frac{1}{2} + \frac{\sqrt{3}}{4}}\)}
				\cjcontinue{\(\cos(\frac{5\pi}{24}) + \cos(\frac{\pi}{24}) = 2\sqrt{\frac{2}{4} + \frac{\sqrt{2}}{4}}\sqrt{\frac{2}{4} + \frac{\sqrt{3}}{4}}\)}
				\cjcontinue{\(\cos(\frac{5\pi}{24}) + \cos(\frac{\pi}{24}) = 2\sqrt{\frac{2 + \sqrt{2}}{4}}\sqrt{\frac{2 + \sqrt{3}}{4}}\)}
				\cjcontinue{\(\cos(\frac{5\pi}{24}) + \cos(\frac{\pi}{24}) = 2(\frac{\sqrt{2 + \sqrt{2}}}{2})(\frac{\sqrt{2 + \sqrt{3}}}{2})\)}
				\cjfa{\(\cos(\frac{5\pi}{24}) + \cos(\frac{\pi}{24}) = \sqrt{2 + \sqrt{2}}(\frac{\sqrt{2 + \sqrt{3}}}{2})\)}
			\end{cjsolution}

		\cjitem{\(\sin(3x)\cos(2x)\) as a sum.}
			\begin{cjsolution}
				\cjwhy{\(\sin(3x)\cos(2x) = \frac{1}{2}(\sin(3x + 2x) + \sin(3x - 2x))\)}{\(\sin(a)\cos(b) = \frac{1}{2}(\sin(a + b) + \sin(a - b))\)}
				\cjcontinue{\(\sin(3x)\cos(2x) = \frac{1}{2}(\sin(5x) + \sin(x))\)}
				\cjfa{\(\sin(3x)\cos(2x) = \frac{\sin(5x) + \sin(x)}{2}\)}
			\end{cjsolution}
	\end{cjsection}



	\begin{cjsection}{Establish the following identities.}
		\cjitem{\(\frac{\cos(3x) + \cos(x)}{\sin(3x) + \sin(x)} = \frac{\cot(x) - \tan(x)}{2}\)}
			\begin{cjsolution}
				\cjwhy{\(\frac{\cot(x) - \tan(x)}{2} = \frac{2\cos(\frac{3x + x}{2})\cos(\frac{3x - x}{2})}{\sin(3x) + \sin(x)}\)}{\(\cos(a) + \cos(b) = 2\cos(\frac{a + b}{2})\cos(\frac{a - b}{2})\)}
				\cjcontinue{\(\frac{\cot(x) - \tan(x)}{2} = \frac{2\cos(\frac{4x}{2})\cos(\frac{2x}{2})}{\sin(3x) + \sin(x)}\)}
				\cjcontinue{\(\frac{\cot(x) - \tan(x)}{2} = \frac{2\cos(2x)\cos(x)}{\sin(3x) + \sin(x)}\)}
				\cjsubwhy{\(\frac{\cot(x) - \tan(x)}{2} = \frac{2\cos(2x)\cos(x)}{2\sin(\frac{3x + x}{2})\cos(\frac{3x - x}{2})}\)}{\(\sin(a) + \sin(b) = 2\sin(\frac{a + b}{2})\cos(\frac{a - b}{2})\)}
				\cjcontinue{\(\frac{\cot(x) - \tan(x)}{2} = \frac{2\cos(2x)\cos(x)}{2\sin(\frac{4x}{2})\cos(\frac{2x}{2})}\)}
				\cjcontinue{\(\frac{\cot(x) - \tan(x)}{2} = \frac{2\cos(2x)\cos(x)}{2\sin(2x)\cos(x)}\)}
				\cjcontinue{\(\frac{\cot(x) - \tan(x)}{2} = \frac{2\cos(2x)}{2\sin(2x)}\)}
				\cjsubwhy{\(\frac{\cot(x) - \tan(x)}{2} = \frac{2(\cos^2(x) - \sin^2(x))}{2\sin(2x)}\)}{\(\cos(2\theta) = \cos^2(\theta) - \sin^2(\theta)\)}
				\cjsubwhy{\(\frac{\cot(x) - \tan(x)}{2} = \frac{2(\cos^2(x) - \sin^2(x))}{2(2\sin(x)\cos(x))}\)}{\(\sin(2\theta) = 2\sin(\theta)\cos(\theta)\)}
				\cjcontinue{\(\frac{\cot(x) - \tan(x)}{2} = \frac{\cos^2(x) - \sin^2(x)}{2\sin(x)\cos(x)}\)}
				\cjcontinue{\(\frac{\cot(x) - \tan(x)}{2} = \frac{1}{2}(\frac{\cos^2(x) - \sin^2(x)}{\sin(x)\cos(x)})\)}
				\cjcontinue{\(\frac{\cot(x) - \tan(x)}{2} = \frac{1}{2}(\frac{\cos^2(x)}{\sin(x)\cos(x)} - \frac{\sin^2(x)}{\sin(x)\cos(x)})\)}
				\cjcontinue{\(\frac{\cot(x) - \tan(x)}{2} = \frac{1}{2}(\frac{\cos(x)}{\sin(x)} - \frac{\sin(x)}{\cos(x)})\)}
				\cjsubwhy{\(\frac{\cot(x) - \tan(x)}{2} = \frac{1}{2}(\cot(x) - \frac{\sin(x)}{\cos(x)})\)}{\(\cot(\theta) = \frac{\cos(\theta)}{\sin(\theta)}\)}
				\cjsubwhy{\(\frac{\cot(x) - \tan(x)}{2} = \frac{1}{2}(\cot(x) - \tan(x))\)}{\(\tan(\theta) = \frac{\sin(\theta)}{\cos(\theta)}\)}
				\cjfa{\(\frac{\cot(x) - \tan(x)}{2} = \frac{\cot(x) - \tan(x)}{2}\)}
			\end{cjsolution}

		\cjitem{\(2\sin^2(\frac{x}{2}) + \tan(\frac{x}{2}) = \frac{1 + \sin(x)}{\csc(x) + \cot(x)}\)}
			\begin{cjsolution}
				\cjwhy{\(\frac{1 + \sin(x)}{\csc(x) + \cot(x)} = 2\sin^2(\frac{x}{2}) + \frac{\sin(x)}{1 + \cos(x)}\)}{\(\tan(\frac{\theta}{2}) = \frac{\sin(\theta)}{1 + \cos(\theta)}\)}
				\cjwhy{\(\frac{1 + \sin(x)}{\csc(x) + \cot(x)} = 2(\frac{1 - \cos(x)}{2}) + \frac{\sin(x)}{1 + \cos(x)}\)}{\(\sin(\frac{\theta}{2}) = \pm \sqrt{\frac{1 - \cos(\theta)}{2}}\)}
				\cjcontinue{\(\frac{1 + \sin(x)}{\csc(x) + \cot(x)} = 1 - \cos(x) + \frac{\sin(x)}{1 + \cos(x)}\)}
				\cjcontinue{\(\frac{1 + \sin(x)}{\csc(x) + \cot(x)} = \frac{(1 - \cos(x))(1 + \cos(x))}{1 + \cos(x)} + \frac{\sin(x)}{1 + \cos(x)}\)}
				\cjsubwhy{\(\frac{1 + \sin(x)}{\csc(x) + \cot(x)} = \frac{1 - \cos^2(x)}{1 + \cos(x)} + \frac{\sin(x)}{1 + \cos(x)}\)}{Factor using difference of two squares.}
				\cjsubwhy{\(\frac{1 + \sin(x)}{\csc(x) + \cot(x)} = \frac{\sin^2(x)}{1 + \cos(x)} + \frac{\sin(x)}{1 + \cos(x)}\)}{\(\sin^2(\theta) + \cos^2(\theta) = 1\)}
				\cjcontinue{\(\frac{1 + \sin(x)}{\csc(x) + \cot(x)} = \frac{\sin^2(x) + \sin(x)}{1 + \cos(x)}\)}
				\cjcontinue{\(\frac{1 + \sin(x)}{\csc(x) + \cot(x)} = \frac{\sin(x)(1 + \sin(x))}{1 + \cos(x)}\)}
				\cjcontinue{\(\frac{1 + \sin(x)}{\csc(x) + \cot(x)} = \frac{1 + \sin(x)}{\frac{1 + \cos(x)}{\sin(x)}}\)}
				\cjcontinue{\(\frac{1 + \sin(x)}{\csc(x) + \cot(x)} = \frac{1 + \sin(x)}{\frac{1}{\sin(x)} + \frac{\cos(x)}{\sin(x)}}\)}
				\cjsubwhy{\(\frac{1 + \sin(x)}{\csc(x) + \cot(x)} = \frac{1 + \sin(x)}{\csc(x) + \frac{\cos(x)}{\sin(x)}}\)}{\(\csc(\theta) = \frac{1}{\sin(\theta)}\)}
				\cjfastep{\(\frac{1 + \sin(x)}{\csc(x) + \cot(x)} = \frac{1 + \sin(x)}{\csc(x) + \cot(x)}\)}{\(\cot(\theta) = \frac{\cos(\theta)}{\sin(\theta)}\)}
			\end{cjsolution}
	\end{cjsection}
\end{document}
