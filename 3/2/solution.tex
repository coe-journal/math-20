% SPDX-FileCopyrightText: Copyright (C) Nile Jocson <novoseiversia@gmail.com>
% SPDX-License-Identifier: MPL-2.0

\documentclass{article}

% SPDX-FileCopyrightText: Copyright (C) Nile Jocson <novoseiversia@gmail.com>
% SPDX-License-Identifier: MPL-2.0

\usepackage[a4paper, margin=1in]{geometry}

\usepackage{adjustbox}

\usepackage{amsmath}
\usepackage{amssymb}
\usepackage{physics}

\usepackage{tabularray}
\usepackage{tkz-tab}
\usepackage{xpatch}

\usepackage{pstricks-add}
\usepackage{pst-eucl}
\usepackage[crop=off]{auto-pst-pdf}



\renewcommand{\arraystretch}{1.75}
\renewcommand{\thesubsubsection}{\thesubsection.\alph{subsubsection}}



\DefTblrTemplate{caption}{default}{}
\DefTblrTemplate{capcont}{default}{}

\xpatchcmd{\tkzTabLine}{$0$}{$\bullet$}{}{}
\tikzset{t style/.style={style=solid}}



\newcommand*{\cjboilerplate}[2]{
	\psset{unit=5.5mm, ticks=none, xlabelsep=1pt, ylabelsep=1pt}

	\author{Nile Jocson \textless{}novoseiversia@gmail.com\textgreater{}}
	\title{Exercise Solutions for #1\\{\large #2}}
	\date{\today}

	\maketitle{}
	\null\vfill\noindent
	Copyright \copyright{} Nile Jocson \textless{}novoseiversia@gmail.com\textgreater{} \\
	Licensed under MPL-2.0. See LICENSE file.
		\pagebreak
}



\newenvironment{cjsection}[1]
{
	\section{#1}
}
{
	\pagebreak
}

\newcommand*{\cjitem}[1]{\subsection{#1}}
\newcommand*{\cjsubitem}[1]{\subsubsection{#1}}



\newcommand*{\cjsolsect}[1]{\hline --- #1: \\}
\newcommand*{\cjneeded}[2]{\hline --- Needed: \\ \(\square\) #1 \(= \mathord{?}\) & #2 \\}
\newcommand*{\cjgiven}[2]{\(\square\) #1 & #2 \\}

\newcommand*{\cjwhy}[2]{\hline \(\Rightarrow\) #1 & #2 \\}
\newcommand*{\cjsubwhy}[2]{\(\Rightarrow\) #1 & #2 \\}
\newcommand*{\cjcontinue}[1]{\(\Rightarrow\) #1 & \\}
\newcommand*{\cjfa}[1]{\hline \(\Rightarrow\) #1 & Final answer. \\}
\newcommand*{\cjfastep}[2]{\hline \(\Rightarrow\) #1 & Final answer. #2 \\}

\newcommand*{\cjsign}[1]{
	\hline & Create a table of signs. \\
	\begin{adjustbox}{width=0.49\textwidth}
		\begin{tikzpicture}
			#1
		\end{tikzpicture}
	\end{adjustbox} \\ \\
}

\newcommand*{\cjgraph}[3]{
	\begin{center}
		\begin{adjustbox}{width=\textwidth}
			\begin{pspicture*}(-#2,-#2)(#2,#2)
				\psaxes[labels=none]{<->}(0,0)(-#2,-#2)(#2,#2)
				#3
			\end{pspicture*}
		\end{adjustbox}
		#1
	\end{center}
	\pagebreak
}

\newcommand*{\cjsystem}[2]{
	\begin{equation*}
		#1
		\begin{cases}
			#2
		\end{cases}
	\end{equation*}
}

\newcommand*{\cjqed}{\(\blacksquare\)}



\NewDocumentEnvironment{cjsolution}{+b}
{
	\begin{longtblr}
	[
		expand = \cjwhy\cjsubwhy\cjcontinue\cjfa\cjfastep\cjsign\cjgiven\cjsolsect\cjneeded\cjsystem
	]
	{
		colspec = {|lX[r]|},
		width = \textwidth
	}
		#1
		& \cjqed{} \\
		\hline
	\end{longtblr}
}{}



\newcommand*{\cjdiv}{\divisionsymbol{}}
\newcommand*{\cjexp}[1]{\times 10^{#1}}
\newcommand*{\cjunit}[1]{\text{ #1}}
\newcommand*{\cjceil}[1]{\lceil#1\rceil}
\newcommand*{\cjlog}[2]{\text{log}_{#1} #2}




\begin{document}
	\cjboilerplate{Math 20}{Angles and Their Measure, Trigonometric Functions of Angles}

	\begin{cjsection}{}
		\cjitem{Complete the following table.}
			\begin{longtblr}{|c|c|c|}
				\hline rev & deg & rad \\
				\hline \(\frac{1}{20}\) & \(18\degree\) & \(\frac{\pi}{10}\) \\
				\hline \(-\frac{2}{3}\) & \(-240\degree\) & \(-\frac{4\pi}{3}\) \\
				\hline
			\end{longtblr}

		\cjitem{Marian eats one slice of a circular pie that is cut into six congruent
		slices. If the arclength of the side she ate is \(24\pi\) inches, what is the
		radius of the pie?}
			\begin{cjsolution}
				\cjwhy{\(\theta = \frac{2\pi}{6}\)}{The pie is sliced into 6 pieces.}
				\cjcontinue{\(\theta = \frac{\pi}{3}\)}
				\cjwhy{\(24\pi = r(\frac{\pi}{3})\)}{\(s = r\theta\)}
				\cjcontinue{\(r = \frac{24\pi}{\frac{\pi}{3}}\)}
				\cjcontinue{\(r = \frac{24\pi(3)}{\pi}\)}
				\cjfa{\(r = 72\cjunit{inches}\)}
			\end{cjsolution}

		\cjitem{If the terminal side of an angle \(\theta > 0\) contains the point \((1, -4\sqrt{3})\),
		find the six trigonometric functions of \(\theta\).}
			\begin{cjsolution}
				\cjwhy{\(r = \sqrt{1^2 + {(-4\sqrt{3})}^2}\)}{Use the Pythagorean Theorem to find \(r\).}
				\cjcontinue{\(r = \sqrt{1 + 16(3)}\)}
				\cjcontinue{\(r = \sqrt{1 + 48}\)}
				\cjcontinue{\(r = \sqrt{49}\)}
				\cjcontinue{\(r = 7\)}
				\cjfastep{\(\text{cos}(\theta) = \frac{1}{7}\)}{\(\text{cos}(\theta) = \frac{x}{r}\)}
				\cjsubwhy{\(\text{sin}(\theta) = -\frac{4\sqrt{3}}{7}\)}{\(\text{sin}(\theta) = \frac{y}{r}\)}
				\cjsubwhy{\(\text{tan}(\theta) = -4\sqrt{3}\)}{\(\text{sin}(\theta) = \frac{y}{x}\)}
				\cjsubwhy{\(\text{cot}(\theta) = -\frac{1}{4\sqrt{3}} = -\frac{\sqrt{3}}{12}\)}{\(\text{sin}(\theta) = \frac{x}{y}\)}
				\cjsubwhy{\(\text{sec}(\theta) = 7\)}{\(\text{sec}(\theta) = \frac{r}{x}\)}
				\cjsubwhy{\(\text{csc}(\theta) = -\frac{7}{4\sqrt{3}} = -\frac{7\sqrt{3}}{12}\)}{\(\text{csc}(\theta) = \frac{r}{y}\)}
			\end{cjsolution}

		\cjitem{Find the six trigonometric functions of \(\alpha\) if \(\text{cos}(\alpha) = -\frac{5}{13}\)
		and \(\alpha\) is in Quadrant III.}
			\begin{cjsolution}
				\cjwhy{\(x = -5, r = 13\)}{\(\text{cos}(\theta) = \frac{x}{r}\)}
				\cjsubwhy{\(y = -\sqrt{13^2 - {(-5)}^2}\)}{Use the Pythagorean Theorem to find \(y\). Since we're in Quadrant III, we want the negative case.}
				\cjcontinue{\(y = -\sqrt{169 - 25}\)}
				\cjcontinue{\(y = -\sqrt{144}\)}
				\cjcontinue{\(y = -12\)}
				\cjfastep{\(\text{sin}(\theta) = -\frac{12}{13}\)}{\(\text{sin}(\theta) = \frac{y}{r}\)}
				\cjsubwhy{\(\text{tan}(\theta) = \frac{-\frac{12}{13}}{-\frac{5}{13}} = \frac{12}{5}\)}{\(\text{tan}(\theta) = \frac{y}{x}\)}
				\cjsubwhy{\(\text{cot}(\theta) = \frac{5}{12}\)}{\(\text{cot}(\theta) = \frac{1}{\text{tan}(\theta)}\)}
				\cjsubwhy{\(\text{sec}(\theta) = -\frac{13}{5}\)}{\(\text{sec}(\theta) = \frac{1}{\text{cos}(\theta)}\)}
				\cjsubwhy{\(\text{csc}(\theta) = -\frac{13}{12}\)}{\(\text{csc}(\theta) = \frac{1}{\text{sin}(\theta)}\)}
			\end{cjsolution}

		\cjitem{Evaluate the following.}
			\cjsubitem{\(\text{csc}(315\degree)\)}
				\begin{cjsolution}
					\cjwhy{\(\overline{\theta} = 360\degree - 315\degree\)}{Find the reference angle.}
					\cjcontinue{\(\overline{\theta} = 45\degree\)}
					\cjsubwhy{\(\text{sin}(315\degree) = -\text{sin}(45\degree) = -\frac{\sqrt{2}}{2}\)}{Since \(315\degree\) is in QIV, the result of sin will be negative.}
					\cjfastep{\(\text{csc}(315\degree) = -\sqrt{2}\)}{\(\text{csc}(\theta) = \frac{1}{\text{sin}(\theta)}\)}
				\end{cjsolution}

			\cjsubitem{\(\text{cot}(420\degree)\)}
				\begin{cjsolution}
					\cjwhy{\(\theta = 60\degree\)}{Since \(420\degree > 360\degree\), find \(420\degree \text{ mod } 360\degree\).}
					\cjsubwhy{\(\text{cos}(60\degree) = \frac{1}{2}, \text{sin}(60\degree) = \frac{\sqrt{3}}{2}\)}{Find \(\text{cos}(\theta)\) and \(\text{sin}(\theta)\). Both will be positive since \(60\degree\) is in QI.}
					\cjsubwhy{\(\text{cot}(\theta) = \frac{\frac{1}{2}}{\frac{\sqrt{3}}{2}}\)}{\(\text{cot}(\theta) = \frac{\text{cos}(\theta)}{\text{sin}(\theta)}\)}
					\cjcontinue{\(\text{cot}(\theta) = \frac{2}{2\sqrt{3}}\)}
					\cjsubwhy{\(\text{cot}(\theta) = \frac{2\sqrt{3}}{2(3)}\)}{Rationalize.}
					\cjfa{\(\text{cot}(\theta) = \frac{\sqrt{3}}{3}\)}
				\end{cjsolution}

			\cjsubitem{\(\text{tan}(\frac{5}{8}\cjunit{rev})\text{cos}(660\degree)\)}
				\begin{cjsolution}
					\cjwhy{\(\theta_1 = \frac{5}{8}(2\pi)\)}{Convert from revolutions to radians.}
					\cjcontinue{\(\theta_1 = \frac{10\pi}{8}\)}
					\cjcontinue{\(\theta_1 = \frac{5\pi}{4}\)}
					\cjsubwhy{\(\text{tan}(\frac{5\pi}{4}) = 1\)}{Since \(\theta_1\) is in QIII, tan will be positive.}
					\cjwhy{\(\theta_2 = 300\)}{Find \(660\degree \text{ mod } 360\degree\).}
					\cjsubwhy{\(\overline{\theta_2} = 360 - 300\)}{Find the reference angle.}
					\cjcontinue{\(\overline{\theta_2} = 60\)}
					\cjsubwhy{\(\text{cos}(660) = \text{cos}(60) = \frac{1}{2}\)}{Since \(\theta_2\) is in QIV, cos will be positive.}
					\cjfa{\(\text{tan}(\frac{5}{8}\cjunit{rev})\text{cos}(660\degree) = \frac{1}{2}\)}
				\end{cjsolution}
	\end{cjsection}
\end{document}
