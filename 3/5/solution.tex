% SPDX-FileCopyrightText: Copyright (C) Nile Jocson <novoseiversia@gmail.com>
% SPDX-License-Identifier: MPL-2.0

\documentclass{article}

% SPDX-FileCopyrightText: Copyright (C) Nile Jocson <novoseiversia@gmail.com>
% SPDX-License-Identifier: MPL-2.0

\usepackage{amsmath}
\usepackage{amssymb}
\usepackage[a4paper, margin=1in]{geometry}
\usepackage{hyperref}
\usepackage{physics}
\usepackage{tabularx}



\renewcommand{\arraystretch}{1.75}



\newcommand{\cjboilerplate}[2]{
	\renewcommand{\thesubsubsection}{\thesubsection.\alph{subsubsection}}

	\author{Nile Jocson \textless{}novoseiversia@gmail.com\textgreater{}}
	\title{Exercise Solutions for #1\\{\large #2}}
	\date{\today}

	\maketitle{}
		\pagebreak

	\tableofcontents{}
		\pagebreak
}



\newenvironment{cjsection}[1]
{
	\section{#1}
}
{
	\pagebreak
}

\newcommand{\cjitem}[1]{\subsection{#1}}
\newcommand{\cjsubitem}[1]{\subsubsection{#1}}



\newcommand{\cjnext}[1]{\(\Rightarrow\) #1}

\newcommand{\cjwhy}[2]{\hline \cjnext{#1} & #2 \\}
\newcommand{\cjcontinue}[1]{\cjnext{#1} & \\}

\newcommand{\cjqed}{\(\blacksquare\)}

\newenvironment{cjsolution}
{
	\tabularx{\textwidth}{| >{\raggedright\arraybackslash}X  >{\raggedleft\arraybackslash}X |}
}
{
		& \cjqed{} \\
		\hline
	\endtabularx
}



\newcommand*{\cjdiv}{\divisionsymbol{}}




\begin{document}
	\cjboilerplate{Math 20}{Sum, Difference, Cofunction, Double Measure Identities}

	\begin{cjsection}{Evaluate the following without using a calculator.}
		\cjitem{\(\sin(\frac{19\pi}{12})\)}
			\begin{cjsolution}
				\cjwhy{\(\sin(\frac{10\pi}{12} + \frac{9\pi}{12})\)}{}
				\cjcontinue{\(\sin(\frac{5\pi}{6} + \frac{3\pi}{4})\)}
				\cjsubwhy{\(\sin(\frac{5\pi}{6})\cos(\frac{3\pi}{4}) + \cos(\frac{5\pi}{6})\sin(\frac{3\pi}{4})\)}{\(\sin(a + b) = \sin(a)\cos(b) + \cos(a)\sin(b)\)}
				\cjcontinue{\((\frac{1}{2})(-\frac{\sqrt{2}}{2}) + (-\frac{\sqrt{3}}{2})(\frac{\sqrt{2}}{2})\)}
				\cjcontinue{\(-\frac{\sqrt{2}}{4} - \frac{\sqrt{3}\sqrt{2}}{4}\)}
				\cjcontinue{\(-\frac{\sqrt{2}}{4} - \frac{\sqrt{6}}{4}\)}
				\cjfa{\(-\frac{\sqrt{2} + \sqrt{6}}{4}\)}
			\end{cjsolution}

		\cjitem{\(\cos(33\degree)\cos(27\degree) - \sin(33\degree)\sin(27\degree)\)}
			\begin{cjsolution}
				\cjwhy{\(\cos(33\degree + 27\degree)\)}{\(\cos(a + b) = \cos(a)\cos(b) - \sin(a)\sin(b)\)}
				\cjcontinue{\(\cos(60\degree)\)}
				\cjfa{\(\frac{1}{2}\)}
			\end{cjsolution}
	\end{cjsection}



	\begin{cjsection}{If \(\cot(\theta) = -\frac{5}{12}\) and \(\theta \in (-\frac{\pi}{2}, 0)\), find \(\cos(\theta + \frac{\pi}{3})\).}
		\begin{cjsolution}
			\cjwhy{\(O = -12, A = 5\)}{\(\cot(\theta) = \frac{A}{O}\), and since \(\theta \in (-\frac{\pi}{2}, 0)\), we are in QIV. Therefore, \(O < 0\) and \(A > 0\).}
			\cjsubwhy{\(H = \sqrt{{(-12)}^2 + 5^2}\)}{\(H = \sqrt{O^2 + A^2}\)}
			\cjcontinue{\(H = \sqrt{144 + 25}\)}
			\cjcontinue{\(H = \sqrt{169}\)}
			\cjcontinue{\(H = 13\)}
			\cjwhy{\(\cos(\theta + \frac{\pi}{3}) = \cos(\theta)\cos(\frac{\pi}{3}) - \sin(\theta)\sin(\frac{\pi}{3})\)}{\(\cos(a + b) = \cos(a)\cos(b) - \sin(a)\sin(b)\)}
			\cjsubwhy{\(\cos(\theta + \frac{\pi}{3}) = \frac{5}{13}\cos(\frac{\pi}{3}) - \sin(\theta)\sin(\frac{\pi}{3})\)}{\(\cos(\theta) = \frac{A}{H}\)}
			\cjsubwhy{\(\cos(\theta + \frac{\pi}{3}) = \frac{5}{13}\cos(\frac{\pi}{3}) + \frac{12}{13}\sin(\frac{\pi}{3})\)}{\(\sin(\theta) = \frac{O}{H}\)}
			\cjcontinue{\(\cos(\theta + \frac{\pi}{3}) = (\frac{5}{13})(\frac{1}{2}) + (\frac{12}{13})(\frac{\sqrt{3}}{2})\)}
			\cjcontinue{\(\cos(\theta + \frac{\pi}{3}) = \frac{5}{26} + \frac{12\sqrt{3}}{26}\)}
			\cjcontinue{\(\cos(\theta + \frac{\pi}{3}) = \frac{5 + 12\sqrt{3}}{26}\)}
		\end{cjsolution}
	\end{cjsection}



	\begin{cjsection}{Given \(\cos(\alpha) = \frac{3}{5}\) where \(\alpha\) lies in the interval \((\frac{\pi}{2}, 2\pi)\), find the following.}
		\cjitem{\(\sin(2\alpha)\)}
			\begin{cjsolution}
				\cjwhy{\(A = 3, H = 5\)}{\(\cos(\theta) = \frac{A}{H}\), and since \(\cos(\alpha) > 0\) and \(\alpha \in (\frac{\pi}{2}, 2\pi)\), we are in QIV. Therefore, \(O < 0\) and \(A > 0\).}
				\cjsubwhy{\(O = -\sqrt{5^2 - 3^2}\)}{From \(H = \sqrt{A^2 + O^2}\), we can derive \(O = \pm \sqrt{H^2 - A^2}\). Remember that in this case, \(A > 0\).}
				\cjcontinue{\(O = -\sqrt{25 - 9}\)}
				\cjcontinue{\(O = -\sqrt{16}\)}
				\cjcontinue{\(O = -4\)}
				\cjsubwhy{\(\sin(\alpha) = -\frac{4}{5}\)}{\(\sin(\theta) = \frac{O}{H}\)}
				\cjwhy{\(\sin(2\alpha) = 2(-\frac{4}{5})(\frac{3}{5})\)}{\(\sin(2\theta) = 2\sin(\theta)\cos(\theta)\)}
				\cjcontinue{\(\sin(2\alpha) = 2(-\frac{12}{25})\)}
				\cjfa{\(\sin(2\alpha) = -\frac{24}{25}\)}
			\end{cjsolution}

		\cjitem{\(\sin(3\alpha)\)}
			\begin{cjsolution}
				\cjwhy{\(\cos(2\alpha) = {(\frac{3}{5})}^2 - {(-\frac{4}{5})^2}\)}{\(\cos(2\theta) = \cos^2(\theta) - \sin^2(\theta)\)}
				\cjcontinue{\(\cos(2\alpha) = \frac{9}{25} - \frac{16}{25}\)}
				\cjcontinue{\(\cos(2\alpha) = -\frac{7}{25}\)}
				\cjwhy{\(\sin(3\alpha) = \sin(\alpha + 2\alpha)\)}{}
				\cjsubwhy{\(\sin(\alpha + 2\alpha) = (-\frac{4}{5})(-\frac{7}{25}) + (\frac{3}{5})(-\frac{24}{25})\)}{\(\sin(a + b) = \sin(a)\cos(b) + \cos(a)\sin(b)\)}
				\cjcontinue{\(\sin(\alpha + 2\alpha) = \frac{28}{125} - \frac{72}{125}\)}
				\cjcontinue{\(\sin(\alpha + 2\alpha) = \frac{44}{125}\)}
				\cjfa{\(\sin(3\alpha) = \frac{44}{125}\)}
			\end{cjsolution}
	\end{cjsection}



	\begin{cjsection}{Establish the following identities.}
		\cjitem{}
	\end{cjsection}
\end{document}
