% SPDX-FileCopyrightText: Copyright (C) Nile Jocson <novoseiversia@gmail.com>
% SPDX-License-Identifier: MPL-2.0

\documentclass{article}

% SPDX-FileCopyrightText: Copyright (C) Nile Jocson <novoseiversia@gmail.com>
% SPDX-License-Identifier: MPL-2.0

\usepackage[a4paper, margin=1in]{geometry}

\usepackage{adjustbox}

\usepackage{amsmath}
\usepackage{amssymb}
\usepackage{physics}

\usepackage{tabularray}
\usepackage{tkz-tab}
\usepackage{xpatch}

\usepackage{pstricks-add}
\usepackage{pst-eucl}
\usepackage[crop=off]{auto-pst-pdf}



\renewcommand{\arraystretch}{1.75}
\renewcommand{\thesubsubsection}{\thesubsection.\alph{subsubsection}}



\DefTblrTemplate{caption}{default}{}
\DefTblrTemplate{capcont}{default}{}

\xpatchcmd{\tkzTabLine}{$0$}{$\bullet$}{}{}
\tikzset{t style/.style={style=solid}}



\newcommand*{\cjboilerplate}[2]{
	\psset{unit=5.5mm, ticks=none, xlabelsep=1pt, ylabelsep=1pt}

	\author{Nile Jocson \textless{}novoseiversia@gmail.com\textgreater{}}
	\title{Exercise Solutions for #1\\{\large #2}}
	\date{\today}

	\maketitle{}
	\null\vfill\noindent
	Copyright \copyright{} Nile Jocson \textless{}novoseiversia@gmail.com\textgreater{} \\
	Licensed under MPL-2.0. See LICENSE file.
		\pagebreak
}



\newenvironment{cjsection}[1]
{
	\section{#1}
}
{
	\pagebreak
}

\newcommand*{\cjitem}[1]{\subsection{#1}}
\newcommand*{\cjsubitem}[1]{\subsubsection{#1}}



\newcommand*{\cjsolsect}[1]{\hline --- #1: \\}
\newcommand*{\cjneeded}[2]{\hline --- Needed: \\ \(\square\) #1 \(= \mathord{?}\) & #2 \\}
\newcommand*{\cjgiven}[2]{\(\square\) #1 & #2 \\}

\newcommand*{\cjwhy}[2]{\hline \(\Rightarrow\) #1 & #2 \\}
\newcommand*{\cjsubwhy}[2]{\(\Rightarrow\) #1 & #2 \\}
\newcommand*{\cjcontinue}[1]{\(\Rightarrow\) #1 & \\}
\newcommand*{\cjfa}[1]{\hline \(\Rightarrow\) #1 & Final answer. \\}
\newcommand*{\cjfastep}[2]{\hline \(\Rightarrow\) #1 & Final answer. #2 \\}

\newcommand*{\cjsign}[1]{
	\hline & Create a table of signs. \\
	\begin{adjustbox}{width=0.49\textwidth}
		\begin{tikzpicture}
			#1
		\end{tikzpicture}
	\end{adjustbox} \\ \\
}

\newcommand*{\cjgraph}[3]{
	\begin{center}
		\begin{adjustbox}{width=\textwidth}
			\begin{pspicture*}(-#2,-#2)(#2,#2)
				\psaxes[labels=none]{<->}(0,0)(-#2,-#2)(#2,#2)
				#3
			\end{pspicture*}
		\end{adjustbox}
		#1
	\end{center}
	\pagebreak
}

\newcommand*{\cjsystem}[2]{
	\begin{equation*}
		#1
		\begin{cases}
			#2
		\end{cases}
	\end{equation*}
}

\newcommand*{\cjqed}{\(\blacksquare\)}



\NewDocumentEnvironment{cjsolution}{+b}
{
	\begin{longtblr}
	[
		expand = \cjwhy\cjsubwhy\cjcontinue\cjfa\cjfastep\cjsign\cjgiven\cjsolsect\cjneeded\cjsystem
	]
	{
		colspec = {|lX[r]|},
		width = \textwidth
	}
		#1
		& \cjqed{} \\
		\hline
	\end{longtblr}
}{}



\newcommand*{\cjdiv}{\divisionsymbol{}}
\newcommand*{\cjexp}[1]{\times 10^{#1}}
\newcommand*{\cjunit}[1]{\text{ #1}}
\newcommand*{\cjceil}[1]{\lceil#1\rceil}
\newcommand*{\cjlog}[2]{\text{log}_{#1} #2}




\begin{document}
	\cjboilerplate{Math 20}{The Point Function and Circular Functions}

	\begin{cjsection}{Determine the quadrant where each of the following points lie.}
		\cjitem{\(P(\frac{23\pi}{5})\)}
			\begin{cjsolution}
				\cjwhy{\(4\pi + \frac{3\pi}{5}\)}{Rewrite.}
				\cjcontinue{\(\frac{3\pi}{5} \in (\pi, \frac{\pi}{2})\)}
				\cjfa{QII.}
			\end{cjsolution}

		\cjitem{\(P(-\frac{17\pi}{9})\)}
			\begin{cjsolution}
				\cjwhy{\(-\frac{17\pi}{9} \in (-\frac{3\pi}{2}, -2\pi)\)}{}
				\cjfa{QI.}
			\end{cjsolution}

		\cjitem{\(P(-2)\)}
			\begin{cjsolution}
				\cjwhy{\(-2 \in (-\frac{\pi}{2}, -\pi)\)}{\(\frac{\pi}{2} = 1.5708\)}
				\cjfa{QIII.}
			\end{cjsolution}

		\cjitem{\(P(4.71)\)}
			\begin{cjsolution}
				\cjwhy{\(4.71 \in (\pi, \frac{3\pi}{2})\)}{\(\frac{3\pi}{2} = 4.7124\)}
				\cjfa{QIII.}
			\end{cjsolution}
	\end{cjsection}



	\begin{cjsection}{Complete the following table.}
		\begin{longtblr}{colspec={|c|c|c|c|c|c|c|c|}}
			\hline \(\theta\) & \(P(\theta)\) & cos\(\theta\) & sin\(\theta\) & tan\(\theta\) & cot\(\theta\) & sec\(\theta\) & csc\(\theta\) \\

			\hline \(0\)
				& \((1, 0)\) & \(1\) & \(0\)
				& \(0\) & undefined
				& \(1\) & undefined \\
			\hline \(\frac{\pi}{6}\)
				& \((\frac{\sqrt{3}}{2}, \frac{1}{2})\) & \(\frac{\sqrt{3}}{2}\) & \(\frac{1}{2}\)
				& \(\frac{\sqrt{3}}{3}\) & \(\sqrt{3}\)
				& \(\frac{2\sqrt{3}}{3}\) & \(2\) \\
			\hline \(\frac{\pi}{4}\)
				& \((\frac{\sqrt{2}}{2}, \frac{\sqrt{2}}{2})\) & \(\frac{\sqrt{2}}{2}\) & \(\frac{\sqrt{2}}{2}\)
				& \(1\) & \(1\)
				& \(\sqrt{2}\) & \(\sqrt{2}\) \\
			\hline \(\frac{\pi}{3}\)
				& \((\frac{1}{2}, \frac{\sqrt{3}}{2})\) & \(\frac{1}{2}\) & \(\frac{\sqrt{3}}{2}\)
				& \(\sqrt{3}\) & \(\frac{\sqrt{3}}{3}\)
				& \(2\) & \(\frac{2\sqrt{3}}{3}\) \\

			\hline \(\frac{\pi}{2}\)
				& \((0, 1)\) & \(0\) & \(1\)
				& undefined & \(0\)
				& undefined & \(1\) \\
			\hline \(\frac{2\pi}{3}\)
				& \((-\frac{1}{2}, \frac{\sqrt{3}}{2})\) & \(-\frac{1}{2}\) & \(\frac{\sqrt{3}}{2}\)
				& \(-\sqrt{3}\) & \(-\frac{\sqrt{3}}{3}\)
				& \(-2\) & \(\frac{2\sqrt{3}}{3}\) \\
			\hline \(\frac{3\pi}{4}\)
				& \((-\frac{\sqrt{2}}{2}, \frac{\sqrt{2}}{2})\) & \(-\frac{\sqrt{2}}{2}\) & \(\frac{\sqrt{2}}{2}\)
				& \(-1\) & \(-1\)
				& \(-\sqrt{2}\) & \(\sqrt{2}\) \\
			\hline \(\frac{5\pi}{6}\)
				& \((-\frac{\sqrt{3}}{2}, \frac{1}{2})\) & \(-\frac{\sqrt{3}}{2}\) & \(\frac{1}{2}\)
				& \(-\frac{\sqrt{3}}{3}\) & \(-\sqrt{3}\)
				& \(-\frac{2\sqrt{3}}{3}\) & \(2\) \\

			\hline \(\pi\)
				& \((-1, 0)\) & \(-1\) & \(0\)
				& \(0\) & undefined
				& \(-1\) & undefined \\
			\hline \(\frac{7\pi}{6}\)
				& \((-\frac{\sqrt{3}}{2}, -\frac{1}{2})\) & \(-\frac{\sqrt{3}}{2}\) & \(-\frac{1}{2}\)
				& \(\frac{\sqrt{3}}{3}\) & \(\sqrt{3}\)
				& \(-\frac{2\sqrt{3}}{3}\) & \(-2\) \\
			\hline \(\frac{5\pi}{4}\)
				& \((-\frac{\sqrt{2}}{2}, -\frac{\sqrt{2}}{2})\) & \(-\frac{\sqrt{2}}{2}\) & \(-\frac{\sqrt{2}}{2}\)
				& \(1\) & \(1\)
				& \(-\sqrt{2}\) & \(-\sqrt{2}\) \\
			\hline \(\frac{4\pi}{3}\)
				& \((-\frac{1}{2}, -\frac{\sqrt{3}}{2})\) & \(-\frac{1}{2}\) & \(-\frac{\sqrt{3}}{2}\)
				& \(\sqrt{3}\) & \(\frac{\sqrt{3}}{3}\)
				& \(-2\) & \(-\frac{2\sqrt{3}}{3}\) \\

			\hline \(\frac{3\pi}{2}\)
				& \((0, -1)\) & \(0\) & \(-1\)
				& undefined & \(0\)
				& undefined & \(-1\) \\
			\hline \(\frac{5\pi}{3}\)
				& \((\frac{1}{2}, -\frac{\sqrt{3}}{2})\) & \(\frac{1}{2}\) & \(-\frac{\sqrt{3}}{2}\)
				& \(-\sqrt{3}\) & \(-\frac{\sqrt{3}}{3}\)
				& \(2\) & \(-\frac{2\sqrt{3}}{3}\) \\
			\hline \(\frac{7\pi}{4}\)
				& \((\frac{\sqrt{2}}{2}, -\frac{\sqrt{2}}{2})\) & \(\frac{\sqrt{2}}{2}\) & \(-\frac{\sqrt{2}}{2}\)
				& \(-1\) & \(-1\)
				& \(\sqrt{2}\) & \(-\sqrt{2}\) \\
			\hline \(\frac{11\pi}{6}\)
				& \((\frac{\sqrt{3}}{2}, -\frac{1}{2})\) & \(\frac{\sqrt{3}}{2}\) & \(-\frac{1}{2}\)
				& \(-\frac{\sqrt{3}}{3}\) & \(-\sqrt{3}\)
				& \(\frac{2\sqrt{3}}{3}\) & \(-2\) \\

			\hline \(2\pi\)
				& \((1, 0)\) & \(1\) & \(0\)
				& \(0\) & undefined
				& \(1\) & undefined \\
			\hline
		\end{longtblr}
	\end{cjsection}
\end{document}
