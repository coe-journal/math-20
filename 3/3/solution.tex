% SPDX-FileCopyrightText: Copyright (C) Nile Jocson <novoseiversia@gmail.com>
% SPDX-License-Identifier: MPL-2.0

\documentclass{article}

% SPDX-FileCopyrightText: Copyright (C) Nile Jocson <novoseiversia@gmail.com>
% SPDX-License-Identifier: MPL-2.0

\usepackage{amsmath}
\usepackage{amssymb}
\usepackage[a4paper, margin=1in]{geometry}
\usepackage{hyperref}
\usepackage{physics}
\usepackage{tabularx}



\renewcommand{\arraystretch}{1.75}



\newcommand{\cjboilerplate}[2]{
	\renewcommand{\thesubsubsection}{\thesubsection.\alph{subsubsection}}

	\author{Nile Jocson \textless{}novoseiversia@gmail.com\textgreater{}}
	\title{Exercise Solutions for #1\\{\large #2}}
	\date{\today}

	\maketitle{}
		\pagebreak

	\tableofcontents{}
		\pagebreak
}



\newenvironment{cjsection}[1]
{
	\section{#1}
}
{
	\pagebreak
}

\newcommand{\cjitem}[1]{\subsection{#1}}
\newcommand{\cjsubitem}[1]{\subsubsection{#1}}



\newcommand{\cjnext}[1]{\(\Rightarrow\) #1}

\newcommand{\cjwhy}[2]{\hline \cjnext{#1} & #2 \\}
\newcommand{\cjcontinue}[1]{\cjnext{#1} & \\}

\newcommand{\cjqed}{\(\blacksquare\)}

\newenvironment{cjsolution}
{
	\tabularx{\textwidth}{| >{\raggedright\arraybackslash}X  >{\raggedleft\arraybackslash}X |}
}
{
		& \cjqed{} \\
		\hline
	\endtabularx
}



\newcommand*{\cjdiv}{\divisionsymbol{}}




\begin{document}
	\cjboilerplate{Math 20}{Graphs of Circular Functions}

	\begin{cjsection}{From the following sine waves, determine the amplitude, period, phase shift,
	and vertical shift and then sketch one cycle of the graph. Label maximum and minimium points.}
		\cjitem{\(f(x) = 2\text{cos}(\frac{1}{2}(x + \pi)) - 1\)}
			\begin{cjsolution}
				\cjfastep{\(a = 2\)}{\(f(x) = a\text{cos}(b(x - c)) + d\)}
				\cjcontinue{\(b = \frac{1}{2}\)}
				\cjcontinue{\(c = -\pi\)}
				\cjcontinue{\(d = -1\)}
				\cjsubwhy{\(M = 1\)}{\(\{M, m\} = \pm a - d\)}
				\cjcontinue{\(m = -3\)}
				\cjcontinue{See Figure 1.}
			\end{cjsolution}
			\cjgraph{Figure 1. Graph of \(f(x) = 2\text{cos}(\frac{1}{2}(x + \pi)) - 1\)}{16}{
				\psplot[algebraic,plotpoints=500]{-16}{16}{2 * cos(0.5 * (x + 3.1416)) - 1}
				\pstGeonode[PosAngle=90](-3.1416, 1){M}
				\pstGeonode[PosAngle=-90](3.1416, -3){m}
			}

		\cjitem{\(f(x) = -\frac{3}{2}\text{sin}(\pi - 2x) + 2\)}
			\begin{cjsolution}
				\cjwhy{\(f(x) = -\frac{3}{2}\text{sin}(-2(x - \frac{\pi}{2})) + 2\)}{Rewrite in standard form.}
				\cjfa{\(a = -\frac{3}{2}\)}
				\cjcontinue{\(b = -2\)}
				\cjcontinue{\(c = \frac{\pi}{2}\)}
				\cjcontinue{\(d = 2\)}
				\cjcontinue{\(M = \frac{7}{2}\)}
				\cjcontinue{\(m = \frac{1}{2}\)}
			\end{cjsolution}
			\cjgraph{Figure 2. Graph of \(f(x) = -\frac{3}{2}\text{sin}(\pi - 2x) + 2\)}{16}{
				\psplot[algebraic,plotpoints=500]{-16}{16}{-1.5 * sin(-2 * (x - 1.5708)) + 2}
				\pstGeonode[PosAngle=90](-0.7854, 3.5){M}
				\pstGeonode[PosAngle=-90](0.7854, 0.5){m}
			}

		\cjitem{\(f(x) = 4\text{sin}(4x - 3\pi)\)}
			\begin{cjsolution}
				\cjwhy{\(f(x) = 4\text{sin}(4(x - \frac{3\pi}{4}))\)}{Rewrite in standard form.}
				\cjfa{\(a = 4\)}
				\cjcontinue{\(b = 4\)}
				\cjcontinue{\(c = \frac{3\pi}{4}\)}
				\cjcontinue{\(d = 0\)}
				\cjcontinue{\(M = 4\)}
				\cjcontinue{\(m = -4\)}
			\end{cjsolution}
			\cjgraph{Figure 3. Graph of \(f(x) = 4\text{sin}(4x - 3\pi)\)}{16}{
				\psplot[algebraic,plotpoints=1000]{-16}{16}{4 * sin(4 * (x - 2.3562))}
				\pstGeonode[PosAngle=90](-0.3927, 4){M}
				\pstGeonode[PosAngle=-90](0.3927, -4){m}
			}

		\cjitem{\(f(x) = -2\text{cos}(\frac{x}{2} - \frac{\pi}{4}) + 1\)}
			\begin{cjsolution}
				\cjwhy{\(f(x) = -2\text{cos}(\frac{1}{2}(x - \frac{\pi}{2})) + 1\)}{Rewrite in standard form.}
				\cjfa{\(a = -2\)}
				\cjcontinue{\(b = \frac{1}{2}\)}
				\cjcontinue{\(c = \frac{\pi}{2}\)}
				\cjcontinue{\(d = 1\)}
				\cjcontinue{\(M = 3\)}
				\cjcontinue{\(m = -1\)}
			\end{cjsolution}
			\cjgraph{Figure 3. Graph of \(f(x) = -2\text{cos}(\frac{x}{2} - \frac{\pi}{4}) + 1\)}{16}{
				\psplot[algebraic,plotpoints=500]{-16}{16}{-2 * cos(0.5 * (x - 1.5708)) + 1}
				\pstGeonode[PosAngle=90](-4.7124, 3){M}
				\pstGeonode[PosAngle=-90](1.5708, -1){m}
			}
	\end{cjsection}
\end{document}
